\RequirePackage[T1]{fontenc}
\documentclass[12pt]{article}

\usepackage[height=8.85in,width=6.45in]{geometry}
%\usepackage{showkeys}
\renewcommand{\baselinestretch}{1.08}

\usepackage[utf8]{inputenc}
\usepackage{amsmath}
\usepackage{amssymb}
\usepackage{mathtools}
\numberwithin{equation}{section}
\usepackage{slashed}
\usepackage{braket}
\usepackage[svgnames,psnames]{xcolor}
\usepackage[colorlinks,citecolor=DarkGreen,linkcolor=FireBrick,linktocpage]{hyperref}
\usepackage{cite}
\usepackage{graphicx}
\usepackage{tikz}
\usepackage{tikz-cd}
\usepackage{times}
\usepackage{courier}
\usepackage{bm}
\usepackage{subfig}
\usepackage{ascmac}

\usepackage{mdframed}
\newenvironment{claim}{  \begin{mdframed}[linecolor=black!0,backgroundcolor=black!10]\noindent\itshape\ignorespaces}{\end{mdframed}}

\let\originalfigure=\figure
\let\endoriginalfigure=\endfigure
\let\originaltable=\table
\let\endoriginaltable=\endtable

\renewenvironment{figure}[1][]{
  \begin{originalfigure}[#1]
    \begin{mdframed}[linecolor=black!0,backgroundcolor=black!1]
}{
    \end{mdframed}
  \end{originalfigure}
}
\renewenvironment{table}[1][]{
  \begin{originaltable}[#1]
    \begin{mdframed}[linecolor=black!0,backgroundcolor=black!1]
}{
    \end{mdframed}
  \end{originaltable}
}%% Comment
\newcommand{\comment}[1]{\textcolor{red}{[#1]}}

%% Color
\usepackage{colortbl}
\definecolor{lightyellow}{rgb}{1.0, 0.95, 0.7}
\definecolor{lightblue}{rgb}{0.7, 0.9, 1.0}
\definecolor{lightpink}{rgb}{1.0, 0.85, 0.95}
\definecolor{lightgreen}{rgb}{0.7, 1.0, 0.4}

%% Yuji's macros
%%list HigherSymmetriesAndDuality
\newcommand*{\bZ}{\mathbb{Z}}
\newcommand*{\bR}{\mathbb{R}}
\newcommand*{\dint}{\displaystyle\int}
\newcommand*{\cP}{\mathcal{P}}
\newcommand*{\fP}{\mathfrak{P}}
\def\Nequals#1{$\mathcal{N}{=}#1$}
\def\SO{\mathrm{SO}}
\def\so{\mathfrak{so}}
\def\Spin{\mathrm{Spin}}

\def\boo{0.0}

\def\xlattice#1#2#3{
\begin{tikzpicture}[scale=.5]
\filldraw[color=black!5!white](-.5,-.5) rectangle (1.5,1.5);
\draw[->] (-1,0) -- (2,0);
\draw[->] (0,-1) -- (0,2);
\foreach \x in {0,1} {
	\foreach \y in {0,1}{
		\pgfmathsetmacro\a{mod(#1 * \x - #2 * \y,2)}
		\ifx\a\boo
			\filldraw[color=#3] (\x,\y) circle (.5em);
		\else
			\filldraw[fill=white,draw=gray] (\x,\y) circle (.5em);
		\fi
	}
}
\end{tikzpicture}
}

\begin{document}

\begin{titlepage}

\begin{flushright}
IPMU-19-0124
\end{flushright}

\vskip 3cm

\begin{center}

{\Large \bfseries Matching higher symmetries across Seiberg duality}


\vskip 1cm
Yasunori Lee$^1$, Kantaro Ohmori$^2$, and Yuji Tachikawa$^1$
\vskip 1cm

\begin{tabular}{ll}
$^1$ & Kavli Institute for the Physics and Mathematics of the Universe (WPI), \\
& University of Tokyo,  Kashiwa, Chiba 277-8583, Japan\\
$^2$ & Simons Center for Geometry and Physics, \\
& SUNY, Stony Brook, NY 11794, USA
\end{tabular}


\vskip 1cm

\end{center}


\noindent 
We study the generalized symmetries of 4d \Nequals1 supersymmetric $\mathfrak{so}(2n_c)$ gauge theory with $N_f=2n_f$ flavors and their anomalies.
We find that they depend on the parity of $n_c$ and $n_f$ 
and also on the choice of the global form of the gauge group and the discrete theta angle,
and that they agree across Seiberg duality.
In particular, when $n_c$ is odd and $n_f$ is even, 
the $\bZ_2$ one-form symmetry nontrivially extends the 0-form flavor symmetry to form a 2-group for $\Spin$ or $\SO_-$ theories,
while the $\bZ_2$ one-form symmetry and the 0-form flavor symmetry remains a direct product but has a nontrivial mixed anomaly for $\SO_+$ theories.


\end{titlepage}

\setcounter{tocdepth}{2}
\tableofcontents

%\newpage

\section{Introduction and summary}
Our understanding of the concept of symmetries in quantum field theories has been greatly improved in the last several years.
We now have the concept of $p$-form symmetries acting on $p$-dimensional objects \cite{Gaiotto:2014kfa}.
This concept  gives a unifying point of view to both
ordinary symmetries acting on point operators for $p=0$
and center symmetries of gauge theories acting on Wilson line operators for $p=1$.
In addition, the 't Hooft magnetic flux \cite{tHooft:1979rtg} can now be thought of as a background gauge field for the 1-form center symmetry.
It is also realized more recently that 0-form symmetries and 1-form symmetries can not only coexist in a direct product but also mix in a more intricate manner, defining a symmetry structure called 2-groups \cite{Cordova:2018cvg,Benini:2018reh}.

These new concepts can be used very effectively to constrain the dynamics of various strongly-coupled systems,
as demonstrated beautifully in \cite{Gaiotto:2017yup} and later works inspired by it.
These were mostly done on non-supersymmetric quantum field theories,
for which the concept of generalized symmetries is a big new addition to the few analytical tools already available.
This should not, however, stop us to apply this new tool to the study of supersymmetric systems.
There are indeed  many existing techniques to study them exactly,
but the concept of generalized symmetries might shed a somewhat different light on them from a different angle.

In this short note we intend to initiate such a study, 
by having a look at the generalized symmetries of 4d \Nequals1 supersymmetric $\mathfrak{so}(2n_c)$ QCD with $N_f=2n_f$ flavors.
It is  a classic fact \cite{Intriligator:1995id} that this theory has a Seiberg dual description as an $\mathfrak{so}(2n_c')$ QCD with $N_f=2n_f$ flavors with additional singlets and a superpotential term, where $n_c'=n_f+2-n_c$.
To fully specify the theory, one needs to pick the global structure of the gauge group and the discrete theta angle.
In the case of $\mathfrak{so}$ QCD,
there are three choices\footnote{%
Here we made a simplifying assumption that the spacetime has a spin structure,
the gauge group and the spacetime symmetry forms a direct product,
and the gauge group is connected.
}, namely $\Spin(2n_c)$, $\SO(2n_c)_+$ and $\SO(2n_c)_-$,
and the Seiberg duality exchanges $\Spin$ and $\SO_-$ while keeping $\SO_+$ fixed \cite{Aharony:2013hda}.
This action of the Seiberg duality was deduced by studying the behavior of line operators acted on by a $\bZ_2$ 1-form symmetry,
and was given a more detailed check by a study of the supersymmetric index on the lens space \cite{Razamat:2013opa}.
Our concrete objective in this paper is to understand the interplay of this $\bZ_2$ 1-form symmetry and the 0-form flavor symmetry acting on $N_f=2n_f$ flavors.

More concretely, we find that interesting phenomena happen when $n_f$ is even and $n_c$ is odd.
We will find that when the gauge group is either $\Spin$ or $\SO_-$, the 0-form flavor symmetry is extended by the $\bZ_2$ one-form symmetry to become a genuine 2-group.
We will also find that when the gauge group is $\SO_+$, the 0-form flavor symmetry
and the $\bZ_2$ one-form symmetry remains a direct product
but that there is a mixed anomaly between them.
We will see that this distribution of the extension and the anomaly is consistent with the action of Seiberg duality.
We will also see that it is consistent also under the gauging of one-form symmetries to go back and forth among various choices of the gauge groups. 

The rest of this short note contains two sections. 
In Sec.~\ref{sec:known}, 
we recall the properties of $\bZ_2$ 1-form symmetries of the $\Spin$, $\SO_+$ and $\SO_-$ quantum chromodynamics,
and the properties of groups whose Lie algebra is $\so(2n_c)$.
Then in Sec.~\ref{sec:unknown},
we study whether the $\bZ_2$ 1-form symmetry extends the 0-form flavor symmetry,
or there is a mixed anomaly between them,
in the three cases of $\Spin$, $\SO_+$ and $\SO_-$.
Finally we check that our findings are consistent with the action of Seiberg duality.

Before proceeding we note that except at the very end we do not use any property of Seiberg duality.
The bulk of the paper is about the study of the global structure formed by the $\bZ_2$ 1-form symmetry and the 0-form flavor symmetry and its anomaly in each of the three choices.
We will also see that the cross-check provided by our analysis on the Seiberg duality of $\so$ theories is not trivial but not strong either.
That said,  the authors believe that it can be fruitfully applied to many other known and unknown \Nequals1 duality pairs.

\section{Known facts}
\label{sec:known}
Let us quickly summarize known facts we will make use of in the rest of the paper.
We are interested in $\mathfrak{so}$ gauge theories with matter fields in the vector representation.

\subsection{Line operators and the three choices $\Spin/\SO_+/\SO_-$}
\label{sec:lines}
Consider an electric Wilson line operator in the spinor representation $W$.
Its square is in the vector representation, and can be screened by the dynamical matter field.
This means that the electric lines can be labeled by  $\bZ_2$.
Let us next consider a magnetic 't Hooft line operator $H$,
which has a minimal unit of allowed magnetic charge with respect to the matter in the vector representation.
It is characterized by a nontrivial Stiefel-Whitney class $w_2\in H^2(S^2,\bZ_2)$ of  the gauge bundle on the $S^2$ around the line operator,
meaning that the magnetic lines can also be labeled by $\bZ_2$.
In total the topological classification of the line operators in an $\mathfrak{so}$ theory with matter fields in the vector representation is given by $\bZ_2\times \bZ_2=\{1,W\}\times \{1,H\}$,
but $W$ and $H$ are mutually non-local.


\begin{figure}
\centering
\begin{tabular}{c@{\qquad}c@{\qquad}c}
\xlattice01{Red} & \xlattice10{Green} & \xlattice11{Blue} \\
$\Spin$ & $\SO_+$ & $\SO_-$
\end{tabular}
\caption{The line operators present in $\Spin$, $\SO_+$, $\SO_-$ quantum chromodynamics.
The horizontal  axis and the vertical axis measure the electric charge and the magnetic charge, respectively.
In all cases, the group of lines is a $\bZ_2$ subgroup of the $\bZ_2\times \bZ_2$ group of possibly mutually-nonlocal lines of $\mathfrak{so}$ quantum chromodynamics.
The $\Spin$, $\SO_+$, $\SO_-$ theory contains an electric line,
a magnetic line, and a dyonic line, respectively.
\label{fig:lines}
}
\end{figure}

To fully specify a quantum field theory, we need to pick a maximal set of mutually-local line operators.
There are three choices: $\{1,W\}$, $\{1,H\}$ and $\{1,D:=WH\}$.
The first choice is the $\Spin$ gauge theory, and the second is the standard $\SO$ gauge theory.
The existence of the third choice was more recently noticed
and is given by an $\SO$ gauge theory with a discrete theta term \begin{equation}
\exp\left(2\pi i \frac14 \int_X \mathfrak{P}(w_2)\right)
\label{pont}
\end{equation} 
in the exponentiated action, where $\mathfrak{P}(w_2)\in H^4(X,\bZ_4)$ is the Pontryagin square of the Stiefel-Whitney class $w_2\in H^2(X,\bZ_2)$ of the gauge bundle on the spacetime $X$.
When $X$ is spin as we assume throughout in the paper,
the integral of $\mathfrak{P}(w_2)$ is even,
producing a sign in the integrand of the path integral.
This makes the 't Hooft line to acquire an electric charge $W$.
For details, we refer the reader to \cite{Aharony:2013hda}.
We summarized the choice of line operators in Fig.~\ref{fig:lines}.

It is a classic result in the study of 4d \Nequals1 supersymmetric theories
that the Seiberg dual of the $\so(N_c)$ gauge theory with $N_f$ flavors
is given by an $\so(N_c')$ gauge theory with  $N_f$ flavors 
with an additional gauge singlet meson fields and a superpotential term,
where $N_c'=N_f+4-N_c$ \cite{Intriligator:1995id}.
In \cite{Aharony:2013hda} the action of the Seiberg duality was refined to take into account the three possibilities $\Spin$, $\SO_+$ and $\SO_-$;
it was found there that $\SO_+$ is self-dual while $\Spin$ and $\SO_-$ are exchanged.
This was indirectly argued in \cite{Aharony:2013hda} from the behavior of the line operators,
and was later given a more direct check via the computation of the supersymmetric index on the lens spaces in \cite{Razamat:2013opa}.

\subsection{Facts concerning $\so(2n)$}
\label{sec:math}
We are going to refine the understanding reviewed above when $N_c$ and $N_f$ are both even.
For this we need to recall a few facts concerning $\so(2n)$.
This is mostly to set the notations;
a reader might want to skip the rest of this section and come back only when an unfamiliar symbol is used.

The simply-connected group $\Spin(2n)$ has a center  of order four,
given by $\bZ_4=\{0,s,v=s^2,c=s^3\}$ when $n$ is odd
and $\bZ_2\times \bZ_2=\{0,s\}\times \{0,c\}$ when $n$ is even;
we define $v=s+c$ for the latter case.
The representations of $\Spin(2n)$ can be classified into four classes $0$, $S$, $C$, $V$
according to how the center acts on them.


\begin{figure}
\[
\begin{tikzpicture}[scale=.5,baseline=(SO.base)]
\node(Spin) at (0,0) {$\Spin(2n)$};
\node(Ss) at (-8,-2) {$\mathrm{Ss}(2n)$};
\node(SO) at (0,-2) {$\SO(2n)$};
\node(Ss') at (+8,-2) {$\mathrm{Ss}'(2n)$};
\node(PSO) at (0,-4) {$\SO(2n)/\bZ_2$};
\node(legend) at (0,-6) {$n$ even};
\draw[->] (Spin) to node[above]{\footnotesize $v_2{=}w_2'$}  (Ss); 
\draw[->] (Spin) to node[left]{\footnotesize $w_2{=}w_2'$}  (SO); 
\draw[->] (Spin) to node[above]{\footnotesize $v_2{=}w_2$}  (Ss'); 
\draw[->] (Ss) to node[below]{\footnotesize $w_2$} (PSO);
\draw[->] (SO) to node[left]{\footnotesize $v_2$} (PSO);
\draw[->] (Ss') to node[below]{\footnotesize $w_2'$} (PSO);
\end{tikzpicture}
\qquad\qquad
\begin{tikzpicture}[scale=.5,baseline=(SO.base)]
\node(Spin) at (0,0) {$\Spin(2n)$};
\node(SO) at (0,-2) {$\SO(2n)$};
\node(PSO) at (0,-4) {$\SO(2n)/\bZ_2$};
\node(legend) at (0,-6) {$n$ odd};
\draw[->] (Spin) to node[left]{\footnotesize $w_2$}  (SO); 
\draw[->] (SO) to node[left]{\footnotesize $v_2$} (PSO);
\draw[->,dashed,out=-10,in=10] (Spin.east) to node[right]{\footnotesize $\tilde v_2$}  (PSO.east); 
\end{tikzpicture}
\]
\caption{%
Various quotients of $\Spin(2n)$ and the characteristic classes associated to their bundles. 
Solid arrows are $\bZ_2$ quotients and the dashed arrow is a $\bZ_4$ quotient.
Each arrow is labeled by the corresponding obstruction classes.
\label{fig:classes}}
\end{figure}

The quotient of $\Spin(2n)$ by $\bZ_2=\{0,v\}$ is $\SO(2n)$.
Accordingly, an $\SO(2n)$ bundle $A$ on a space $X$ has the second Stiefel-Whitney class $w_2(A)\in H^2(X,\bZ_2)$, which is trivial if and only if the bundle $A$ lifts to a $\Spin(2n)$ bundle.
The group $\SO(2n)$ has the center $\bZ_2=\{0,s=c\}$,
and we can form a further quotient $\SO(2n)/\bZ_2$.
Analogously to the case just mentioned above,
an $\SO(2n)/\bZ_2$ bundle $A'$ on a space $X$ has an obstruction class in $H^2(X,\bZ_2)$, which is trivial if and only if the bundle $A'$ lifts to an $\SO(2n)$ bundle.
There does not seem to be a common name for this class; 
we denote it by $v_2(A')\in H^2(X,\bZ_2)$.

When $n$ is odd, $\pi_1(\SO(2n)/\bZ_2)=\bZ_4$, and therefore
there is an obstruction class $\tilde v_2(A')\in H^2(X,\bZ_4)$ which controls if $A'$ lifts all the way to $\Spin(2n)$,
and $v_2(A')$ is simply the mod-2 reduction of $\tilde v_2(A')$.
When $n$ is even, $\pi_1(\SO(2n)/\bZ_2)=\bZ_2\times \bZ_2$.
Correspondingly, we have two classes $w_2(A')$, $w_2'(A')\in H^2(X,\bZ_2)$
controlling whether the bundle $A'$ lifts to 
an $\mathrm{Ss}(2n)=\Spin(2n)/\{0,s\}$ bundle or  
an $\mathrm{Ss}'(2m)=\Spin(2n)/\{0,c\}$ bundle.\footnote{%
$\mathrm{Ss}$ stands for semi-spin.
}
Here we have  $v_2(A')=w_2(A')+w_2'(A')$.
The information are summarized in Fig.~\ref{fig:classes}.


\section{The mixed anomaly and the 2-group}
\label{sec:unknown}
Let us now consider $\so(2n_c)$ gauge theory with $2n_f$ flavors in the vector representation.
We chose the number of colors $N_c$ to be even, $N_c=2n_c$,
to see the effect of the larger center of $\Spin(2n_c)$ which is order 4.
When the matter fields are massless, the system has $\mathfrak{su}(2n_f)$ flavor symmetry,
but we choose to use only the $\mathfrak{so}(2n_f)$ sub-symmetry, which is compatible with a nonzero mass term.
Our motivation to let the number of flavors $N_f$ to be even, $N_f=2n_f$, is again the same.

Now the matter field is in the bivector representation of $\so(2n_c)$ and $\so(2n_f)$.
Let us then take the global form of the combined gauge and flavor group to be 
\begin{equation}
\frac{\Spin(2n_c)\times \SO(2n_f)}{\bZ_2},\quad
\frac{\SO(2n_c)_+\times \SO(2n_f)}{\bZ_2},\quad
\frac{\SO(2n_c)_-\times \SO(2n_f)}{\bZ_2} 
\label{foo}
\end{equation}
where the quotient is in general with respect to the diagonal subgroup of $\bZ_2$ centers which act trivially on the bivector representation, 
with the exception that we consider \begin{equation}
\frac{\Spin(2n_c)\times \SO(2n_f)}{\bZ_4}
\label{bar}
\end{equation}
when $n_c$  is odd, for which we use the $\bZ_4$ subgroup generated by $(1,1) \in \{0,1,2,3\}\times \{0,1\}$.

The 0-form part of the flavor symmetry is then $\SO(2n_f)/\bZ_2$,
and we are interested in its interrelationship with the $\bZ_2$ 1-form symmetries of the gauge part we reviewed in Sec.~\ref{sec:lines}.

\subsection{Appearance of 2-group in the $\Spin(2n_c)$ theory}
Let us start by analyzing the charge of the line operators of the $\Spin(2n_c)$ theory.
An important step in our brief review in Sec.~\ref{sec:lines}
was to declare that the Wilson line in the vector representation is considered topologically trivial
since there are dynamical fields in the vector representation.
We now take into account the flavor symmetry.
The dynamical fields are in the bivector representation of $\so(2n_c)$ and $\so(2n_f)$.
This means that a gauge Wilson line in the vector of $\so(2n_c)$
and a flavor Wilson line in the vector of $\so(2n_f)$ are considered topologically identical.

Consider then a gauge Wilson line $W$ in the spinor of $\so(2n_c)$,
which is charged under the $\bZ_2$ one-form symmetry of the $\Spin(2n_c)$ theory.
When $n_c$ is odd, $W^2$ is in the vector representation of $\so(2n_c)$,
which is then equivalent to a flavor Wilson line in the vector of $\so(2n_f)$.
Let us furthermore assume that $n_f$ is even.
Then there is no flavor Wilson line such that its square is in the vector of $\so(2n_f)$.
We find that a flavor Wilson line in the vector of $\so(2n_f)$,
which is not divisible by two if we only consider flavor Wilson lines,
becomes divisible by two if we consider gauge Wilson lines.
This means that the $\bZ_2$ 1-form symmetry extends the 0-form flavor symmetry nontrivially.

To see why, consider an ordinary group extension \begin{equation}
0\to \bZ_2\to \bZ_8 \to \bZ_4 \to 0.
\end{equation} Taking the Pontryagin duals (i.e.~the groups formed by one-dimensional representations), we have \begin{equation}
0\to \hat\bZ_4\to \hat\bZ_8 \to \hat\bZ_2 \to 0.
\end{equation}
In words,  a charge-1 representation of $\bZ_4$ corresponds to a charge-2 representation of $\bZ_8$,  which is clearly divisible by 2 as a $\bZ_8$ representation.
In our case, the role of $\bZ_2$ is played by the 1-form $\bZ_2$ symmetry
and the role of $\bZ_4$ is played by the flavor symmetry $\SO(2n_f)/\bZ_2$.
The extension can be written as \begin{equation}
0\to \bZ_2[1] \to  H \to \SO(2n_f)/\bZ_2 \to 0
\label{2group}
\end{equation} where $[1]$ after $\bZ_2$ is to signify it is a one-form symmetry;
$H$ is a nontrivial extension of a continuous flavor symmetry by a discrete one-form symmetry,
and is an example of a genuine 2-group.
We note that in general an extension \begin{equation}
0\to A[p] \to H \to G \to 0
\end{equation} of a 0-form symmetry $G$ by an Abelian $p$-form symmetry $A$ is specified by a class \begin{equation}
e\in H^{2+p} (BG,A)
\end{equation} known as the Postnikov class of the extension;
this reduces to the ordinary extension class in $H^2(BG,A)$ in the case of an extension of a 0-form symmetry by another 0-form symmetry.
We will determine $e$ below.

The appearance of a nontrivial 2-group can also be seen from the point of view of the background fields coupled to the symmetry.
This point of view allows us to find the Postnikov class of the extension.
The background $E\in H^2(X,\bZ_2)$ for the $\bZ_2$ one-form symmetry of the $\Spin(2n_c)$ theory sets the Stiefel-Whitney class $w_2(c)\in H^2(X,\bZ_2)$ of the $\SO(2n_c)$ gauge bundle to be $E=w_2(c)$.
In the presence of a nontrivial $\SO(2n_f)/\bZ_2$ background $f$, however, 
the gauge bundle is not necessarily an $\SO(2n_c)$ bundle.
Rather, it is in general an $\SO(2n_c)/\bZ_2$ bundle such that $v_2(c)=v_2(f)$,
where $v_2$ is the obstruction class controlling the lift from $\SO/\bZ_2$ to $\SO$.

When $n_c$ is even this poses no problem since the classes $v_2$ and $w_2$ exist independently in $\SO(2n_c)/\bZ_2$.
When $n_c$ is odd, however, there is a problem, since the $(\bZ_2)_{w_2}$ controlling $\Spin(2n_c)\to \SO(2n_c)$
and $(\bZ_2)_{v_2}$ controlling $\SO(2n_c) \to \SO(2n_c)/\bZ_2$ form a nontrivial extension \begin{equation}
0\to (\bZ_2)_{w_2} \to (\bZ_4)_{\tilde v_2} \to (\bZ_2)_{v_2} \to 0.
\end{equation}
This is reflected in the fact that the quotient is by $\bZ_4$ in \eqref{bar} when $n_c$ is odd.
In this case, what is guaranteed to exist is the $\bZ_4$ class $\tilde v_2(c) \in H^2(X,\bZ_4)$.

The data contained in $\tilde v_2(c)$ can be encoded by a closed cochain $v_2(c)\in H^2(X,\bZ_2)$
and  a not-necessarily-closed cochain $w_2(c)\in C^2(X,\bZ_2)$ satisfying \begin{equation}
\delta w_2(c) = \beta v_2(c).\label{asin}
\end{equation} where $\delta$ is the coboundary and $\beta$ is the Bockstein.
Recalling $v_2(c)=v_2(f)$, and $E=w_2(c)$, we see that the background fields for the $\Spin(2n_c)$ theory are given by a pair of an $\SO(2n_f)/\bZ_2$ bundle $f$
and a cochain $E\in C^2(X,\bZ_2)$ satisfying the relation \begin{equation}
\delta E= \beta v_2(f).
\label{ext}
\end{equation}
This means that the 2-group extension \eqref{2group} is specified by the extension class \begin{equation}
\beta v_2 \in H^3(B\SO(2n_f)/\bZ_2, \bZ_2).
\end{equation}
Recall that  the Bockstein $\beta x$ of a mod-2 class $x$ is defined so that it controls whether $x$ is a mod-2 reduction of the $\bZ_4$.
When $n_f$ is odd, $v_2$ is a mod-2 reduction of the $\bZ_4$ class $\tilde v_2$ controlling the lift from $\SO(2n_f)/\bZ_2$  all the way to $\Spin(2n_f)$.
Therefore the extension class $\beta v_2$ is trivial when $n_f$ is odd,
whereas it  is nontrivial when $n_f$ is even.
We conclude that we have a nontrivial 2-group only when $n_c$ is odd and $n_f$ is even.

\subsection{Mixed anomaly in the $\SO(2n_c)_+$ theory}
Let us next consider the $\SO(2n_c)_+$ theory.
Here, the charged objects under the $\bZ_2$ one-form symmetry are the 't Hooft loops $H$.
The background field $B\in H^2(X,\bZ_2)$ of the $\bZ_2$ one-form symmetry then enters in the exponentiated action as \begin{equation}
\exp\left( 2\pi i \frac12 \int_X B w_2(c) \right)\label{magnetic}
\end{equation} where $c$ is the $\SO(2n_c)$ gauge bundle. 

We now introduce a background bundle $f$  for the flavor symmetry $\SO(2n_f)/\bZ_2$.
Recall that the matter fields are in the bivector of $\SO(2n_c)\times \SO(2n_f)$.
Therefore, the gauge bundle $c$ is in general an $\SO(2n_c)/\bZ_2$ bundle, such that $v_2(c)=v_2(f)$.
This does not pose a problem when $n_c$ is even, since $w_2$ and $v_2$ exist independently as a $\bZ_2$-valued class, as we recalled in Sec.~\ref{sec:math}.
When $n_c$ is odd, however, there is a potential problem, since $w_2(c)$ is no longer a closed cochain in general, but satisfies $\delta w_2(c)=\beta v_2(c)=\beta v_2(f)$ as in \eqref{asin}.
Therefore we find that \begin{equation}
\delta (B w_2(c)) = B \beta v_2(f).
\end{equation}
This means that the non-closed-ness of the coupling \eqref{magnetic}
is purely given by a combination of the background fields,
and can be cancelled by putting the 4d theory on the boundary $X$ of the 5d bulk $Y$ with the coupling \begin{equation}
\exp\left(2\pi i \frac12 \int_Y B \beta v_2(f)\right)
\label{anom}
\end{equation}
We again note that $\beta v_2(f)=0$ when $n_f$ is odd while it is nontrivial when $n_f$ is even.
We should also pose here to mention that the analysis up to this point shows that
the $\bZ_2$ one-form symmetry and the $\SO(2n_f)/\bZ_2$ flavor symmetry remain a direct product,
and do not form a nontrivially-extended 2-group.

Summarizing, we found that the $\SO(2n_c)_+$ theory has a mixed anomaly when $n_c$ is odd and $n_f$ is even, and anomaly-free otherwise.
The anomaly is between the $\bZ_2$ 1-form symmetry and the $\SO(2n_f)/\bZ_2$ 0-form flavor symmetry  and is given by \eqref{anom}.

Before continuing, we recall that the $\SO(2n_c)_+$ theory is obtained by gauging the electric $\bZ_2$ 1-form symmetry (with the background field $E$) of the $\Spin(2n_c)$ theory,
while the $\Spin(2n_c)$ theory is obtained by gauging the magnetic $\bZ_2$ 1-form symmetry (with the background field $B$) of the $\SO(2n_c)_+$ theory \cite{Kapustin:2014gua}. 
In Sec.~3.2 of \cite{Tachikawa:2017gyf}, it was already shown that when the electric background $E$ satisfies the extension of the form \eqref{ext}, the magnetic background $B$ have the anomaly of the form \eqref{anom}, 
where a general element $\Omega\in H^3(BG,\bZ_2)$ was used instead of our specific choice $\beta v_2(f)$.
The analysis in \cite{Tachikawa:2017gyf} was general but no concrete example was provided.
The new point here is that a 4d $\so$ quantum chromodynamics provides a concrete example.

\newpage

\subsection{Appearance of 2-group in the $\SO(2n_c)_-$ theory}
Let us finally study the $\SO(2n_c)_-$ theory.
This theory has a $\bZ_2$ 1-form symmetry,
whose charged object is the dyonic line $WH$.
Its square $(WH)^2$ when $n_c$ is odd is a gauge Wilson line in the vector representation,
which is equivalent to a flavor Wilson line in the vector representation.
This behavior is the same as the $\Spin(2n_c)$ theory with odd $n_c$,
and therefore we have the same nontrivial extension \eqref{2group}:
\begin{equation}
0\to \bZ_2[1] \to  H \to \SO(2n_f)/\bZ_2 \to 0.
\end{equation}
The extension should then be given by \eqref{ext}: \begin{equation}
\delta B= \beta v_2(f),
\label{ext-}
\end{equation} where we denoted the background field for $\bZ_2$ 1-form symmetry by $B$.

Let us see how the relation \eqref{ext-} can be understood from the point of view of the Lagrangian.
The issue is the discrete theta angle term \eqref{pont}.
Together with the magnetic coupling to $B$, we have the following contribution to the exponentiated action \begin{equation}
\exp\left(2\pi i\int_X ( \frac14 \mathfrak{P}(w_2(c)) +\frac12 B w_2(c))\right).
\label{boo}
\end{equation}
When $n_c$ is odd, $w_2(c)$ is not necessarily closed.
Then the second term in \eqref{boo} is not closed, 
and to even talk about the first term in \eqref{boo}, one first needs to extend the definition of the Pontryagin square $\fP$ to non-closed cochains.

Let us show that the coupling \eqref{boo} can be made a well-defined, non-anomalous coupling after an addition of a counterterm, when $B$ satisfies the extension condition \eqref{ext-}.
To do this, we recall that the non-closed-ness of $w_2(c)$ is governed by the \begin{equation}
\delta w_2(c)=\beta v_2(c)= \beta v_2(f), 
\end{equation} see \eqref{asin}.
Combining with \eqref{ext-}, we see $w_2(c)+B$  is closed. 
Therefore we can always consider the well-defined coupling \begin{equation}
\exp \left(2\pi i \frac14 \int_X \fP(w_2(c)+B)\right) .
\label{aho}
\end{equation}
In the special case when $B$ and $w_2(c)$ are closed, we can use the standard formula $\fP(x+y)=\fP(x)+\fP(y)+2 xy$ to show that the expression \eqref{aho} equals
\begin{equation}
= \exp\left(2\pi i \int_X 
 ( \frac14 \mathfrak{P}(w_2(c)) +\frac12 B w_2(c) +\frac14 \mathfrak{P}(B) )
\right) .
\label{fPB}
\end{equation}

This means that the well-defined expression \eqref{aho} equals the naive expression \eqref{boo} 
up to an addition of the Pontryagin square of the background field $B$ and the terms which vanish when $ w_2(c)$ and/or $B$ are closed.
This is what we wanted to demonstrate.
Note that the $\SO_-$ theory with a general 0-form symmetry background necessarily involves the pure background term $\fP(B)$ in it.
We will come back to this point in Sec.~\ref{sec:last}.

\begin{table}
\centering
\begin{tabular}{ll}
				$\Spin$: &
\begin{tabular}{l|cc}
	 & $n_c$ even & $n_c$ odd\\
		\hline
		$n_f$ even   & -- &  extension \\
		$n_f$ odd    & -- & --
\end{tabular} \\[2em]
 $\SO_+$:  &
\begin{tabular}{l|cc}
		 & $n_c$ even & $n_c$ odd\\
		\hline
		$n_f$ even   & -- &  anomaly \\
		$n_f$ odd    & -- & --
\end{tabular} \\[2em]
$\SO_-$: &
\begin{tabular}{l|cc}
				 & $n_c$ even & $n_c$ odd\\
		\hline
		$n_f$ even   & -- &  extension \\
		$n_f$ odd    & -- & --
\end{tabular} 
\end{tabular} 
\caption{The extension and the anomaly of the $\so(2n_c)$ gauge theory with $N_f=2n_f$ flavors, in the three cases $\Spin$, $\SO_+$ and $\SO_-$.
The symbol -- signifies that there is neither nontrivial extension nor nontrivial anomaly.
\label{tab}}
\end{table}

\subsection{Summary}
Let us summarize what we found. 
We studied the interrelationship of the $\bZ_2$ 1-form symmetry of the $\Spin(2n_c)$, $\SO(2n_c)_+$ and $\SO(2n_c)_-$ theories and 
their $\SO(2n_f)/\bZ_2$ flavor symmetry. 
We found the  behavior summarized in Table~\ref{tab}.

The Seiberg duality maps $\mathfrak{so}(N_c)$ with $N_f$ flavors with $\mathfrak{so}(N_c')$ with $N_f$ flavors,
where $N_c'=N_f+4-N_c$.
In terms of $N_c=2n_c$ and $N_f=2n_f$, this means $n_c'=n_f+2-n_c$.
Therefore it acts on the parity of $n_c$ and $n_f$  as 
\begin{center}
	\begin{tabular}{l|cc}
		& $n_c$ even & $n_c$ odd\\
		\hline
		$n_f$ even   & $\circlearrowleft$ & $\circlearrowleft$\\
		$n_f$ odd    & \multicolumn{2}{c}{$\xleftrightarrow{\qquad\ }$}
	\end{tabular},
\end{center}
and we also know that $\Spin$ and $\SO_-$ are exchanged while $\SO_+$ is self-dual.
We see that the information summarized in Table~\ref{tab} is consistent with this action of the Seiberg duality.


\subsection{A comment on $\Spin\leftrightarrow\SO_-$ duality}
\label{sec:last}
Let us consider the situation with a general flavor background $f$ for the 0-form $\SO(2n_f)/\bZ_2$ symmetry.
The analysis in this paper shows, at least when $n_f$ is even and $n_c$ is odd, 
that the $\SO_-$ theory requires the coupling $\fP(B)$ to be present in the action, 
see the discussions around \eqref{fPB}.
Conversely, we can \emph{not} introduce the coupling $\fP(E)$ in the dual $\Spin$ theory, 
since $E$ is not closed in general and $\fP(E)$ alone does not make sense.
This means the following, even after turning off the 0-form flavor background $f$, at least when $n_f$ is even and $n_c$ is odd:
\begin{claim}
The $\Spin\leftrightarrow\SO_-$ duality maps
the $\Spin$ theory \emph{without} the $\fP(E)$ discrete theta angle to
the $\SO$ theory \emph{with} the $\fP(B)$ discrete theta angle,
where the background fields $E$ and $B$ are of course identified across the duality.
\end{claim}
Let us show that this property holds generally under the Seiberg duality between the $\Spin(N_c)$ theory with $N_f$ flavors and the $\SO(N_c')_-$ theory with $N_f$ flavors, where $N_c+N_c'=N_f+4$,
without any restriction on the values of $N_c$ and $N_f$.

To see this, we first give a large non-zero mass term to the flavors in the electric description,
and perform the corresponding deformation on the magnetic side.
The system is in the confined phase. 
We put the system in a large compact spacetime, and 
study the dependence of the phase of the partition function on the background field $E$ or $B$ for the 1-form symmetry.

Consider the $\Spin$ theory without the $\fP(E)$ discrete theta term.
Recall that the 1-form symmetry background $E$ is the Stiefel-Whitney class $w_2(c)$ of the gauge bundle $c$,
which is physically known as the 't Hooft flux. 
The system is confining, and therefore the 't Hooft flux is screened,
and the phase of the partition function on a large compact spacetime is independent of it, i.e.~
\begin{equation}
\text{$Z^\Spin(E=w_2(c))$ is independent of $E$.}
\label{AAA}
\end{equation}

Consider next the $\SO_-$ theory without the $\fP(B)$ discrete theta term.
The partition function is of the form \begin{equation}
	Z^{\SO_-}(B)\propto \sum_{w_2(c)}  Z^\Spin(w_2(c)) e^{2\pi i \int (\frac14 \mathfrak{P}(w_2(c)) +\frac12 B w_2(c)) }
	 \propto e^{2\pi i \frac14 \int \mathfrak{P}(B)} \label{BBB}
\end{equation} which also means that  \begin{equation}
\text{$Z^{\SO_-}(B) e^{2\pi i \frac14 \mathfrak{P}(B)} $ is independent of $B$.}
\label{CCC}
\end{equation}
Comparing \eqref{AAA}, \eqref{BBB} and \eqref{CCC},
we see that the dual of the $\Spin$ theory without $\fP(E)$ is the $\SO_-$ theory with $\fP(B)$.

This simple argument does not determine how the $\SO_+$ theories with and without $\fP(B)$ interaction are mapped among themselves.
This is because \begin{equation}
	Z^{\SO_+}(B)\propto \sum_{w_2(c)}  Z^\Spin(w_2(c)) e^{2\pi i \int \frac12 B w_2(c) } 
	 \propto \delta(B)
\end{equation} and therefore both $Z^{\SO_+}(B)$ 
and $Z^{\SO_+}(B) e^{2\pi i \frac14 \int \fP(B)}$ 
are proportional to a delta function $\delta(B)$ of the background field $B$.


\section*{Acknowledgements}
Y.L.~is partially supported by the Programs for Leading Graduate Schools, MEXT, Japan, via the Leading Graduate Course for Frontiers of Mathematical Sciences and Physics. 
Y.T.~is partially supported  by JSPS KAKENHI Grant-in-Aid (Wakate-A), No.17H04837 
and JSPS KAKENHI Grant-in-Aid (Kiban-S), No.16H06335,
and also by WPI Initiative, MEXT, Japan at IPMU, the University of Tokyo.

\bibliographystyle{ytphys}
\baselineskip=.95\baselineskip
\bibliography{ref}

\end{document}