\pdfoutput=1
\RequirePackage[T1]{fontenc}
\documentclass[12pt]{article}

\usepackage[height=8.85in,width=6.45in]{geometry}
%\usepackage{showkeys}
\renewcommand{\baselinestretch}{1.05}

\usepackage[utf8]{inputenc}
\usepackage{amsmath}
\usepackage{amssymb}
\usepackage{mathtools}
\numberwithin{equation}{section}
\usepackage{slashed}
\usepackage{braket}
\usepackage[svgnames,psnames]{xcolor}
\usepackage[colorlinks,citecolor=DarkGreen,linkcolor=FireBrick,urlcolor=FireBrick,linktocpage,unicode]{hyperref}
\urlstyle{rm}
%\usepackage[colorlinks,citecolor=black,linkcolor=black]{hyperref}
\usepackage{cite}
\usepackage{graphicx}
\usepackage{tikz}
\usepackage{tikz-cd}
\newcommand{\tikzmark}[1]{\tikz[remember picture,overlay]\node (#1){};}
\tikzset{>=stealth}

\usepackage{times}
\usepackage{courier}
\usepackage{bm}
\usepackage{subfig}

\usepackage{xcolor}
\usepackage{mdframed}
\newenvironment{claim}{  \begin{mdframed}[linecolor=black!0,backgroundcolor=black!10]\noindent\itshape\ignorespaces}{\end{mdframed}}

\let\originalfigure=\figure
\let\endoriginalfigure=\endfigure

\renewenvironment{figure}[1][]{
  \begin{originalfigure}[#1]
    \begin{mdframed}[linecolor=black!0,backgroundcolor=black!1]
}{
    \end{mdframed}
  \end{originalfigure}
}
%% Comment
\newcommand{\comment}[1]{\textcolor{red}{[#1]}}

\usepackage{ascmac}
\usepackage{spectralsequences}
\usepackage{dashbox}

\DeclareSseqGroup\tower {} {
	\class(0,0)\foreach \i in {1,...,8} {
		\class(0,\i)
		\structline(0,\i-1,-1)(0,\i,-1)
	}
}

\usepackage{xcolor}
\usepackage{colortbl}
\usepackage{arydshln}
\definecolor{lightyellow}{rgb}{1.0, 0.95, 0.7}
\definecolor{blue}{rgb}{0.0, 0.4, 1.0}
\definecolor{Blue}{rgb}{0,0,1}
\definecolor{darkgreen}{rgb}{0.,0.6,0.}
\newcommand*{\red}[1]{\textcolor{red}{#1}}
\newcommand*{\Blue}[1]{\textcolor{Blue}{#1}}
\newcommand*{\blue}[1]{\textcolor{blue}{#1}}
\newcommand*{\green}[1]{\textcolor{darkgreen}{#1}}
\newcommand*{\black}[1]{\textcolor{black}{#1}}

\definecolor{colorA}{rgb}{1,0,0}
\definecolor{colorB}{rgb}{0,0.3,1}
\definecolor{colorC}{rgb}{0.9,0.8,0.2}
\definecolor{colorD}{rgb}{0,0.65,0}

\newcommand*{\colorA}[1]{\textcolor{colorA}{#1}}
\newcommand*{\colorB}[1]{\textcolor{colorB}{#1}}
\newcommand*{\colorC}[1]{\textcolor{colorC}{#1}}
\newcommand*{\colorD}[1]{\textcolor{colorD}{#1}}

\let\tilde\widetilde

%% Yuji's macros
%%list SeibergDual



%% Color
\usepackage{colortbl}
\definecolor{lightyellow}{rgb}{1.0, 0.95, 0.7}
\definecolor{lightblue}{rgb}{0.7, 0.9, 1.0}
\definecolor{lightpink}{rgb}{1.0, 0.85, 0.95}
\definecolor{lightgreen}{rgb}{0.7, 1.0, 0.4}

\def\Nequals#1{$\mathcal{N}{=}#1$}
\def\bR{\mathbb{R}}
\def\bQ{\mathbb{Q}}
\def\bZ{\mathbb{Z}}
\def\MF{\mathrm{MF}}
\def\TMF{\mathrm{TMF}}
\def\Tmf{\mathrm{Tmf}}
\def\tmf{\mathrm{tmf}}
\def\tr{\mathop{\mathrm{tr}}}

\def\fP{\mathfrak{P}}

\def\U{\mathrm{U}}
\def\SU{\mathrm{SU}}
\def\SO{\mathrm{SO}}
\def\USp{\mathrm{USp}}
\def\u{\mathfrak{u}}
\def\su{\mathfrak{su}}
\def\so{\mathfrak{so}}
\def\usp{\mathfrak{usp}}
\def\Spin{\mathrm{Spin}}
\def\SL{\mathrm{SL}}
\def\boo{0.0}
\def\xlattice#1#2#3{
\begin{tikzpicture}[scale=.5]
\filldraw[color=black!5!white](-.5,-.5) rectangle (1.5,1.5);
\draw[->] (-1,0) -- (2,0);
\draw[->] (0,-1) -- (0,2);
\foreach \x in {0,1} {
	\foreach \y in {0,1}{
		\pgfmathsetmacro\a{mod(#1 * \x - #2 * \y,2)}
		\ifx\a\boo
			\filldraw[color=#3] (\x,\y) circle (.5em);
		\else
			\filldraw[fill=white,draw=gray] (\x,\y) circle (.5em);
		\fi
	}
}
\end{tikzpicture}
}

\begin{document}



\begin{titlepage}

\begin{flushright}
% IPMU-21-XXXX
\end{flushright}

\vskip 3cm

\begin{center}

{\Large \bfseries Matching higher symmetries\\[.8em]
across Intriligator-Seiberg duality}

\vskip 1cm
Yasunori Lee$^1$, Kantaro Ohmori$^2$, and Yuji Tachikawa$^1$
\vskip 1cm

\begin{tabular}{ll}
$^1$ & Kavli Institute for the Physics and Mathematics of the Universe (WPI), \\
& University of Tokyo,  Kashiwa, Chiba 277-8583, Japan\\
$^2$ & Department of Physics, Faculty of Science, \\
& University of Tokyo, Bunkyo, Tokyo 113-0033, Japan
\end{tabular}

\vskip 1cm

\end{center}


\noindent
We study higher symmetries and anomalies of 4d  $\so(2n_c)$ gauge theory with $N_f=2n_f$ flavors.
We find that they depend on the parity of $n_c$ and $n_f$,
the global form of the gauge group, and the discrete theta angle.
The contribution from the fermions plays a central role in our analysis.
Furthermore, our conclusion applies to \Nequals1 supersymmetric cases as well, and
we see that higher symmetries and anomalies match across the duality between
$\so(2n_c)\leftrightarrow\so(2n_f-2n_c+4)$, originally found by Intriligator and Seiberg.


\end{titlepage}

\setcounter{tocdepth}{2}
\tableofcontents

\section{Introduction and summary}
\label{sec:introduction}
Our understanding of the concept of symmetries in quantum field theories (QFT) has been greatly improved in the last several years.
We now have the concept of $p$-form symmetries acting on $p$-dimensional objects \cite{Gaiotto:2014kfa}.
This concept  gives a point of view which unifies both
ordinary symmetries acting on point operators for $p=0$
and center symmetries of gauge theories acting on Wilson line operators for $p=1$.
In addition, the 't Hooft magnetic flux \cite{tHooft:1979rtg} can now be thought of as a background gauge field for the 1-form center symmetry.
It is also realized more recently that 0-form symmetries and 1-form symmetries can not only coexist in a direct product but also mix in a more intricate manner.
They can have mixed anomalies between them,
or they can also combine to form a \textit{higher} symmetry structure called 2-groups \cite{Cordova:2018cvg,Benini:2018reh}.

In this paper, we study these issues in the case of 4d $\so$ quantum chromodynamics (QCD),
i.e.~$\so(N_c)$ gauge theories with 
$N_f$ flavors of fermion fields in the vector representation.
Let us quickly recall the 0-form and 1-form symmetries these theories have.

As for the 1-form symmetry, we first need to recall that 
such theories come in three versions, $\Spin$, $\SO_+$ and $\SO_-$,
distinguished firstly by the global form of the gauge group ($\Spin$ vs.~$\SO$)
and further by the choice of the discrete theta angle ($\SO_+$ vs.~$\SO_-$) \cite{Aharony:2013hda}\footnote{%
This is when the theories are considered on spin manifolds.
For non-spin manifolds, a further distinction needs to be made \cite{Ang:2019txy}.
For simplicity, we only consider spin manifolds in this paper.
}.
They also differ by the nontrivial line operator they possess: 
the $\Spin$ theory has the Wilson line $W$ in the spinor representation,
the $\SO_+$ theory has the 't Hooft line $H$ which is mutually non-local with respect to $W$,
and the $\SO_-$ theory has the dyonic line $D=WH$. 
Furthermore, these line operators are charged under corresponding $\bZ_2$ 1-form symmetries,
which we respectively call electric, magnetic and dyonic 1-form symmetries.

As for the 0-form symmetry, 
we focus our attention on the $\su(N_f)$ symmetry acting on $N_f$ flavors of matter fields 
in the vector representation.
There can be and definitely are other discrete symmetries, but we will not consider them in this paper for brevity.

The main question is then how the $\bZ_2$ 1-form symmetry and the $\su(N_f)$ 0-form symmetry are related.\footnote{%
A partial answer was given in \cite{Hsin:2020nts}, but the contribution from fermions was not taken into account in that reference.
Our conclusion is therefore somewhat different from theirs.
}
We concentrate on the case when $N_c$ and $N_f$ are both even: $N_c=2n_c$ and $N_f=2n_f$.
We introduce three possible behaviors, which we call `extension', `anomaly' and `none'.

\paragraph{The `extension':} 
Take for example the $\Spin(2n_c)$ gauge theory with $2n_f$ flavors.
When $n_c$ is odd,
two copies of the Wilson line $W$ in the spinor representation
form a Wilson line in the vector representation.
This can be screened by a dynamical fermion, which was why $W^2=1$ 
as far as the 1-form symmetry charge was concerned.
Now let us recall that this dynamical fermion transforms nontrivially under $-1\in \SU(2n_f)$.
Therefore, when we further take the flavor symmetry into account, $W^2$ is still nontrivial.
As we will detail in this paper,  this means that the $\bZ_2$ 1-form symmetry extends the $\SU(2n_f)/\bZ_2$ 0-form symmetry in a nontrivial manner, forming a 2-group $H$\begin{equation}
0\to \bZ_2[1] \to H\to \SU(2n_f)/\bZ_2\to 0,\label{2-group}
\end{equation}
whose extension class is specified by \begin{equation}
\beta a_2 \in H^3(B(\SU(2n_f)/\bZ_2);\bZ_2),\label{postnikov}
\end{equation}
where $a_2$ is the generator of $H^2(B(\SU(2n_f)/\bZ_2);\bZ_2)=\bZ_2$~\cite{KonoMimura1974}
and $\beta$ is the Bockstein homomorphism.

\paragraph{The `anomaly':} 
The $\SO_+$ gauge theory is obtained by gauging the $\bZ_2$ 1-form symmetry of the $\Spin$ gauge theory \cite{Kapustin:2014gua}.
The presence of the fermions significantly complicates the analysis.
For the moment let us suppose that we have $N_f$ scalars instead of fermions in the vector representation.
Then, the argument of \cite{Tachikawa:2017gyf} immediately applies, and 
we see that the $\bZ_2$ 1-form symmetry of the $\SO(2n_f)$ theory and the $\SU(2n_f)/\bZ_2$ 0-form flavor symmetry remains a direct product but with a mixed anomaly given by \begin{equation}
2\pi i \cdot \frac12 \int B\beta a_2.\label{mixed}
\end{equation}


In the rest of the paper, we will carefully analyze how the $\bZ_2$ 1-form symmetry and the $\SU(2n_f)/\bZ_2$ 0-form symmetry are combined.
The derivation will be detailed in the following, and here we simply summarize the result  in Table~\ref{table:main}.
There, `none' specifies that they remain a direct product without mixed anomaly;
`anomaly' implies that they remain a direct product but with mixed anomaly of the form \eqref{mixed};
and `extension' means that they combine into a 2-group given by \eqref{2-group} with the extension class \eqref{postnikov}.


\begin{table}
\centering
\begin{tabular}{r@{\,}l|ccc}
$(n_c,$&$n_f)$ & $\Spin$ & $\SO_+$ & $\SO_-$\\
\hline
\vphantom{$\Bigm($}
(even,&even) & no\tikzmark{A}ne & no\tikzmark{B}ne & no\tikzmark{C}ne \\
\vphantom{$\Bigm($}
(odd,&even) & exte\tikzmark{P}nsion & ano\tikzmark{Q}maly & exte\tikzmark{R}nsion \\
\vphantom{$\Bigm($}
(even,&odd) & ano\tikzmark{S}maly & exte\tikzmark{T}nsion & exte\tikzmark{U}nsion \\
\vphantom{$\Bigm($}
(odd,&odd) & exte\tikzmark{L}nsion & exte\tikzmark{M}nsion & ano\tikzmark{N}maly 
\end{tabular}
\caption{How the $\bZ_2$ 1-form symmetry and the $\SU(2n_f)/\bZ_2$ 0-form symmetry are combined
in $\so(2n_c)$ QCD.
`none' implies that they remain a direct product without mixed anomaly;
`anomaly' means that they remain a direct product but with mixed anomaly;
and `extension' is when they combine into a nontrivial 2-group. 
The orange lines show how the duality of Intriligator and Seiberg acts on this set of theories.
\label{table:main}}

\tikz[overlay,remember picture]{%
\draw[<->,bend left,color=Orange,line width=1.5] (A.north) to (C.north);
\draw[<->,color=Orange,line width=1.5] (B.north west) .. controls +(75:0.5) ..  (B.north east);
\draw[<->,bend left,color=Orange,line width=1.5] (P.north) to (R.north);
\draw[<->,color=Orange,line width=1.5] (Q.north west) .. controls +(75:0.5) ..  (Q.north east);
\draw[<->,color=Orange,line width=1.5] (S.south) to ([yshift=.2em]N.north);
\draw[<->,color=Orange,line width=1.5] ([yshift=+.2em]T.south) to ([yshift=0em]M.north);
\draw[<->,color=Orange,line width=1.5] (U.south) to ([yshift=.2em]L.north);
}
\end{table}

Our result is equally applicable in the case of \Nequals1 supersymmetric QCD, 
since they are connected to the non-supersymmetric QCD by a continuous deformation 
preserving all the symmetries we care about.
Now, let us recall that Intriligator and Seiberg found in \cite{Intriligator:1995id} a duality exchanging $\so(N_c)$ and $\so(N_f-N_c+4)$,
which in our notation sends $n_c$ to $n_c'=n_f-n_c+2$.
In \cite{Aharony:2013hda}, this duality was refined to account for the global form of the gauge group and the discrete theta angle, and it was concluded that $\Spin$ is exchanged with $\SO_-$ while $\SO_+$ maps to itself.
This mapping was given a further check e.g.~by using supersymmetric localization on $S^3/\bZ_n \times S^1$ in \cite{Razamat:2013opa}.
Our analysis allows us to check this duality by comparing how the 1-form symmetry and the 0-form symmetry are combined in the dual pairs.
We superimposed the action of the duality on our main Table~\ref{table:main}.
It is satisfying to see that the duality action correctly preserves the behaviors `none', `anomaly' and `extension'.

The rest of the paper is organized as follows.
In Sec.~\ref{sec:2-group},
we determine exactly when the electric/magnetic/dyonic $\bZ_2$ 1-form symmetries
and the $\SU(2n_f)/\bZ_2$ flavor 0-form symmetry form a nontrivial 2-group,
by examining the charges of line operators in each theory.
In Sec.~\ref{sec:sl2z},
we exploit the $\SL(2,\bZ_2)$ actions on theories with $\bZ_2$ 1-form symmetries, including our $\so(2n_f)$ QCDs.
This will allow us to determine the 't Hooft anomalies they possess.
Combining the results with those obtained in Sec.~\ref{sec:2-group},
one can completely determine the structures of symmetries and anomalies of $\so(2n_f)$ QCDs,
and can further confirm that they are indeed compatible with the Intriligator-Seiberg duality.
Although the result itself is satisfactory, the analysis leading to it is somewhat ad-hoc, 
so we partially complement it with a more direct computation of fermion anomalies in Sec.~\ref{sec:fermion}.
Finally, the two appendices provide technical details of the mathematical facts used in the main part;
in Appendix~\ref{sec:bordism}, we compute relevant bordism groups capturing the fermion anomalies associated to various symmetries;
and in Appendix~\ref{sec:nonclosedP}, we describe some subtleties concerning the Pontrjagin square.

Before proceeding, we list the obstruction classes we frequently encounter in this paper.
In general, given a group $G$, a subgroup $\bZ_n$ in the center of $G$,
and a $G/\bZ_n$ bundle on a manifold $X$, 
there is a obstruction class $\in H^2(X; \bZ_n)$ controlling whether this bundle lifts to $G$.
For $G=\Spin(N_c)$ and $G/\bZ_2=\SO(N_c)$ this is the familiar second Stiefel-Whitney class $w_2$.
The classes we use are listed in Table~\ref{table:w2}.

\begin{table}
\[
\begin{array}{c| cc| c|c}
\text{name} & G/\bZ_n & G & \bZ_n & \text{comments}\\
\hline 
w_2 & \SO(2n_c) & \Spin(2n_c) & \bZ_2\\
v_2 & \SO(2n_c)/\bZ_2 & \SO(2n_c) &\bZ_2\\
x_2 & \SO(2n_c)/\bZ_2 & \Spin(2n_c) &\bZ_4 & (n_c:\text{odd})\\
\hline
a_2 & \SU(2n_f)/\bZ_2 & \SU(2n_f) &\bZ_2\\
a_2 & \USp(2n_f)/\bZ_2 & \USp(2n_f) &\bZ_2\\
\end{array}
\]
\caption{Our names for the obstruction classes $\in H^2(X,\bZ_n)$ controlling whether a $G/\bZ_n$ bundle
on $X$ lifts to a $G$ bundle.  \label{table:w2}}
\end{table}


\section{2-group structure}
\label{sec:2-group}
Let us first study whether the $\bZ_2$ 1-form symmetry and the $\SU(2n_f)/\bZ_2$ 0-form flavor symmetry form a nontrivial 2-group or not. 
This can be found rather physically by studying the line operators. 

\subsection{$\Spin$}
We start by discussing the $\Spin(2n_c)$ gauge theories. 
The results presented in this section was originally found in \cite[Sec.~4.4]{Hsin:2020nts}.

First, recall that the center of $\Spin(2n_c)$ is $\bZ_2\times \bZ_2$ or $\bZ_4$ depending on whether $n_c$ is even or odd.
This corresponds to the fact that the tensor square of a spinor representation contains the identity representation when $n_c$ is even while it contains the vector representation when $n_c$ is odd.

We now consider the Wilson line $W$ in the spinor representation in the $\Spin(2n_c)$ gauge theory with $2n_f$ fermions in the vector representation.
When $n_c$ is even, $W^2$ contains the identity representation, and therefore we simply have a $\bZ_2$ 1-form symmetry independent of the flavor symmetry, and there is nothing to see here.

When $n_c$ is odd, $W^2$ contains the vector representation.
This can be screened by the dynamical fermion, which however carries the fundamental representation of $\su(2n_f)$ flavor symmetry, 
and in particular transforms nontrivially under $-1\in \SU(2n_f)$.
In other words, the flavor Wilson line in the vector representation of $\SU(2n_f)$ can now be considered as the square  of the gauge Wilson line in the spinor representation of $\Spin(2n_c)$.
This means that we have the following extension of groups \begin{equation}
0\to \underbrace{\bZ_2}_{\substack{\text{group of}\\
\text{charges under}\\
\text{$\{\pm1\}\in \SU(2n_f)$}}}
\to \bZ_4 
\to \underbrace{\bZ_2}_{\substack{\text{group of}\\\text{gauge Wilson lines }\\
\text{up to screening}}} \to 0.
\label{charge-extension}
\end{equation}
As the groups of charges of $\su(2n_f)$ 0-form symmetry and $\bZ_2$ 1-form symmetry are combined nontrivially, 
the symmetry groups themselves are also combined nontrivially.
This can be seen most clearly by considering their background fields.

The fermion fields are simultaneously in the vector representation of $\SO(2n_c)$ and the fundamental representation of $\SU(2n_f)$,
and therefore are in a representation of $G=[\SO(2n_c)\times \SU(2n_f)]/\bZ_2$.
Given a $G$-bundle on a manifold $X$,
there is an $\SO(2n_c)/\bZ_2$ bundle and an $\SU(2n_f)/\bZ_2$ bundle associated to it.
Let us denote by $v_2,a_2\in H^2(X;\bZ_2)$ the obstruction classes controlling whether they lift to $\SO(2n_c)$ and $\SU(2n_f)$ respectively. 
Then we have $v_2=a_2$ for a $G$-bundle.
The flavor Wilson line in the fundamental representation is charged under $-1\in \SU(2n_f)$ in the center,
and $a_2$ can be considered as the background field for this $\bZ_2$ 1-form center symmetry.

Now, without the flavor background,  the background $E\in H^2(X;\bZ_2)$ for the electric $\bZ_2$ 1-form symmetry of the $\Spin(2n_c)$ theory sets the Stiefel-Whitney class $w_2\in H^2(X;\bZ_2)$ of the $\SO(2n_c)$ gauge bundle to be $E=w_2$, which controls whether it lifts to a $\Spin(2n_c)$ bundle.
When the flavor background $a_2$ is nontrivial,
the obstruction class $v_2$ controlling the lift from $\SO(2n_c)/\bZ_2$ to $\SO(2n_c)$ is nontrivial.
In this situation when $n_c$ is odd, $w_2$ can no longer be defined as a closed cochain; rather it satisfies $\delta w_2 = \beta v_2$, where $\beta$ is the Bockstein operation,
since together they specify the obstruction class $x_2\in H^2(X;\bZ_4)$ 
controlling the lift from $\SO(2n_c)/\bZ_2=\Spin(2n_c)/\bZ_4$ to $\Spin(2n_c)$.
As $E=w_2$ and $v_2=a_2$, we conclude that the background field satisfies \begin{equation}
\delta E=\beta a_2.
\label{ba}
\end{equation}

In general, a 2-group $H$ combining a 1-form symmetry $A$ and a 0-form symmetry $G$ of the form \begin{equation}
0\to A[1]\to H\to G\to 0
\end{equation} with the extension class $\alpha\in H^3(BG,A)$ is defined as a symmetry whose background field is given by a pair of a degree-2 cochain $E\in C^2(X,A)$ and a background $G$ field $g:X\to BG$ such that $\delta E = g^*(\alpha)$.
Here $A[1]$ means the Abelian group $A$ regarded as a 1-form symmetry,
and we drop the pull-back symbol $g^*$ when its presence is clear from the context.
In our case, we see that the $\bZ_2$ 1-form symmetry and the $\SU(2n_f)/\bZ_2$ 0-form flavor symmetry form the 2-group $H$ fitting in the sequence \begin{equation}
0\to \bZ_2[1]\to H \to \SU(2n_f)/\bZ_2 \to 0
\label{group-extension}
\end{equation} with the extension class being $\beta a_2 \in H^3(B(\SU(2n_f)/\bZ_2);\bZ_2)$.

Note that having the extension of groups of charges of line operators as in \eqref{charge-extension}
is equivalent to having a nontrivial 2-group extension \eqref{group-extension} 
whose background field satisfies \eqref{ba}.
Therefore, to find a nontrivial 2-group extension, we can simply study the group of charges of line operators,
which we will carry out for $\SO_\pm$ gauge theories next.

\subsection{$\SO_\pm$}

We would like to study how the magnetic $\bZ_2$ 1-form symmetry of the $\SO(2n_c)_+$ theory is combined with the $\su(2n_f)$ flavor symmetry.
In accord with the discussions in the previous subsection, 
we consider what happens when we take two copies of the 't Hooft line operator $H$ and fuse them.
At the very naive level, $H^2$ can be screened by dynamical monopoles,
but dynamical monopoles can receive flavor/gauge center charges from the fermion zero modes.

\paragraph{Deformations:}
To study these issues, it is useful to deform the theory and make it simpler
by performing the following steps:
\begin{itemize}
\item Reduce the flavor symmetry from $\su(2n_f)$ to $\usp(2n_f)$. 
The fundamental representation still transforms nontrivially under $-1\in \USp(2n_f)$, which is enough for our purposes.
\item Add an adjoint scalar $\Phi_{[ab]}$ and the interaction $\psi^{ai}_\alpha \psi^{bj}_\beta J_{ij} \Phi_{ab}\epsilon^{\alpha\beta} +c.c.$.
Here $a,b$ and $i,j$ are vector indices of $\so(2n_c)$ and $\usp(2n_f)$,
$\alpha,\beta$ are the spinor indices,
and  $J_{[ij]}$ is the constant matrix for the $\usp(2n_f)$.
\item Give a generic vacuum expectation value (vev) to $\Phi_{ab}$ to break $\SO(2n_c)$ to $\SO(2)^{n_c}$.
\end{itemize}

The 't Hooft lines in the resulting $\SO(2)^{n_c}$ theory can be labeled by their magnetic charges $(m_1,\ldots,m_{n_c})\in \bZ^{n_c}$.
The dynamical monopoles have the charges in the `adjoint class', which are
in the root lattice $\Lambda$ of $\SO(2n_c)$. 
Then, the group of the magnetic charges of 't Hooft lines up to screening by the dynamical monopoles is \begin{equation}
\bZ^{n_c}/\Lambda = \bZ_2,
\end{equation}
which agrees with the 1-form symmetry before the deformation.
We now would like to study how this $\bZ_2$ is combined with the flavor/gauge center $\bZ_2$ charge.

\paragraph{Reduction to  the $\so(4)$ case:}
For this purpose we need to know slightly more details of the dynamical monopoles.
The dynamical monopoles associated to the breaking of a gauge group to its Cartan were analyzed in many places, 
e.g.~in \cite{Weinberg:1979zt}.
There, the following was shown.
Let $\phi$ be the scalar vev in the real Cartan subalgebra, $\phi\in \mathfrak{h}\subset \mathfrak{g}$.
This determines the simple roots $\alpha$.
Then you can embed the standard spherically-symmetric 't Hooft-Polyakov monopole
using the $\mathfrak{su}(2)$ subalgebra associated to $\alpha$,
and have a monopole solution without additional bosonic moduli. 

Let us say we chose the standard $\phi$ such that the simple roots are 
\begin{equation}
(1,-1,\ldots,0),\ 
(0,1,-1,\ldots,0),\ldots,
(0,\ldots,1,-1),\ 
(0,\ldots,1,+1)\in \bZ^{n_c},
\end{equation}
which we call simple dynamical monopoles.
Now, consider the group $\bZ^{n_c}\times \bZ_2$ which combines the magnetic charges in $\bZ^{n_c}$ and the flavor/gauge center charge $q \in \bZ_2$.
What we are after is the quotient of $\bZ^{n_c}\times \bZ_2$ 
by the subgroup generated by the charges of simple dynamical monopoles,
which we denote respectively by
\begin{equation}
(1,-1,\ldots,0;q_1),\ 
(0,1,-1,\ldots,0;q_2),\ldots,
(0,\ldots,1,-1;q_{n_c-1}),\ 
(0,\ldots,1,+1;q_{n_c}).
\end{equation}

To determine this quotient,
we do not have to determine the all $q_i$'s; 
we simply use $q_{i=1,\ldots, n_c-2}$ to rewrite any charge vector $
(m_1,\ldots, m_{n_c-2}, m_{n_c-1},m_{n_c}; q)
$
into the form $
(0,\ldots,0, m, m' ; q').
$
Then, only $q_{n_c-1}$ and $q_{n_c}$ need to be determined.
This reduces the study to the case of $n_c=2$ and $\so(2n_c)\simeq \su(2)_1 \times \su(2)_2$,
where the monopoles associated to the simple roots are just 't Hooft-Polyakov monopoles associated to the two factors of $\su(2)$'s.

\paragraph{Analysis of the $\so(4)$ case:}

The vev of the adjoint scalar in this basis can be written as $(a_1,a_2)$, which we assume to be $a_1>a_2>0$.
Here, the fermion is in the vector representation of $\so(4)$.
Under the monopole in $\su(2)_1$, it is a doublet coupled to an adjoint vev of size $a_1$ with bare mass $a_2$,
and similarly for the monopole in $\su(2)_2$.

Now, the explicit analysis in \cite[Sec.~IV]{Callias:1977kg} concerning the number of zero modes in the 't Hooft-Polyakov monopole says that 
a doublet fermion coupled to an adjoint vev of size $a$ with bare mass $\mu$ has
a zero mode if $|a|>|\mu|$ and 
has no zero modes if $|a|<|\mu|$.
With our assumption $a_1>a_2>0$, this means that the monopole in $\su(2)_1$ has a zero mode, while the monopole in $\su(2)_2$ does not.
In our original basis, this means that the monopole with $(0,\ldots,1,-1;q_{n_c-1})$ does not produce any zero modes and $q_{n_c-1}=0$,
while the monopole with $(0,\ldots,1,+1;q_{n_c})$  has two zero modes per flavor.
The 1-form symmetry group is obtained by dividing $\bZ^2\times \bZ_2$ by the subgroup generated by $(1,-1;0)$ and $(1,+1;q_{n_c})$.
This is $\bZ_2\times \bZ_2$ or $\bZ_4$ depending on whether $q_{n_c}$ is $0$ or $1$.

Let us determine $q_{n_c}$, the center charge of the monopole in $\su(2)_1$.
We saw that there are two zero modes per flavor; 
this means that there are fermionic zero modes transforming in \begin{equation}
V_{2} \otimes R_{2n_f} 
\end{equation} where $V_2$ is the doublet of $\su(2)_2$\footnote{
	It is actually broken to $\u(1)$, but keeping $\su(2)_2$ representation is useful in organizing the answer.
}, 
$R_{2n_f}$ is the fundamental of $\usp(2n_f)$,
and we need to impose the reality condition using the pseudo-reality of both factors,
so that there are $4n_f$ Majorana fermion in total.

To determine the flavor/gauge center charge $q_{n_c}$ of the monopole,
it suffices to consider the case $n_f=1$;
the general case is given simply by multiplying it by $n_f$.
When $n_f=1$, there are $4$ Majorana fermions.
Quantizing them, we find the monopoles in \begin{equation}
(V_2 \otimes \mathbf{1}) \oplus (\mathbf{1}\otimes R_2).
\end{equation}
It has the `vector' charge under $\usp(2)\simeq\su(2)$ flavor symmetry or is a doublet under $\su(2)_2$,
which corresponds to the `vector' charge under $\so(4)$ gauge symmetry.
In either case, they have the flavor/gauge center charge $1 \in \{0,1\}=\bZ_2$.
Therefore we conclude the flavor/gauge center charge $q_{n_c}$ is simply given by $n_f$ mod 2.

\paragraph{Summary:}
Combining the intermediate steps we described above, we conclude the following: 
for the $\SO(2n_c)_+$ theory,
the group $\bZ_2$ of magnetic charges of 't Hooft lines is extended by the flavor/gauge center symmetry $\bZ_2$ to become $\bZ_4$ when $n_f$ is odd,
while they remain separate when $n_f$ is even.

The analysis of the $\SO(2n_c)_-$ theory is largely the same;
the only difference is that the discrete theta angle gives an additional gauge center charge to the simple dynamical monopole with the magnetic charge $(0,0,\ldots,1,+1)$, so that $q_{n_c}=n_f+n_c$ mod 2.
Therefore, we conclude the following:
for the $\SO(2n_c)_-$ theory,
the group $\bZ_2$ of magnetic charges of 't Hooft lines is extended by the flavor/gauge center symmetry $\bZ_2$ to become $\bZ_4$ when $n_f+n_c$ is odd,
while they remain separate when $n_f+n_c$ is even.

The result of the analysis is summarized in Table~\ref{table:2group}.
There, `product' means that the $\bZ_2$ 1-form symmetry and $\SU(2n_f)/\bZ_2$ flavor symmetry are kept separate and form a direct product,
while `extended' means that they form a nontrivial 2-group.
We remark that the nontrivial 2-group is always given by the extension \eqref{group-extension}  whose background fields satisfy \eqref{ba}.

\begin{table}
\centering
\renewcommand{\arraystretch}{1.2}
\begin{tabular}{r@{\,}l|ccc}
$(n_c,$&$n_f)$ & $\Spin$ & $\SO_+$ & $\SO_-$\\
\hline
(even,&even) & product & product & product \\
(odd,&even) & extended & product & extended \\
(even,&odd) & product & extended & extended \\
(odd,&odd) & extended & extended & product
\end{tabular}
\caption{How the $\bZ_2$ 1-form symmetry and the flavor symmetry $\SU(N_f)/\bZ_2$ are combined
in $\so(2n_c)$ QCD.
`product' means that they form a direct product,
and `extended' means that they form a nontrivial 2-group.
\label{table:2group}}
\end{table}

\section{$\SL(2,\bZ_2)$ action and the anomalies}
\label{sec:sl2z}
In the last section we determined the 2-group structure of the $\so(2n_c)$ gauge theories with $2n_f$ flavors, by studying the group of the charges of line operators. 
%In a QFT, any symmetry can come with an't Hooft anomaly.
Here we determine the anomalies of these symmetries, 
utilizing the $\SL(2,\bZ_2)$ action on the set of QFTs with $\bZ_2$ 1-form symmetry.

\subsection{$\SL(2,\bZ_2)$ action and $\so$ gauge theories}

Let us say that we are given a four-dimensional spin QFT $Q$ with $\bZ_2$ 1-form symmetry.
We denote its partition function on a manifold $X$ by $Z_Q[E]$, 
where we suppress the dependence on $X$ in the notation and $E\in H^2(X;\bZ_2)$ is the background field for the $\bZ_2$ 1-form symmetry.
We then define $SQ$ and $TQ$ to be QFTs with partition functions given by the formula \begin{equation}
Z_{SQ}[B] \propto \sum_{E} (-1)^{\int_X B\cup E} Z_Q[E],\qquad
Z_{TQ}[E]= (-1)^{\int_X \tfrac12 \fP(E)} Z_Q[E].
\end{equation}
We can show that $S^2=T^2=1$ and $(ST)^3=1$, meaning that they generate $\SL(2,\bZ_2)$.
This operation was introduced in  \cite{Gaiotto:2014kfa} as an analogue of the $\SL(2,\bZ)$ action on 3d theories with $\U(1)$ symmetry of  \cite{Witten:2003ya} and then further studied in \cite{Bhardwaj:2020ymp}.


Importantly, $\Spin(2n_c)$ and $\SO(2n_c)_\pm$  gauge theories with $2n_f$ flavors with the same $n_c$ and $n_f$ form a single orbit under this $\SL(2,\bZ)$ action.
More precisely, we need to make the distinction between $\Spin(2n_c)$ and $T(\Spin(2n_c))$ and  similarly between $\SO(2n_c)_\pm$ and $T(\SO(2n_c)_\pm)$, respectively.
Here the theories with $T$ prepended are different from the original ones only by its discrete theta coupling to the background.
Then we have the following chain of actions: \begin{equation}
\begin{tikzpicture}
	\node at (0,0) {$
		T(\Spin) \stackrel{T}{\longleftrightarrow} 
		\Spin \stackrel{S}{\longleftrightarrow} 
		\SO_+ \stackrel{T}{\longleftrightarrow} 
		T(\SO_+) \stackrel{S}{\longleftrightarrow} 
		T(\SO_-) \stackrel{T}{\longleftrightarrow} 
		\SO_-.
	$};
	\draw[->] (-5,-1) -- (-5,-0.5);
	\draw[->] (5.4,-1) -- (5.4,-0.5);
	\draw (-5,-1) -- node[above] {\scriptsize $S$} (5.4,-1);
\end{tikzpicture}
\label{sl2z-on-so}
\end{equation}

\subsection{$\SL(2,\bZ_2)$ actions with extra background}

Let us now study what happens if we perform this $\SL(2,\bZ_2)$ action when the $\bZ_2$ 1-form symmetry in question is part of a larger symmetry group.
So far we have been considering the effect of $\SU(2n_f)/\bZ_2$ flavor symmetry,
but the discussions in the last section shows that, at a formal level, only the background field $a_2 \in H^2(X;\bZ_2)$ matters, which controls the lift from $\SU(2n_f)/\bZ_2$ to $\SU(2n_f)$.
Let us regard $a_2$ as the background field for a flavor $\bZ_2$ 1-form symmetry.

The combined 1-form symmetry is either $\bZ_2\times \bZ_2$ or $\bZ_4$,
and we perform the $\SL(2,\bZ_2)$ action  by picking a $\bZ_2$ subgroup.
The symmetry and the anomaly of the resulting theory can be determined by a formal argument independent of the dynamics of the theory,
once the symmetry and the anomaly of the original theory 
and the action of the anomaly-free subgroup to be gauged are given, as discussed in \cite{Tachikawa:2017gyf}.

Let us work at the level of anomalies described by cohomology, since we do not need to deal with more general anomalies described by bordism.
We consider a $d$-dimensional QFT with a symmetry group $G$ with an anomaly specified by a cochain $\alpha\in C^{d+1}(G;\U(1))$. 
We pick a subgroup $H\subset G$ such that $\alpha$ trivializes in it, so that one can find $\mu \in C^d(H;\U(1))$ such that $\delta \mu = \alpha|_H$. 
We then gauge $H$, using $\mu$ as the action.
What determines the symmetry and the anomaly of the gauged theory is the data $(\mu,\alpha)$.
Clearly, given $\nu\in C^d(G;\U(1))$, the pair $(\mu,\alpha)$ and the pair $(\mu-\nu|_H,\alpha-\delta\nu)$ should give the same result, since we merely added the counterterm $\nu|_H$ as the action.
This allows us to always choose the pair of the form $(0,\alpha')$ equivalent to a given $(\mu,\alpha)$, 
by taking $\nu$ to be an arbitrary lift of $\mu$ from $H$ to $G$.\footnote{%
At this stage, the residual identifications $(0,\alpha')\sim (0,\alpha'')$ are of the form 
$\alpha''=\alpha'+\delta \nu$, where $\nu\in C^{d}(G;U(1))$ is required to satisfy $\nu|_H=0$.
Their equivalence classes form the relative cohomology group $H^{d+1}(G,H;U(1))$.
It might be interesting to study anomalies taking values in the relative cohomology groups.
}
This is convenient in discussing the $\SL(2,\bZ_2)$ action,
since our $S$ operation is defined in the convention that $\mu=0$.

\paragraph{The four choices:}
Now, what are the possible choices of $(\mu,\alpha)\sim(0,\alpha')$ we need to discuss?
Let us first consider $\bZ_2\times \bZ_2$ 1-form symmetry.
As detailed in the Appendix~\ref{sec:bordism}, the only possible anomaly for 4d spin QFTs with this symmetry is \begin{equation}
\alpha=%2\pi i \cdot 
\frac12  B\beta E  \label{anom}
\end{equation} where $B,E\in H^2(Y;\bZ_2)$ are the background fields on the bulk 5d spin manifold $Y$,
and we use $\bQ/\bZ$-valued cochains to describe the anomaly.
Its restriction to $\bZ_2$ 1-form symmetry subgroup is trivial i.e. $\alpha|_{H=\bZ_2} = 0$,
and thus the possible choice of $\mu$ is simply the discrete theta angle \begin{equation}
\mu =% 2\pi i \cdot
 \frac14 \fP(E),\label{P}
\end{equation}
where $\fP$ is the Pontrjagin square.
This $\mu$ can be lifted from the $\bZ_2$ subgroup to the entire $\bZ_2\times \bZ_2$ group as a closed cochain, 
and therefore does not affect the gauging process.
Therefore, we only have to consider pairs $(0,0)$ and $(0,\alpha)$.

Next, we consider $\bZ_4$ 1-form symmetry.
In the Appendix~\ref{sec:bordism}, we show that there is no anomaly for $\bZ_4$ 1-form symmetry.
Therefore we can pick $\alpha=0$. Then the only possible choice of $\mu$ for the $\bZ_2$ 1-form subgroup is again the discrete theta angle \eqref{P}.
One difference here is that the discrete theta angle \eqref{P} cannot be lifted as a closed cochain to the entire $\bZ_4$ 1-form subgroup.
As discussed in the Appendix~\ref{sec:nonclosedP}, with $\delta E=\beta a_2$,
where $a_2\in H^2(X;\bZ_4/\bZ_2)$,
one finds \begin{equation}
\alpha':=\delta \mu = %2\pi i \cdot 
\frac12 a_2 \beta_2 a_2,\label{anom'}
\end{equation}
where $\beta_2$ is the Bockstein associated to the sequence $0\to \bZ\to \bZ \to \bZ_4\to 0$,
whose action on $a_2$ can be defined since $a_2$ lifts to a $\bZ_4$-valued cochain.
Summarizing, the pairs  we need to consider for the $\bZ_4$ one-form symmetry are $(0,0)$ and $(\mu,0)\sim (0,\alpha')$. 

\def\Textended{extended$_T$}
Summarizing,  we need to consider the following four choices, namely:
\begin{itemize}
\item For $\bZ_2\times \bZ_2$, the pairs $(0,0)$ and $(0,\alpha)$, which we call `none' and `anomaly'
\item For $\bZ_4$, the pairs $(0,0)$ and $(\mu,0)\sim (0,\alpha')$, which we call `extended' and `\Textended'.
\end{itemize}

\paragraph{$\SL(2,\bZ_2)$ action on the four choices:}
Let us now determine how the $\SL(2,\bZ_2)$ action affects these data.
The case `none' is very easy.
The additional $\bZ_2$ factor plays no role, and we find the chain of actions given by \begin{equation}
\begin{tikzpicture}
	\node at (0,0) {$
		\text{none} \stackrel{T}{\longleftrightarrow} 
		\text{none} \stackrel{S}{\longleftrightarrow} 
		\text{none} \stackrel{T}{\longleftrightarrow} 
		\text{none} \stackrel{S}{\longleftrightarrow} 
		\text{none} \stackrel{T}{\longleftrightarrow} 
		\text{none}.
	$};
	\draw[->] (-4.6,-1) -- (-4.6,-0.5);
	\draw[->] (4.5,-1) -- (4.5,-0.5);
	\draw (-4.6,-1) -- node[above] {\scriptsize $S$} (4.5,-1);
\end{tikzpicture}
\label{trivial-chain}
\end{equation}

Let us now consider $\bZ_4$ 1-form symmetry and gauge its $\bZ_2$ subgroup.
The case  `extended' was analyzed in \cite{Tachikawa:2017gyf},
and the resulting theory has $\bZ_2\times \bZ_2$ 1-form symmetry with the anomaly \eqref{anom}, 
which we decided to call `anomaly'.

The case `\Textended' was analyzed in \cite{Hsin:2020nts},
where it was shown that the gauged theory has again `\Textended'.
Let us quickly recall why this is the case. 
The gauging process involves the term
 \begin{equation}
\exp\left[2\pi i\int_X \left( \frac14 \fP(E) +\frac12 B E\right)\right]
\label{boo}
\end{equation}
where $E$ is the variable to be gauged and $B$ is the newly introduced background field.
When $\bZ_2$ to be gauged is the $\bZ_2$ subgroup of a $\bZ_4$ symmetry,
$E$ is not necessarily closed, but rather satisfies the relation \begin{equation}
\delta E = \beta a_2
\end{equation}
where $a_2$ is the background field for the quotient $\bZ_4/\bZ_2$ 1-form symmetry.
Then the second term in \eqref{boo} is not closed, 
and to even talk about the first term in \eqref{boo}, one first needs to extend the definition of the Pontryagin square $\fP$ to non-closed cochains.

To make  the coupling \eqref{boo}  well-defined, we consider adding a counterterm depending solely on the newly introduced field $B$ to \eqref{boo} so that we have \begin{equation}
\exp\left[2\pi i\int_X \left( \frac14 \fP(E) +\frac12 B E + \frac14 \fP(B) \right)\right]
= 
\exp \left[2\pi i \int_X \frac14 \fP(E+B)\right] .
\end{equation}
This is perfectly well-defined and has no problem at all with the lifting,
if the newly-introduced background field $B$  also satisfies \begin{equation}
\delta B=\beta a_2.
\end{equation}
This means that, starting from `\Textended', performing $S$ and then $T$, we find a theory with the data `extended'.
Therefore, simply performing $S$ for the theory of the type `\Textended',  one  finds `\Textended'.

Combined, the preceding arguments allow us to see the chain of actions \begin{equation}
\begin{tikzpicture}
	\node at (0,0) {$
		\text{\Textended} \stackrel{T}{\longleftrightarrow} 
		\text{extended} \stackrel{S}{\longleftrightarrow} 
		\text{anomaly} \stackrel{T}{\longleftrightarrow} 
		\text{anomaly} \stackrel{S}{\longleftrightarrow} 
		\text{extended}\stackrel{T}{\longleftrightarrow} 
		\text{\Textended}.
	$};
	\draw[->] (-6.5,-1) -- (-6.5,-0.5);
	\draw[->] (6.3,-1) -- (6.3,-0.5);
	\draw (-6.5,-1) -- node[above] {\scriptsize $S$} (6.3,-1);
\end{tikzpicture}
\label{nontrivial-chain}
\end{equation}


\subsection{Anomalies from $\SL(2,\bZ_2)$ action} 

Let us now combine our result in Table~\ref{table:2group},
which summarizes our knowledge whether the 0-form part and the 1-form part form a nontrivial 2-group,
and the $SL(2,\bZ_2)$ action on the four choices, \eqref{trivial-chain} and \eqref{nontrivial-chain}, we determined above.
We first need to double each column of Table~\ref{table:2group},
since we need to distinguish $\Spin$ from $T\Spin$ and $\SO_\pm$ from $T\SO_\pm$.
The entry `product' in Table~\ref{table:2group} corresponds to either `none' or `anomaly',
and the entry `extended' there corresponds to either `extension' or `\Textended'.
We now demand that the $\SL(2,\bZ_2)$ action \eqref{sl2z-on-so} on $\so$ QCDT to be compatible with the $\SL(2,\bZ_2)$ action on the labels, \eqref{trivial-chain} and \eqref{nontrivial-chain}.
The only consistent assignment is given in Table~\ref{table:refined}.
As the way we determined the symmetry structures were somewhat indirect,
we confirm the symmetry structures of the $\Spin$ case in the next section
in a different means.


\begin{table}
\centering
\[
\begin{array}{r@{\,}l|cccccc}
(n_c,&n_f)  & T(\Spin) & \Spin & \SO_+ & T(\SO_+) & T(\SO_-) & \SO_- \\
\hline
\vphantom{\Bigm|}(\text{even},&\text{even}) & \text{none}\tikzmark{a1}& \color{Purple}\text{none}\tikzmark{a2}& \text{none}\tikzmark{a3}& \tikzmark{a4}\text{none}& \tikzmark{a5}\text{none}& \tikzmark{a6}\text{none}\\
\vphantom{\Bigm|}(\text{odd},&\text{even}) & \text{\Textended}\tikzmark{b1} &\color{Purple}\text{extended}\tikzmark{b2} & \text{anomaly}\tikzmark{b3} & \tikzmark{b4}\text{anomaly} & \tikzmark{b5}\text{extended } & \tikzmark{b6}\text{\Textended} \\
\vphantom{\Bigm|}(\text{even},&\text{odd}) & \text{anomaly}\tikzmark{c1} & \color{Purple}\text{anomaly}\tikzmark{c2} & \text{extended}\tikzmark{c3} & \tikzmark{c4}\text{\Textended}  & \tikzmark{c5}\text{\Textended} & \tikzmark{c6}\text{extended} \\
\vphantom{\Bigm|}(\text{odd},&\text{odd}) & \text{extended}\tikzmark{d1} & \color{Purple}\text{\Textended}\tikzmark{d2} & \text{\Textended}\tikzmark{d3} & \tikzmark{d4}\text{extended}
 &  \tikzmark{d5}\text{anomaly} & \tikzmark{d6}\text{anomaly}   \end{array} 
\]
\tikz[overlay,remember picture]{%
\draw[<->,bend left,color=Orange,line width=1.5] (a1.north) to (a6.north);
\draw[<->,bend left,color=Orange,line width=1.5] (a2.north) to (a5.north);
\draw[<->,bend left,color=Orange,line width=1.5] (a3.north) to (a4.north);
\draw[<->,color=Orange,line width=1.5] (b1.north) .. controls +(10:10em) ..  (b6.north);
\draw[<->,color=Orange,line width=1.5] (b2.north) .. controls +(10:5em) ..  (b5.north);
\draw[<->,color=Orange,line width=1.5] ([yshift=.2em]b3.west) to ([yshift=.2em]b4.east);
\draw[<->,color=Orange,line width=1.5] (c1.south) to (d6.north);
\draw[<->,color=Orange,line width=1.5] (c2.south) to (d5.north);
\draw[<->,color=Orange,line width=1.5] (c3.south) to (d4.north);
\draw[<->,color=Orange,line width=1.5] (c4.south) to (d3.north);
\draw[<->,color=Orange,line width=1.5] (c5.south) to (d2.north);
\draw[<->,color=Orange,line width=1.5] (c6.south) to (d1.north);
}
\caption{The symmetry structure of $\so(2n_c)$ QCD with $2n_f$ flavors,
as deduced from the 2-group structures found in Sec.~\ref{sec:2-group}
and from the $\SL(2,\bZ_2)$ action discussed in this section.
The symmetry structure of the $\Spin$ case will be checked independently in the next section.
\label{table:refined}}
\end{table}


We can also use this Table~\ref{table:refined}
to give a further check of the Intriligator-Seiberg duality,
which is known to act as follows:
\begin{equation}
\renewcommand{\arraystretch}{1.1}
\begin{array}{ccc}
	\Spin(2n_c) & \leftrightarrow & T(\SO_-(2n_f-2n_c+4)),\\
	\SO_+(2n_c) & \leftrightarrow & T(\SO_+(2n_f-2n_c+4)),\\ 
	\SO_-(2n_c) & \leftrightarrow & T(\Spin(2n_f-2n_c+4)).
\end{array}
\end{equation} 
We see that the symmetry structure is indeed preserved across the duality.
This duality action  taking  the finer distinction of $Q$ and $TQ$ into account was
first discussed in \cite[Sec.~6]{Gaiotto:2014kfa} and further discussed in \cite{Bhardwaj:2020ymp}.


\section{Fermion contribution to anomalies}
\label{sec:fermion}
So far, we first determined the 2-group structure in Sec.~\ref{sec:2-group} by studying the charges of line operators,
and then determined the anomalies in Sec.~\ref{sec:sl2z} by matching it to the action of $\SL(2,\bZ_2)$.
Going over  the entries on the column $\Spin$ of Table~\ref{table:refined},
we find that the anomaly is trivial when $(n_c,n_f)$ is (even,even) or (odd,even),
while it is $\alpha$ given in \eqref{anom} when $(n_c,n_f)$ is (odd,even) 
and $\alpha'$ given in \eqref{anom'} when (odd,odd).
Since the 1-form symmetry background in the $\Spin$ theory is simply the Stiefel-Whitney class $w_2$ of the $\SO(2n_c)$ gauge bundle, 
these anomalies should simply come from the anomalies of fermions charged under $[\SO(2n_c)\times \USp(2n_f)]/\bZ_2$.
Here we use $\USp(2n_f)$ instead of $\SU(2n_f)$, because under the latter we also have perturbative anomalies, which would complicate the analysis.

For even $n_c$, the anomaly should be given by  \begin{equation}
\alpha=% 2\pi i \cdot 
\frac12 w_2 \beta a_2
\end{equation}
where $w_2,a_2\in H^2(X;\bZ_2)$ controls the lift from $\SO(2n_c)$ to $\Spin(2n_c)$
and from $\USp(2n_f)/\bZ_2$ to $\USp(2n_f)$, respectively.
For odd $n_c$, the anomaly cochain should be given by 
 \begin{equation}
\alpha'=%2\pi i \cdot
 \frac12 v_2 \beta_2 v_2
\end{equation} where $v_2$ is the class in $H^2(X;\bZ_2)$ which is the mod-2 reduction of
the class $x_2 \in H^2(X;\bZ_4)$ controlling the lift from $\SO(2n_c)/\bZ_2=\Spin(2n_c)/\bZ_4$ to $\Spin(2n_c)$.
We note that $\alpha'$ is exact, but as explained in the previous section, this still affects the gauging process.


The aim of this last section is to give a check of these anomalies from a different point of view.
We will proceed as follows. 
Starting from the theory where the fermions are charged under $\SO(2n_c)\times \USp(2n_f)$,
we add scalar fields which are adjoint under $\USp(2n_f)$ in the system, and break it down to a subgroup.
We then determine the effective interaction induced by the fermion zero modes.
The next step is to see what happens when the symmetry group is changed from $\SO(2n_c)\times \USp(2n_f)$ to its $\bZ_2$ quotient;
we will see that the effective interaction will have the required anomalies.

Before proceeding, we have two remarks.
First, this method was first used in \cite[Sec.~4]{Witten:1995gf} to understand `a curious minus sign' 
appearing in the topologically-twisted Seiberg-Witten theory, 
which was more recently recognized as determining an anomaly in \cite[Sec.~2.4.3]{Cordova:2018acb}.
It was also used in \cite[Sec.~3.1 and Sec.~5.1.2]{Wang:2018qoy} to relate the `new' $\SU(2)$ anomaly 
with the effective interaction in the $\U(1)$ theory.
Second, in this section we can only say that the effective interaction we find is compatible with the anomalies
as found in Sec.~\ref{sec:sl2z},
and will not be able to determine it completely.
This is mostly due to the fact that the computation of the spin bordism group $\Omega^\text{spin}_d([\SO(2n_c)\times \USp(2n_f)]/\bZ_2)$ which governs the anomaly is quite hard, 
because even the integral cohomology of the group in question is hard to obtain.
Only in a couple of cases we can say more, as we comment along the way.

\subsection{Effective interaction}
We break $\USp(2n_f)$ down to $\U(n_f)$ using a scalar field,
such that the fundamental representation of $\USp(2n_f)$ splits into the fundamental plus the anti-fundamental representation of $\U(n_f)$.
The monopole charge is given by the first Chern class $c_1$ of the low-energy $\U(n_f)$ flavor symmetry.

Take a standard 't Hooft-Polyakov monopole associated to $\U(1)\subset \USp(2)$
and embed it into  $\U(n_f)\subset \USp(2n_f)$.
The fermion zero modes form a vector representation of $\SO(2n_c)$,
whose quantization leads to the spinor representation.

As first discussed in \cite{Thorngren:2014pza} and also used in \cite[Sec.~3.1]{Wang:2018qoy},
this means that there is an effective interaction \begin{equation}
%2\pi i \cdot
 \frac12 \int_X w_2 c_1.
 \label{bulk}
\end{equation} %where $w_2$ is the Stiefel-Whitney class of the $\SO(2n_c)$ bundle
%and $c_1$ is the first Chern class of $\U(n_f)$.
One way to understand it is as follows.

We started from a system which has $\SO(2n_c)$ symmetry,
but the spinor representation is only a projective representation of this symmetry.
This is an anomaly at the core of the monopole, which needs to flow in from the bulk.
Indeed, take the spacetime  to be $X=\bR_{\ge 0} \times \bR_t \times S^2$ around the monopole.
Reducing the bulk term \eqref{bulk} on $S^2$ with $\int_{S^2} c_1=1$, 
we have the  effective interaction $\int_Y w_2$
on the half-space $Y=\bR_{\ge 0}\times \bR_t$,
and the monopole lives on the boundary. 
Therefore, the degree of freedom on the boundary is in the projective representation characterized by 
$w_2\in H^2(\SO(2n_c),\bZ_2)$.

\subsection{Anomalies}
We now take the $\bZ_2$ quotient, changing the symmetry group from 
$\SO(2n_c)\times \USp(2n_f)$ to $[\SO(2n_c)\times \USp(2n_f)]/\bZ_2$.
Note that $\pi_1(\U(n_f)/\bZ_2)=\bZ\times \bZ_2$  or $\bZ$ depending on whether $n_f$ is even or odd.
We denote by $a_2$ the obstruction class to lift a $\U(n_f)/\bZ_2$ bundle to a $\U(n_f)$ bundle.
This implies the following:
\begin{itemize}
\item When $n_f$ is even, $c_1(\U(n_f)) = c_1(\U(n_f)/\bZ_2)$ and $a_2(\U(n_f)/\bZ_2)=a_2(\USp(2n_f))$.
\item When $n_f$ is odd, $c_1(\U(n_f)/\bZ_2)=2 c_1(\U(n_f))$ when the latter is well-defined.
More generally, $a_2(\U(n_f)/\bZ_2)$ is the mod-2 reduction of $c_1(\U(n_f)/\bZ_2)$.
\end{itemize}
We now compute the anomaly cochains in the four cases separately:
\paragraph{$(n_c,n_f)$=(even,even):}
$w_2(\SO(2n_c))$ and $c_1(\U(n_f))$ can be generalized to closed cochains of $\SO(2n_c)/\bZ_2$ 
and of $\U(n_f)/\bZ_2$
without any problem,
and therefore \begin{equation}
\delta\left(\frac12 w_2(\SO(2n_c)) c_1(\U(n_f))\right) = 0.
\end{equation}

\paragraph{$(n_c,n_f)$=(odd,even):}
$w_2(\SO(2n_c))$ needs to be upgraded to a $\bZ_4$-valued cochain $x_2(\SO(2n_c)/\bZ_2)$.
The original interaction is then \begin{equation}
\frac14 x_2(\SO(2n_c)/\bZ_2) c_1(\U(n_f))
\end{equation}
which is closed without problem,
and therefore taking $\delta$ results in zero.

\paragraph{$(n_c,n_f)$=(even,odd):}
Here we need to replace $c_1(\U(n_f))$ by $c_1(\U(n_f)/\bZ_2)/2$.
The effective interaction is then \begin{equation}
\frac14 w_2(\SO(2n_c)/\bZ_2) c_1(\U(n_f)/\bZ_2) 
\end{equation} and \begin{align}
\delta\left(\frac14 w_2(\SO(2n_c)/\bZ_2) c_1(\U(n_f)/\bZ_2) \right)
&= \frac12 \left(\frac12\delta w_2(\SO(2n_c)) c_1(\U(n_f)/\bZ_2) \right) \\
&= \frac12 \Big(\beta w_2(\SO(2n_c))\Big)c_1(\U(n_f)/\bZ_2) 
\end{align}
which is a pull-back of the anomaly cochain
\begin{equation}
 \frac12 \Big(\beta w_2(\SO(2n_c))\Big)a_2(\USp(2n_f)/\bZ_2). 
\end{equation}
When $n_c=2$ and $n_f=1$, we can confirm that this is indeed the entire anomaly,
since we can compute $\mathrm{Hom}(\Omega^\text{spin}_5([\SO(4)\times \USp(2)]/\bZ_2),\U(1))$
and show that this is the only nontrivial element there.
For details, see Appendix~\ref{sec:so4su2}.

\paragraph{$(n_c,n_f)$=(odd,odd):}
Now we make the replacement on both sides and therefore the effective interaction is \begin{equation}
\frac18 x_2(\SO(2n_c)/\bZ_2) c_1(\U(n_f)/\bZ_2) 
\end{equation}
and \begin{align}
\delta\left(\frac18 x_2(\SO(2n_c)/\bZ_2) c_1(\U(n_f)/\bZ_2) \right)
&= \frac12 \left(\frac14\delta v_2(\SO(2n_c)/\bZ_2) c_1(\U(n_f)/\bZ_2) \right) \\
&= \frac12 \Big(\beta_2 v_2(\SO(2n_c)/\bZ_2)\Big)c_1(\U(n_f)/\bZ_2) 
\end{align}
which is the pull-back of \begin{equation}
 \frac12 \Big(\beta_2 v_2(\SO(2n_c))\Big)a_2(\USp(2n_f)).
\end{equation}
We now recall that the symmetry we are considering now is $[\SO(2n_c)\times \USp(2n_f)]/\bZ_2$,
and therefore there is a single degree-2 obstruction cochain which equals both $v_2$ and $a_2$,
and therefore the anomaly cochain is \begin{equation}
 \frac12 v_2\beta_2 v_2.
\end{equation}
This is what we wanted to show.
Recall that this anomaly cochain is exact as we have repeatedly mentioned.
This is consistent with our computation of the bordism group in Appendix~\ref{sec:so(2odd)},
where we show that $\Omega^\text{spin}_5([\SO(2n_c)\times \USp(2n_f)]/\bZ_2=0$ when $n_c$ is odd.

\section*{Acknowledgments}
The authors thank discussions with 
Lakshya Bhardwaj,
Clay C\'ordova,
Po-shen Hsin,
Ho Tat Lam,
Sakura Sch\"afer-Nameki,
Nati Seiberg,
and Yunqin Zheng.

Y.L.~is partially supported by the Programs for Leading Graduate Schools, MEXT, Japan, via the Leading Graduate Course for Frontiers of Mathematical Sciences and Physics
and also by JSPS Research Fellowship for Young Scientists.
Y.T.~is partially supported  by JSPS KAKENHI Grant-in-Aid (Wakate-A), No.17H04837 
and also by WPI Initiative, MEXT, Japan at IPMU, the University of Tokyo.


\appendix
\section{Bordism group computations}
\label{sec:bordism}
\subsection{$\bZ_4$ 1-form symmetry}
According to \cite[Appendix C.3]{Clement2002} and \cite[Eq.\,(6.3)]{Wan:2018bns}, it seems that we have
\begin{equation}
	\begin{array}{ccc}
		E^2_{p,q}=H_p\big(K(\bZ_4,2);\Omega_q^{\text{spin}}\big) && \tilde\Omega_{p+q}^{\text{spin}}(K(\bZ_4,2))\vspace{4mm}\\
		\begin{array}{c|c:cccccccccccc}
			6  &&&&&& \\
			5  & \cellcolor{lightyellow} & \hphantom{\bZ_2} & \hphantom{\bZ_2} & \hphantom{\bZ_2} & \hphantom{\bZ_2} & \hphantom{\bZ_2} \\
			4  & \bZ & \cellcolor{lightyellow} & \ast && \ast & \ast & \ast\\
			3  &  && \cellcolor{lightyellow} &&&\\
			2  & \bZ_2 &  & \red{\fbox{\black{$\bZ_2$}}} & \Blue{\fbox{\black{$\bZ_2$}}}\cellcolor{lightyellow} & \ast & \ast & \ast\\
			1  & \bZ_2 && \red{\dbox{\black{$\bZ_2$}}} & \Blue{\dbox{\black{$\bZ_2$}}} & \red{\fbox{\black{$\bZ_2$}}}\cellcolor{lightyellow} & \Blue{\fbox{\black{$\bZ_2$}}} & \ast\\
			0 & \bZ &  & \bZ_4 &  & \red{\dbox{\black{$\bZ_8$}}} & \Blue{\dbox{\black{$\bZ_2$}}} \cellcolor{lightyellow} & \ast\\
			\hline
			& 0 & 1 & 2 & 3 & 4 & 5 & 6 \\
		\end{array}
		& \quad\longrightarrow & 
		\begin{array}{c|c}
			6  & \ast\\
			5  & \cellcolor{lightyellow}\\
			4  & \bZ_4\\
			3  & \\
			2  & \bZ_4\\
			1  & \\
			0 & \\
			\hline\\
		\end{array}
	\end{array}
\end{equation}

The corresponding invariant in 4d is simply \begin{equation}
\exp(2\pi i \frac{p}{4} \int \frac12\fP(a) )
\end{equation}
where $\fP:H^2(-;\bZ_4)\to H^4(-;\bZ_8)$ is the Pontryagin square,
which is even mod 8 on a spin manifold.

\subsection{$\bZ_2\times \bZ_2$ 1-form symmetry}
Exploiting the fact that $K(\bZ_2\times \bZ_2, 2) = K(\bZ_2, 2) \times K(\bZ_2, 2)$,
it seems that we have
\begin{equation}
	\begin{array}{ccc}
		E^2_{p,q}=H_p\big(K(\bZ_2\times \bZ_2,2);\Omega_q^{\text{spin}}\big)
		&& \tilde\Omega_{p+q}^{\text{spin}}(K(\bZ_2\times \bZ_2,2))\vspace{4mm}\\
		\begin{array}{c|c:cccccccccccc}
			6  &&&&&& \\
			5  & \cellcolor{lightyellow} & \hphantom{\bZ_2} & \hphantom{\bZ_2} & \hphantom{\bZ_2} & \hphantom{\bZ_2} & \hphantom{\bZ_2} \\
			4  & \bZ & \cellcolor{lightyellow} & \ast && \ast & \ast & \ast\\
			3  &  && \cellcolor{lightyellow} &&&\\
			2  & \bZ_2 &  & \red{\fbox{\black{$\bZ_2^{\oplus 2}$}}} & \Blue{\fbox{\black{$\bZ_2^{\oplus 2}$}}}\cellcolor{lightyellow} & \ast & \ast & \ast\\
			1  & \bZ_2 && \red{\dbox{\black{$\bZ_2^{\oplus 2}$}}} & \Blue{\dbox{\black{$\bZ_2^{\oplus 2}$}}} & \green{\fbox{\red{\fbox{\black{$\bZ_2^{\oplus 3}$}}}}}\cellcolor{lightyellow} & \Blue{\fbox{\black{$\bZ_2^{\oplus 6}$}}} & \ast\\
			0 & \bZ &  & \bZ_2^{\oplus 2} &  & \red{\dbox{\black{$\bZ_4^{\oplus 2}\oplus \bZ_2$}}} & \Blue{\dbox{\black{$\bZ_2^{\oplus 3}$}}} \cellcolor{lightyellow} & \green{\fbox{\black{$\ast$}}}\\
			\hline
			& 0 & 1 & 2 & 3 & 4 & 5 & 6 \\
		\end{array}
		& \quad\longrightarrow & 
		\begin{array}{c|c}
			6  & \ast\\
			5  & \bZ_2\cellcolor{lightyellow}\\
			4  & \bZ_2^{\oplus 3}\\
			3  & \\
			2  & \bZ_2^{\oplus 2}\\
			1  & \\
			0 & \\
			\hline\\
		\end{array}
	\end{array}
\end{equation}
The $\bZ$ homology of $K(\bZ_2, 2)$ is again read off from \cite{Clement2002},
while the $\bZ_2$ (co)homology is known \cite{Serre1953} to be
\begin{equation*}
	H^\ast(K(\bZ_2,2);\bZ_2)
	=
	\bZ_2[x_2, Sq^1 x_2, Sq^2Sq^1 x_2, \cdots].
\end{equation*}

The corresponding bordism invariants in 4d are $\fP(a)/2$, $ab$, $\fP(b)/2$, 
and the one in 5d is $a\beta b$.

\subsection{$[\SO(4)\times \SU(2)]/\bZ_2$}
\label{sec:so4su2}

\subsubsection{Leray-Serre SS (preparatory)}
For the various input of cohomology groups, see Appendix A of our WZW paper.
For the fibration
\begin{equation}
	BSO(3)
	\ \to\ 
	B\left(SO(3)\times SO(3)\right)
	=
	B\left(\tfrac{SO(4)}{\bZ_2}\right)
	\ \to\ 
	BSO(3)
\end{equation}
one has
\begin{equation}
	\begin{array}{ccc}
		E_2^{p,q}=H^p\big(BSO(3);H^q(BSO(3);\bZ)\big) && H^{p+q}(B\left(\tfrac{SO(4)}{\bZ_2}\right);\bZ)\vspace{4mm}\\
		\begin{array}{c|ccccccccccccc}
			6 & \bZ_2 && \ast & \ast & \ast & \ast & \ast \\
			5  & \hphantom{\bZ_2}\cellcolor{lightyellow} & \hphantom{\bZ_2} & \hphantom{\bZ_2} & \hphantom{\bZ_2} & \hphantom{\bZ_2} & \hphantom{\bZ_2} \\
			4  & \bZ & \cellcolor{lightyellow} && \ast & \ast && \ast\\
			3  & \bZ_2 && \bZ_2\cellcolor{lightyellow} & \bZ_2 & \ast & \ast & \ast \\
			2  & &  && \cellcolor{lightyellow} &&\\
			1  &  &&  && \cellcolor{lightyellow} &&\\
			0 & \bZ &  && \bZ_2 & \bZ & \cellcolor{lightyellow} & \bZ_2\\
			\hline
			& 0 & 1 & 2 & 3 & 4 & 5 & 6 \\
		\end{array}
		& \longrightarrow & 
		\begin{array}{c|c}
			6  &\bZ_2^{\oplus 3}\\
			5  & \bZ_2\cellcolor{lightyellow}\\
			4  & \bZ^{\oplus 2}\\
			3  & \bZ_2^{\oplus 2}\\
			2  & \\
			1  & \\
			0 & \bZ\\
			\hline\\
		\end{array}
	\end{array}
\end{equation}
Here we expect non-trivial differentials to be absent (for the region of interest)
from the explicit consideration of generators (since there are $W_3$ and $W'_3$, there should be $(W_3)^2$, $(W'_3)^2$, and $W_3W'_3$)
or by requiring proper reproduction of the $\bZ_2$ cohomology (which we expect to be generated by $w_2$, $w'_2$, $w_3$, and $w'_3$).
Then, for the fibration
\begin{equation}
	BSU(2)
	\ \to\ 
	B\left(\tfrac{SO(4)\times SU(2)}{\bZ_2}\right)
	\ \to\ 
	B\left(\tfrac{SO(4)}{\bZ_2}\right)
\end{equation}
we can further plug it into
\begin{equation}
	\begin{array}{ccc}
		E_2^{p,q}=H^p\big(B\left(\tfrac{SO(4)}{\bZ_2}\right);H^q(BSU(2);\bZ)\big) && H^{p+q}(B\left(\tfrac{SO(4)\times SU(2)}{\bZ_2}\right);\bZ)\vspace{4mm}\\
		\begin{array}{c|ccccccccccccc}
			6  &&&&&& \\
			5  & \cellcolor{lightyellow} & \hphantom{\bZ_2} & \hphantom{\bZ_2} & \hphantom{\bZ_2} & \hphantom{\bZ_2} & \hphantom{\bZ_2} \\
			4  & \red{\fbox{\black{$\bZ$}}} & \cellcolor{lightyellow} && \ast & \ast & \ast & \ast\\
			3  &  && \cellcolor{lightyellow} &&&\\
			2  &&  && \cellcolor{lightyellow} &&\\
			1  &  &&  && \cellcolor{lightyellow} &&\\
			0 & \bZ &  && \bZ_2^{\oplus 2} & \bZ^{\oplus 2} & \red{\fbox{\black{$\bZ_2$}}}\cellcolor{lightyellow} & \bZ_2^{\oplus 3}\\
			\hline
			& 0 & 1 & 2 & 3 & 4 & 5 & 6 \\
		\end{array}
		& \longrightarrow & 
		\begin{array}{c|c}
			6  & \bZ_2^{\oplus 3}\\
			5  & \cellcolor{lightyellow}\\
			4  & \bZ^{\oplus 3}\\
			3  & \bZ_2^{\oplus 2}\\
			2  & \\
			1  & \\
			0 & \bZ\\
			\hline\\
		\end{array}
	\end{array}
\end{equation}
Taking the normalization of instanton number into account (see Ohmori-san's e-mail on 2020-08-19),
the differential \red{\fbox{\black{$d_2 : E_{0,4} \to E_{5,0}$}}} seems to be non-trivial.

So we believe the integral cohomology structure to be
\begin{equation}
	\label{case1}
	\renewcommand{\arraystretch}{1.2}
	\begin{array}{c|cccccccccccccccc}
		d & 0 & 1 & 2 & 3 & 4 & 5 & 6 & \cdots \\
		\hline
		H^d(B\left(\tfrac{SO(4)\times SU(2)}{\bZ_2}\right);\bZ) & \bZ & 0 & 0 & \bZ_2^{\oplus 2} & \bZ^{\oplus 3} & 0 & \bZ_2^{\oplus 3} & \cdots\\
		\hline
		\text{generator} & 1 & - & - & W_3 & p_1 & - & (W_3)^2 & \cdots\\
		&&&& W'_3 & p'_1 && (W'_3)^2\\
		&&&&& 2c_2 && W_3W'_3
	\end{array}
\end{equation}
where the reduction to $\bZ_2$ cohomology are
\begin{equation}
	\begin{array}{ccc}
		W_3 & \to & w_3\\
		p_1 & \to & (w_2)^2\\
%		y_5 & \to & w_2w'_3 + w_3w'_2 \,(?)\\
	\end{array}
\end{equation}

\subsubsection{Atiyah-Hirzebruch SS}
%Turning cohomology groups into homology groups via the universal coefficient theorem,
Having obtained (co)homology groups,
one can fill in the $E^2$-page of the AHSS:
\begin{equation}
	\begin{array}{ccc}
		E^2_{p,q}=H_p\big(B\left(\tfrac{SO(4)\times SU(2)}{\bZ_2}\right);\Omega_q^{\text{spin}}\big)\vspace{2mm}\\
		\begin{array}{c|c:cccccccccccc}
			6  &&&&&& \\
			5  & \cellcolor{lightyellow} & \hphantom{\bZ_2} & \hphantom{\bZ_2} & \hphantom{\bZ_2} & \hphantom{\bZ_2} & \hphantom{\bZ_2} \\
			4  & \bZ & \cellcolor{lightyellow} & \bZ_2^{\oplus 2} && \ast & \ast & \ast\\
			3  &  && \cellcolor{lightyellow} &&&\\
			2  & \bZ_2 &  & \red{\dbox{\black{$\bZ_2^{\oplus 2}$}}} & \Blue{\dbox{\black{$\bZ_2^{\oplus 2}$}}}\cellcolor{lightyellow} & \bZ_2^{\oplus 3} & \ast & \ast\\
			1  & \bZ_2 && \red{\fbox{\black{$\bZ_2^{\oplus 2}$}}} & \Blue{\fbox{\black{$\bZ_2^{\oplus 2}$}}} & \green{\fbox{\black{\red{\dbox{\black{$\bZ_2^{\oplus 3}$}}}}}}\cellcolor{lightyellow} & \Blue{\dbox{\black{$\bZ_2^{\oplus 3}$}}} & \ast\\
			0 & \bZ &  & \bZ_2^{\oplus 2} &  & \red{\fbox{\black{$\bZ^{\oplus 3}$}}} & \Blue{\fbox{\black{$\bZ_2^{\oplus 3}$}}} \cellcolor{lightyellow} & \green{\fbox{\black{$\ast$}}}\\
			\hline
			& 0 & 1 & 2 & 3 & 4 & 5 & 6 \\
		\end{array}
	\end{array}
\end{equation}
Based on our belief, 
\red{\fbox{\black{$d^2 : E^2_{4,0} \to E^2_{2,1}$}}} and
\red{\dbox{\black{$d^2 : E^2_{4,1} \to E^2_{2,2}$}}} should be a dual of
\begin{equation}
	\begin{array}{ccccc}
		Sq^2 w_2 & = & (w_2)^2 \\
		Sq^2 w'_2 & = & (w'_2)^2
	\end{array}
\end{equation}
and also
\Blue{\fbox{\black{$d^2 : E^2_{5,0} \to E^2_{3,1}$}}} and
\Blue{\dbox{\black{$d^2 : E^2_{5,1} \to E^2_{3,2}$}}} should be a dual of
\begin{equation}
	\begin{array}{ccccc}
		Sq^2 w_3 & = & w_2w_3\\
		Sq^2 w'_3 & = & w'_2w'_3
	\end{array}
\end{equation}
and finally \green{\fbox{\black{$d^2 : E^2_{6,0} \to E^2_{4,1}$}}} should be a dual of
\begin{equation}
	\begin{array}{ccccc}
		Sq^2 (w_2w'_2) & = & w_3w'_3 + (w_2)^2w'_2 + w_2(w'_2)^2
	\end{array}
\end{equation}
then %one can proceed, and 
the would-be-$E_3$-page is given by
\begin{equation}
	\label{e3page1}
	\begin{array}{c|c:cccccccccccc}
		6  &&&&&& \\
		5  & \cellcolor{lightyellow} & \hphantom{\bZ_2} & \hphantom{\bZ_2} & \hphantom{\bZ_2} & \hphantom{\bZ_2} & \hphantom{\bZ_2} \\
		4  & \bZ & \cellcolor{lightyellow} & \ast && \ast & \ast & \ast\\
		3  &  && \cellcolor{lightyellow} &&&& \hphantom{\bZ_2}\\
		2  & \bZ_2 &  && \cellcolor{lightyellow} & \ast & \ast & \ast\\
		1  & \bZ_2 &&  && \cellcolor{lightyellow} & \ast & \ast\\
		0 & \bZ &  & \bZ_2^{\oplus 2} &  & \bZ^{\oplus 3} & \bZ_2\cellcolor{lightyellow} & \ast\\
		\hline
		& 0 & 1 & 2 & 3 & 4 & 5 & 6 \\
	\end{array}
\end{equation}


\subsubsection{Adams SS}
According to our naive guess, the module $\tilde H^\ast(B\left(\tfrac{SO(4)\times SU(2)}{\bZ_2}\right);\bZ_2)_{\leq 5}$ consists of
\begin{equation}
	\begin{tikzpicture}[thick,baseline=(m-2-4.south)]
		\matrix (m) [
			matrix of math nodes,
			row sep= 1em,
			column sep=6em,
		]{
			&&& \colorB{\bullet}\\
			&&& \colorB{\bullet}\\
			&&& \colorB{\bullet}\\
			&& \colorB{\bullet} & \colorB{\bullet}\\
			\colorA{\bullet} && \colorB{\bullet} && \colorC{\bullet}\\
		    \colorA{\bullet} && \colorB{\bullet} && \colorC{\bullet} & \colorD{\bullet}\\
			\colorA{\bullet} && \colorB{\bullet} && \colorC{\bullet}\\
			\colorA{\bullet} &&&& \colorC{\bullet}\\
			\colorA{\bullet} &&&& \colorC{\bullet}\\
		};
		\draw[colorA] (m-7-1.center) to [out=200, in=160] (m-9-1.center);
		\draw[colorA] (m-9-1.center) to (m-8-1.center);
		\draw[colorA] (m-5-1.center) to [out=200, in=160] (m-7-1.center);
		\draw[colorA] (m-6-1.center) to (m-5-1.center);
		\draw[colorA] (m-6-1.center) to [out=340, in=20] (m-8-1.center);
		%
		\node[colorA, right=4pt] at (m-9-1) {$w_2$};
		\node[colorA, right=6pt] at (m-8-1) {$w_3$};
		\node[colorA, left=6pt] at (m-7-1) {$(w_2)^2$};
		\node[colorA, right=6pt] at (m-6-1) {$w_2w_3$};
		\node[colorA, left=4pt] at (m-5-1) {$(w_3)^2$};
		%
		\draw[colorB] (m-7-3.center) to [out=160, in=200] (m-5-3.center);
		\draw[colorB, dotted] (m-7-3.center) to (m-6-3.center);
		\draw[colorB, dotted] (m-6-3.center) to [out=5, in=185] (m-4-4.center);
		\draw[colorB] (m-5-3.center) to (m-4-3.center);
		\draw[colorB, dotted] (m-5-3.center) to [out=5, in=185] (m-3-4.center);
		\draw[colorB, dotted] (m-4-4.center) to (m-3-4.center);
		\draw[colorB] (m-4-3.center) to [out=5, in=185] (m-2-4.center);
		\draw[colorB, dotted] (m-2-4.center) to (m-1-4.center);
		\draw[colorB, dotted] (m-3-4.center) to [out=5, in=355] (m-1-4.center);
		%
		\node[colorB, right=4pt] at (m-7-3) {$w_2w'_2$};
		\node[colorB, right=4pt] at (m-6-3) {$w_3w'_2 + w_2w'_3$};
		\node[colorB, left=4pt] at (m-5-3) {$w_3w'_3 + (w_2)^2w'_2 + w_2(w'_2)^2$};
		\node[colorB, left=4pt] at (m-4-3) {$(w_2)^2w'_3 + w_3(w'_2)^2$};
		%
		\node[colorB, right=4pt] at (m-4-4) {$\begin{array}{r}
			w_2w_3w'_2 + w_3(w'_2)^2
			+ (w_2)^2w'_3 + w_2w'_2w'_3
		\end{array}$};
		\node[colorB, right=4pt] at (m-3-4) {$\begin{array}{r}
			(w_3)^2w'_2 + w_2w_3w'_3
			+ w_3w'_2w'_3 + w_2(w'_3)^2
		\end{array}$};
		\node[colorB, left=4pt] at (m-2-4) {$\begin{array}{r}
			(w_3)^2w'_3 + (w_2)^2w'_2w'_3
			+ w_2w_3(w'_2)^2 + w_3(w'_3)^2
		\end{array}$};
		\node[colorB, right=4pt] at (m-1-4) {$\begin{array}{r}
			(w_2)^2(w'_3)^2 + (w_3)^2(w'_2)^2
		\end{array}$};
		%
		\draw[colorC] (m-7-5.center) to [out=200, in=160] (m-9-5.center);
		\draw[colorC] (m-9-5.center) to (m-8-5.center);
		\draw[colorC] (m-5-5.center) to [out=200, in=160] (m-7-5.center);
		\draw[colorC] (m-6-5.center) to (m-5-5.center);
		\draw[colorC] (m-6-5.center) to [out=340, in=20] (m-8-5.center);
		%
		\node[colorC, right=4pt] at (m-9-5) {$w'_2$};
		\node[colorC, right=6pt] at (m-8-5) {$w'_3$};
		\node[colorC, left=6pt] at (m-7-5) {$(w'_2)^2$};
		\node[colorC, right=6pt] at (m-6-5) {$w'_2w'_3$};
		\node[colorC, left=4pt] at (m-5-5) {$(w'_3)^2$};
		%
		\node[colorD, right=4pt] at (m-6-6) {(*)};
	\end{tikzpicture}
\end{equation}
To be consistent with the AHSS computation,
it seems that $w_3w_2' + w_2w'_3$ should be modded out
(is it an obvious consequence of the transgression in LSSS...?)
and the remaing part \colorD{(*)} turns out to be
\begin{equation}
	\begin{tikzpicture}[colorD, thick,baseline=(m-2-1.south)]
		\matrix (m) [
			matrix of math nodes,
			row sep= 1em,
			column sep=6em,
		]{
			\bullet\\
		    \bullet\\
			\bullet\\
			\bullet\\
			\bullet\\
		};
		\draw (m-3-1.center) to [out=200, in=160] (m-5-1.center);
		\draw (m-5-1.center) to (m-4-1.center);
		\draw (m-1-1.center) to [out=200, in=160] (m-3-1.center);
		\draw (m-2-1.center) to (m-1-1.center);
		\draw (m-2-1.center) to [out=340, in=20] (m-4-1.center);
		%
		\node[right=4pt] at (m-5-1) {$w_2w'_3 = w_3w'_2$};
		\node[right=6pt] at (m-4-1) {$w_3w'_3$};
		\node[left=6pt] at (m-3-1) {$(w_2)^2w'_3 + w_2w'_2w'_3$};
		\node[right=6pt] at (m-2-1) {$w_2w_3w'_3 + w_3w'_2w'_3$};
		\node[left=4pt] at (m-1-1) {$(w_3)^2w'_3 + w_3(w'_3)^2$};
	\end{tikzpicture}
\end{equation}
and therefore one concludes
\begin{equation}
	\tilde H^\ast(B\left(\tfrac{SO(4)\times SU(2)}{\bZ_2}\right);\bZ_2)_{\leq 5} = \colorA{J[2]} \oplus \colorC{J[2]} \oplus \colorB{\mathcal{A}_1/\!\!/\mathcal{E}_0[4]} \oplus \colorD{J[5]}.
\end{equation}
This leads to the following Adams chart:
\begin{center}
	\begin{sseqdata}[
		name=M,
		Adams grading,
		classes = fill,
		xrange = {0}{5},
		yrange = {0}{3},
		%,x label = {$t-s$}, y label = {$s$}
	]
		\tower[colorB](4,0)
		\class[white](4,0)
		\class[white](4,0)
		\class[colorA](2,0)
		\class[colorC](2,0)
		\tower[colorA](4,1)
		\tower[colorC](4,1)
		\class[colorD](5,0)
	\end{sseqdata}
	\printpage[name = M,page = 2]
\end{center}
and it indeed seems to be compatible with the AHSS computation.
If the above argument (and beliefs) is correct, then the anomaly should be captured by
\begin{equation}
	w_2w'_3\ (= w_3w'_2).
\end{equation}

\subsection{$[SO(4n'_c+2)\times SU(2n_f)]/\bZ_2$}
\label{sec:so(2odd)}

\subsubsection{Cohomology of $BPSO(4n'_c+2)$}
Recalling that the $\bZ_2$ cohomology read off from \cite{KonoMimura1974}
allowed us to determine the $\bZ$ cohomology by using Bockstein SS
(see \texttt{wzw-memo.pdf}, copy later), we had
\begin{equation}
	\renewcommand{\arraystretch}{1.2}
	\begin{array}{c|cccccccccccccccc}
		d & 0 & 1 & 2 & 3 & 4 & 5 & 6 & \cdots \\
		\hline
		\multicolumn{1}{l|}{H^d(BPSO(4n'_c+2);\bZ)} & \bZ & 0 & 0 & \bZ_4 & \bZ & 0 & \bZ_2 & \cdots\\
		\multicolumn{1}{l|}{H^d(BPSO(4n'_c+2);\bZ_2)} & \bZ_2 & 0 & \bZ_2 & \bZ_2 & \bZ_2 & \bZ_2 & \bZ_2^{\oplus 2} & \cdots\\
		\hline
		\text{generator} \,(\bZ_2) & 1 & - & v_2 & y'(1) & (v_2)^2 & y'(2) & (v_2)^3 & \cdots\\
		&&&&&&& y'(1)^2\\
	\end{array}
\end{equation}
and if we assume the suspension $\bar\theta: H^\ast(PSO(4n'_c+2);\bZ_2) \to H^\ast(BPSO(4n'_c+2);\bZ_2)$
to commute with Steenrod squares (which seems to be true according to Nishimoto-san's notes),
then these elements are supposed to be related as
\begin{equation*}
	\begin{array}{ccl}
		\beta_2 v_2 & = & y'(1)\\
		Sq^2 y'(1) & = & y'(2)\\
		Sq^1 y'(2) & = & y'(1)^2\\
	\end{array}
\end{equation*}
where $\beta_f$ is a higher Bockstein operator (reduced to $\bZ_2$ cohomology as a whole),
whose image corresponds to $\bZ_{2^f}$ in integral cohomology.

\subsubsection{Leray-Serre SS (preparatory)}
For the fibration
\begin{equation}
	BSU(2n_f)
	\ \to\ 
	B\left(\tfrac{SO(4n'_c+2)\times SU(2n_f)}{\bZ_2}\right)
	\ \to\ 
	BPSO(4n'_c+2)
\end{equation}
we have
\begin{equation}
	\begin{array}{ccc}
		E_2^{p,q}=H^p\big(BPSO(4n'_c+2);H^q(BSU(2n_f);\bZ)\big) && H^{p+q}(B\left(\tfrac{SO(4n'_c+2)\times SU(2n_f)}{\bZ_2}\right);\bZ)\vspace{4mm}\\
		\begin{array}{c|ccccccccccccc}
			6  & \bZ &&& \ast & \ast && \ast\\
			5  & \cellcolor{lightyellow} & \hphantom{\bZ_2} & \hphantom{\bZ_2} & \hphantom{\bZ_2} & \hphantom{\bZ_2} & \hphantom{\bZ_2} \\
			4  & \bZ & \cellcolor{lightyellow} && \ast & \ast & & \ast\\
			3  &  && \cellcolor{lightyellow} &&&\\
			2  &&  && \cellcolor{lightyellow} &&\\
			1  &  &&  && \cellcolor{lightyellow} &&\\
			0 & \bZ &  && \bZ_4 & \bZ & \cellcolor{lightyellow} & \bZ_2\\
			\hline
			& 0 & 1 & 2 & 3 & 4 & 5 & 6 \\
		\end{array}
		& \longrightarrow & 
		\begin{array}{c|c}
			6  & \bZ\oplus \bZ_2\\
			5  & \cellcolor{lightyellow}\\
			4  & \bZ^{\oplus 2}\\
			3  & \bZ_4\\
			2  & \\
			1  & \\
			0 & \bZ\\
			\hline\\
		\end{array}
	\end{array}
\end{equation}


\subsubsection{Atiyah-Hirzebruch SS}
Having obtained (co)homology groups,
one can fill in the $E^2$-page of the AHSS:
\begin{equation}
	\begin{array}{c}
		E^2_{p,q}=H_p\big(B\left(\tfrac{SO(4n'_c+2)\times SU(2n_f)}{\bZ_2}\right);\Omega_q^{\text{spin}}\big)\vspace{2mm}\\
		\begin{array}{c|c:cccccccccccc}
			6  &&&&&& \\
			5  & \cellcolor{lightyellow} & \hphantom{\bZ_2} & \hphantom{\bZ_2} & \hphantom{\bZ_2} & \hphantom{\bZ_2} & \hphantom{\bZ_2} \\
			4  & \bZ & \cellcolor{lightyellow} & \ast && \ast & \ast & \ast\\
			3  &  && \cellcolor{lightyellow} &&&\\
			2  & \bZ_2 &  & \red{\dbox{\black{$\bZ_2$}}} & \Blue{\dbox{\black{$\bZ_2$}}}\cellcolor{lightyellow} & \ast & \ast & \ast\\
			1  & \bZ_2 && \red{\fbox{\black{$\bZ_2$}}} & \Blue{\fbox{\black{$\bZ_2$}}} & \green{\fbox{\black{\red{\dbox{\black{$\bZ_2^{\oplus 2}$}}}}}}\cellcolor{lightyellow} & \Blue{\dbox{\black{$\ast$}}} & \ast\\
			0 & \bZ &  & \bZ_4 &  & \red{\fbox{\black{$\bZ^{\oplus 2}$}}} & \Blue{\fbox{\black{$\bZ_2$}}}\cellcolor{lightyellow} & \green{\fbox{\black{$\ast$}}}\\
			\hline
			& 0 & 1 & 2 & 3 & 4 & 5 & 6 \\
		\end{array}
	\end{array}
\end{equation}
Based on our belief, 
\red{\fbox{\black{$d^2 : E^2_{4,0} \to E^2_{2,1}$}}} and
\red{\dbox{\black{$d^2 : E^2_{4,1} \to E^2_{2,2}$}}} should be a dual of
\begin{equation}
	\begin{array}{ccccc}
		Sq^2 v_2 & = & (v_2)^2
	\end{array}
\end{equation}
and also
\Blue{\fbox{\black{$d^2 : E^2_{5,0} \to E^2_{3,1}$}}} and
\Blue{\dbox{\black{$d^2 : E^2_{5,1} \to E^2_{3,2}$}}} should be a dual of
\begin{equation}
	\begin{array}{ccccc}
		Sq^2 y'(1) & = & y'(2)\\
	\end{array}
\end{equation}
and finally
\green{\fbox{\black{$d^2 : E^2_{6,0} \to E^2_{4,1}$}}} should be a dual of
\begin{equation}
	\begin{array}{ccccc}
		Sq^2 c_2 & = & c_3
	\end{array}
\end{equation}
then %one can proceed, and 
the would-be-$E_3$-page is given by
\begin{equation}
	\begin{array}{c|c:cccccccccccc}
		6  &&&&&& \\
		5  & \cellcolor{lightyellow} & \hphantom{\bZ_2} & \hphantom{\bZ_2} & \hphantom{\bZ_2} & \hphantom{\bZ_2} & \hphantom{\bZ_2} \\
		4  & \bZ & \cellcolor{lightyellow} & \ast && \ast & \ast & \ast\\
		3  &  && \cellcolor{lightyellow} &&&& \hphantom{\bZ_2}\\
		2  & \bZ_2 &&& \cellcolor{lightyellow} & \ast & \ast & \ast\\
		1  & \bZ_2 &&&& \cellcolor{lightyellow} & \ast & \ast\\
		0 & \bZ &  & \bZ_4 &  & \bZ^{\oplus 2} & \cellcolor{lightyellow} & \ast\\
		\hline
		& 0 & 1 & 2 & 3 & 4 & 5 & 6 \\
	\end{array}
\end{equation}


\section{Pontrjagin square for non-closed cochains}
\label{sec:nonclosedP}
We take the definition of the Pontrjagin square for an element $x\in C^\bullet(-;\bZ_2)$
to be \begin{equation}
\fP(x):= \tilde x \cup\tilde x - \tilde x \cup_1 \delta \tilde x
\end{equation} where $\tilde x $ is an integral cochain which reduces to $x$.
The variation of the Pontrjagin square is then given by 
\begin{equation}
	\label{deltaPontrjagin}
	\begin{array}{ccl}
		\delta \left(\dfrac{1}{2^{m+1}}\fP(x)\right)
		& = & \dfrac{1}{2^{m+1}}\cdot \delta \Big(\tilde x \cup \tilde x - \tilde x \cup_1 \delta \tilde x\Big)\\
		& = & \dfrac{1}{2^{m+1}}\cdot\left[\Big(\delta \tilde x \cup \tilde x + \tilde x \cup \delta \tilde x\Big)
				- \Big(\tilde x \cup \delta \tilde x - \delta \tilde x \cup \tilde x + \delta \tilde x \cup_1 \delta \tilde x\Big)\right]\vspace{2mm}\\
		& = & \dfrac{1}{2^{m+1}}\cdot\Big[2\cdot \delta \tilde x \cup \tilde x - \delta \tilde x \cup_1 \delta \tilde x\Big].\\
	\end{array}
\end{equation}
If $x$ is a $\bZ_{2^m}$-cocycle, 
$\tilde x \in C^\bullet(-;\bZ)$ is a cocycle mod $2^m$ \textit{i.e.}~$\delta \tilde x = 0$ (mod $2^m$),
and the right hand side is $0$ mod 1,
which then means that $\fP(x)$ is a $\bZ_{2^{m+1}}$-cocycle as desired, 
defining a cohomology operation $\fP : H^\bullet(-;\bZ_{2^m}) \to H^{2\bullet}(-;\bZ_{2^{m+1}})$.

However, when $x$ is not a cocycle but merely a cochain,
$\fP(x)$ is also not a cocycle.
For the purpose of this paper, we only need to consider the case when
\begin{equation}
\delta x = \beta y
\end{equation} 
for a cocycle $y\in Z^2(-;\bZ_2)$.
In this case, $x$ and $y$ combine to define a cocycle $z\in Z^2(-;\bZ_4)$,
such that $z = y$ modulo 2 and $\tilde z=2\tilde x$ when $y=0$.
This motivates us to consider the term \begin{equation}
\frac 14 \cdot \frac14 \fP(\tilde z) 
\end{equation} as a replacement for \begin{equation}
\frac14 \fP(\tilde x).
\end{equation}
Using \eqref{deltaPontrjagin}, we find \begin{equation}
\delta\left(
\frac14\cdot\frac14 \fP(\tilde z)
\right)
 = \frac 12 (\frac14\delta\tilde z) \cup \tilde z
 = \frac12 \beta_2 y \cup y
\end{equation} where equalities are all modulo 1.

Here, $\beta_2$ is the higher Bockstein operation
defined for cocycles $y\in H^2(-;\bZ_2)$ known to be liftable to $z\in H^2(-;\bZ_4)$,
by the formula \begin{equation}
\beta_2 y := \frac14 \delta \tilde z
\end{equation} where $\tilde z$ is a lift to an integral cochain of $z$.

In general,
%This problem arises when we consider $SO(2n_c)$ QCD,
%but it turns out that this can be saved somewhat miraculously as follows.
the short exact sequence
\begin{equation*}
	0
	\to
	\bZ
	\overset{\times 2^{f}}{\longrightarrow}
	\bZ
	\overset{p}{\longrightarrow}
	\bZ_{2^f}
	\to
	0
\end{equation*}
buys us a cohomology operation called the higher Bockstein $\beta_f: H^{\bullet}(-;\bZ_{2^f}) \to H^{\bullet+1}(-;\bZ)$, defined for the element $z \in Z^{\bullet}(-;\bZ_{2^f})$ 
by 
\begin{equation*}
	\beta_f (z) := \frac{1}{2^f}
	\delta (\tilde z)
	\in C^\bullet(-;\bZ).
\end{equation*}
where $\tilde z$ is a lift of $z$ as an integral cochain.
Then, $\beta_{n+1} y$  for $y \in H^2(-;\bZ_2)$ is defined only for those $y$ satisfying $\beta_{n}y=0$,
and is the simplest example of a higher cohomology operation.


%Now, let us consider the case of odd $n_c$.
%Here, the cochain $w_2(c) \in C^2(SO(4n'_c+2)\times SU(2n_f); \bZ_2)$
%can be thought of as a mod-2 reduction of $\tilde w_2(c)\in C^2(SO(4n'_c+2)\times SU(2n_f); \bZ_4)$.
%Dividing the $SO\times SU$ by $\bZ_2$, these cochains become non-closed
%\begin{equation*}
%	\delta \tilde w_2(c) = 2 \beta v_2(c)
%\end{equation*}
%where $v_2(c)\in C^2\left(\tfrac{SO(4n'_c+2)\times SU(2n_f)}{\bZ_2}; \bZ_2\right)$.
%\begin{equation*}
%	\begin{array}{ccccccccc}
%		&& w_2 && \tilde w_2 & \longmapsto & v_2\\
%		&& \rotatebox[]{270}{$\in$} && \rotatebox[]{270}{$\in$} && \rotatebox[]{270}{$\in$}\\
%		0
%		& \longrightarrow &
%		C^2(-;\bZ_2)
%		& \overset{i^\ast}{\longrightarrow} &
%		C^2(-;\bZ_4)
%		& \overset{p^\ast}{\longrightarrow} &
%		C^2(-;\bZ_2)
%		& \longrightarrow &
%		0\vspace{1mm}\\
%		&& \downarrow\,\delta && \downarrow\,\delta && \downarrow\vspace{1mm}\\
%		0
%		& \longrightarrow  &
%		C^3(-;\bZ_2)
%		& \longrightarrow &
%		C^3(-;\bZ_4)
%		& \longrightarrow &
%		C^3(-;\bZ_2)
%		& \longrightarrow &
%		0\\
%		&& \rotatebox[]{90}{$\in$} && \rotatebox[]{90}{$\in$}\\
%		&& \delta w_2 && \delta \tilde w_2
%	\end{array}
%\end{equation*}
%Since this implies $\delta \tilde w_2(c) = 0$ or $2$ mod 4
%and furthermore $\delta (2\tilde w_2(c)) = 0$ mod 4,
%one can safely define the Pontrjagin square $\tfrac{1}{8}\fP\big(2\tilde w_2(c)\big)$.\footnote{
%	Be careful that this factor $2$ here is not the usual map sending $\{0,1\}=\bZ_2$ to $\{0,2\}\subset \bZ_4$.
%	This time we are \textit{really} multiplying by $2$.
%}
%Note that this can naively be regarded as $2\cdot \tfrac{1}{4}\fP\big(w_2(c)\big)$ at the integral cochain level.
%While the second term in the last line of \eqref{deltaPontrjagin} together with the overall factor takes value in
%$\tfrac{(2^m)^2}{2^{m+1}}\bZ = 2^{m-1}\bZ = 2\bZ$ and dividing by two does not cause any trouble,
%the first term does as it takes value in $\tfrac{2\cdot 2^m}{2^{m+1}}\bZ = \bZ$.

\if0
Let us take a closer look at the latter.
The long exact sequence of cochain groups implies that $2\tilde w_2(c)$ can be replaced by
$\tilde v_2(c)$, which is a cocycle mod 4.
Then the term of interest is
\begin{equation*}
	\dfrac{1}{8}\cdot 2\cdot 4\beta_2\tilde v_2 \cup \tilde v_2.
\end{equation*}
Therefore, the anomalous variation of the Pontrjagin square term seems to result in
\begin{equation*}
	\delta\left(
		\dfrac{1}{4}\fP(w_2(c))
	\right)
	=
	\dfrac{1}{2}v_2 \beta_2 v_2.
\end{equation*}
\fi

\def\arxivfont{\rm}
\bibliographystyle{ytamsalpha}

\baselineskip=.98\baselineskip
\let\originalthebibliography\thebibliography
\renewcommand\thebibliography[1]{
  \originalthebibliography{#1}
%  \setlength{\parskip}{0pt}
  \setlength{\itemsep}{0pt plus 0.3ex}
}

\bibliography{ref}

\end{document}