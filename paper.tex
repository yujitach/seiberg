\pdfoutput=1
\RequirePackage[T1]{fontenc}
\documentclass[12pt]{article}

\usepackage[height=8.85in,width=6.45in]{geometry}
%\usepackage{showkeys}
\renewcommand{\baselinestretch}{1.05}

\usepackage[utf8]{inputenc}
\usepackage{amsmath}
\usepackage{amssymb}
\usepackage{mathtools}
\numberwithin{equation}{section}
\usepackage{slashed}
\usepackage{braket}
\usepackage[svgnames,psnames]{xcolor}
\usepackage[colorlinks,citecolor=DarkGreen,linkcolor=FireBrick,urlcolor=FireBrick,linktocpage,unicode]{hyperref}
\urlstyle{rm}
%\usepackage[colorlinks,citecolor=black,linkcolor=black]{hyperref}
\usepackage{cite}
\usepackage{graphicx}
\usepackage{tikz}
%\usepackage{tikz-cd}
\newcommand{\tikzmark}[1]{\tikz[remember picture,overlay]\node (#1){};}
\tikzset{>=stealth}

\usepackage{times}
\usepackage{courier}
\usepackage{bm}
\usepackage{subfig}

\usepackage{xcolor}
\usepackage{mdframed}
\newenvironment{claim}{  \begin{mdframed}[linecolor=black!0,backgroundcolor=black!10]\noindent\itshape\ignorespaces}{\end{mdframed}}

\let\originalfigure=\figure
\let\endoriginalfigure=\endfigure

\renewenvironment{figure}[1][]{
  \begin{originalfigure}[#1]
    \begin{mdframed}[linecolor=black!0,backgroundcolor=black!1]
}{
    \end{mdframed}
  \end{originalfigure}
}
%% Comment
\newcommand{\comment}[1]{\textcolor{red}{[#1]}}

\usepackage{ascmac}
\usepackage{spectralsequences}
\usepackage{dashbox}

\DeclareSseqGroup\tower {} {
	\class(0,0)\foreach \i in {1,...,8} {
		\class(0,\i)
		\structline(0,\i-1,-1)(0,\i,-1)
	}
}

\usepackage{xcolor}
\usepackage{colortbl}
\usepackage{arydshln}
\definecolor{lightyellow}{rgb}{1.0, 0.95, 0.7}
\definecolor{blue}{rgb}{0.0, 0.4, 1.0}
\definecolor{Blue}{rgb}{0,0,1}
\definecolor{darkgreen}{rgb}{0.,0.6,0.}
\newcommand*{\red}[1]{\textcolor{red}{#1}}
\newcommand*{\Blue}[1]{\textcolor{Blue}{#1}}
\newcommand*{\blue}[1]{\textcolor{blue}{#1}}
\newcommand*{\green}[1]{\textcolor{darkgreen}{#1}}
\newcommand*{\black}[1]{\textcolor{black}{#1}}

\definecolor{colorA}{rgb}{1,0,0}
\definecolor{colorB}{rgb}{0,0.3,1}
\definecolor{colorC}{rgb}{0.9,0.8,0.2}
\definecolor{colorD}{rgb}{0,0.65,0}

\newcommand*{\colorA}[1]{\textcolor{colorA}{#1}}
\newcommand*{\colorB}[1]{\textcolor{colorB}{#1}}
\newcommand*{\colorC}[1]{\textcolor{colorC}{#1}}
\newcommand*{\colorD}[1]{\textcolor{colorD}{#1}}


%% Yuji's macros
%%list SeibergDual



%% Color
\usepackage{colortbl}
\definecolor{lightyellow}{rgb}{1.0, 0.95, 0.7}
\definecolor{lightblue}{rgb}{0.7, 0.9, 1.0}
\definecolor{lightpink}{rgb}{1.0, 0.85, 0.95}
\definecolor{lightgreen}{rgb}{0.7, 1.0, 0.4}

\def\Nequals#1{$\mathcal{N}{=}#1$}
\def\bZ{\mathbb{Z}}
\def\MF{\mathrm{MF}}
\def\TMF{\mathrm{TMF}}
\def\Tmf{\mathrm{Tmf}}
\def\tmf{\mathrm{tmf}}
\def\tr{\mathop{\mathrm{tr}}}

\def\fP{\mathfrak{P}}

\def\U{\mathrm{U}}
\def\SU{\mathrm{SU}}
\def\SO{\mathrm{SO}}
\def\USp{\mathrm{USp}}
\def\u{\mathfrak{u}}
\def\su{\mathfrak{su}}
\def\so{\mathfrak{so}}
\def\usp{\mathfrak{usp}}
\def\Spin{\mathrm{Spin}}
\def\SL{\mathrm{SL}}
\def\boo{0.0}
\def\xlattice#1#2#3{
\begin{tikzpicture}[scale=.5]
\filldraw[color=black!5!white](-.5,-.5) rectangle (1.5,1.5);
\draw[->] (-1,0) -- (2,0);
\draw[->] (0,-1) -- (0,2);
\foreach \x in {0,1} {
	\foreach \y in {0,1}{
		\pgfmathsetmacro\a{mod(#1 * \x - #2 * \y,2)}
		\ifx\a\boo
			\filldraw[color=#3] (\x,\y) circle (.5em);
		\else
			\filldraw[fill=white,draw=gray] (\x,\y) circle (.5em);
		\fi
	}
}
\end{tikzpicture}
}

\begin{document}



\begin{titlepage}

\begin{flushright}
% IPMU-21-XXXX
\end{flushright}

\vskip 3cm

\begin{center}

{\Large \bfseries Higher symmetries and anomalies \\[1em]
in $\so$ QCD  and \Nequals1 duality}

\vskip 1cm
Yasunori Lee$^1$, Kantaro Ohmori$^2$, and Yuji Tachikawa$^1$
\vskip 1cm

\begin{tabular}{ll}
$^1$ & Kavli Institute for the Physics and Mathematics of the Universe (WPI), \\
& University of Tokyo,  Kashiwa, Chiba 277-8583, Japan\\
$^2$ & Department of Physics, Faculty of Science, \\
& University of Tokyo, Bunkyo, Tokyo 113-0033, Japan
\end{tabular}

\vskip 1cm

\end{center}


\noindent
We study higher symmetries and anomalies of 4d  $\so(2n_c)$ gauge theory with $N_f=2n_f$ flavors.
We find that they depend on the parity of $n_c$ and $n_f$,
on the global form of the gauge group, and the discrete theta angle.
The contribution from the fermions plays a central role in our analysis.
Furthermore, our conclusion applies to \Nequals1 supersymmetric cases as well, and
we see that higher symmetries and anomalies match across the duality 
$\so(2n_c)\leftrightarrow\so(2n_f-n_c+4)$ of Intriligator and Seiberg.


\end{titlepage}

\setcounter{tocdepth}{3}
\tableofcontents

\section{Introduction and summary}
\label{sec:introduction}
Our understanding of the concept of symmetries in quantum field theories has been greatly improved in the last several years.
We now have the concept of $p$-form symmetries acting on $p$-dimensional objects \cite{Gaiotto:2014kfa}.
This concept  gives a unifying point of view to both
ordinary symmetries acting on point operators for $p=0$
and center symmetries of gauge theories acting on Wilson line operators for $p=1$.
In addition, the 't Hooft magnetic flux \cite{tHooft:1979rtg} can now be thought of as a background gauge field for the 1-form center symmetry.
It is also realized more recently that 0-form symmetries and 1-form symmetries can not only coexist in a direct product but also mix in a more intricate manner.
They can have mixed anomalies between them.
They can also combine to form a symmetry structure called 2-groups \cite{Cordova:2018cvg,Benini:2018reh}.

In this paper we study these issues in the case of 4d $\so(N_c)$ gauge theories with 
$N_f$ flavors of fermion fields in vector representation.
Let us quickly recall the 0-form and 1-form symmetries these theories have.

As for the 1-form symmetry, we first need to recall that 
such theories come in three versions, $\Spin$, $\SO_+$ and $\SO_-$,
distinguished by the global form of the gauge group ($\Spin$ vs.~$\SO$)
and by the choice of a discrete theta angle ($\SO_+$ vs.~$\SO_-$) \cite{Aharony:2013hda}\footnote{%
This is when the theories are considered on spin manifolds.
On more general manifolds a further distinction needs to be made \cite{Ang:2019txy}.
For simplicity we only consider spin manifolds in this paper.
}.
They also differ by the nontrivial line operator they possess: 
the $\Spin$ theory has the Wilson line $W$ in the spinor representation,
the $\SO_+$ theory has the 't Hooft line $H$ which is mutually non-local with respect to $W$,
and the $\SO_-$ theory has the dyonic line $D=WH$. 
Furthermore, these line operators are charged under corresponding $\bZ_2$ 1-form symmetries,
which we can all electric, magnetic and dyonic 1-form symmetries.

As for the 0-form symmetry, 
we focus our attention on the $\su(N_f)$ symmetry acting on $N_f$ flavors of matter fields 
in the vector representation.
There can be and definitely are other discrete symmetries, but we will not consider them in this paper for brevity.

The main question is then how the $\bZ_2$ 1-form symmetry and the $\su(N_f)$ 0-form symmetry are related.\footnote{%
A partial answer was given in \cite{Hsin:2020nts}, but the contribution from fermions was not taken into account in that reference.
Our conclusion is therefore somewhat different from theirs.
}
We concentrate on the case when $N_c$ and $N_f$ are both even, $N_c=2n_c$ and $N_f=2n_f$.

Take for example the $Spin(2n_c)$ gauge theory with $2n_f$ flavors,
when $n_c$ is odd.
Take two copies of  the Wilson line $W$ in the spinor representation. 
They form a Wilson line in the vector representation.
This can be screened by a dynamical fermion, which was why $W^2=1$ 
as far as the 1-form symmetry charge was concerned.
Now let us recall that this dynamical fermion transforms nontrivially under $-1\in \SU(2n_f)$.
Therefore, when we take the flavor symmetry into account, $W^2$ is still nontrivial.
As we will recall below,  formally this means that the $\bZ_2$ 1-form symmetry extends the $\SU(2n_f)/\bZ_2$ 0-form symmetry in a nontrivial manner, forming a nontrivial 2-group  $H$\begin{equation}
0\to \bZ_2[1] \to H\to \SU(2n_f)/\bZ_2\to 0,\label{2-group}
\end{equation}
whose Postnikov class is specified by \begin{equation}
\beta v_2 \in H^3(\SU(2n_f)/\bZ_2,\bZ_2),\label{postnikov}
\end{equation}
where $a_2$ is the generator of $H^2(\SU(2n_f)/\bZ_2,\bZ_2)=\bZ_2$ and $\beta$ is the Bockstein.

The $\SO_+$ gauge theory is then obtained by gauging the $\bZ_2$ 1-form symmetry \cite{Kapustin:2014gua}.
The presence of the fermions significantly complicates the analysis.
For the moment let us suppose that we have $N_f$ scalars instead of fermions in the vector representation.
Then, the argument of \cite{Tachikawa:2017gyf} immediately applies, and 
we see that the $\bZ_2$ 1-form symmetry of the $\SO(2n_f)$ theory and the $\SU(2n_f)/\bZ_2$ 0-form flavor symmetry remains a direct product but with a mixed anomaly given by \begin{equation}
2\pi i \frac12 \int B\beta v_2.\label{mixed}
\end{equation}


In the rest of the paper, we will carefully analyze how the $\bZ_2$ 1-form symmetry and the $\SU(2n_f)/\bZ_2$ 0-form symmetry are combined.
The derivation will be detailed in the following, and here we simply summarize the result  in Table~\ref{table:main}.
There, `none' specifies that they remain a direct product without mixed anomaly;
`anomaly' implies that they remain a direct product but with mixed anomaly of the form \eqref{mixed};
and `extension' means that they combine into a 2-group given by \eqref{2-group} with the Postnikov class \eqref{postnikov}.


\begin{table}
\centering
\begin{tabular}{r@{\vphantom{$\Bigm|$}\,}l|ccc}
$(n_c,$&$n_f)$ & $\Spin$ & $\SO_+$ & $\SO_-$\\
\hline
(even,&even) & no\tikzmark{A}ne & no\tikzmark{B}ne & no\tikzmark{C}ne \\
(odd,&even) & exte\tikzmark{P}nsion & ano\tikzmark{Q}maly & exte\tikzmark{R}nsion \\
(even,&odd) & ano\tikzmark{S}maly & exte\tikzmark{T}nsion & exte\tikzmark{U}nsion \\
(odd,&odd) & exte\tikzmark{L}nsion & exte\tikzmark{M}nsion & ano\tikzmark{N}maly 
\end{tabular}
\caption{How the $\bZ_2$ 1-form symmetry and the $\SU(2n_f)/\bZ_2$ 0-form symmetry are combined
in $\so(2n_c)$ QCD.
`none' implies that they remain a direct product without mixed anomaly;
`anomaly' means that they remain a direct product but with mixed anomaly;
and `extension' is when they combine into a nontrivial 2-group. 
The orange lines show how the duality of Intriligator and Seiberg acts on this set of theories.
\label{table:main}}

\tikz[overlay,remember picture]{%
\draw[<->,bend left,color=Orange,line width=1.5] (A.north) to (C.north);
\draw[<->,color=Orange,line width=1.5] (B.north west) .. controls +(80:0.5) ..  (B.north east);
\draw[<->,bend left,color=Orange,line width=1.5] (P.north) to (R.north);
\draw[<->,color=Orange,line width=1.5] (Q.north west) .. controls +(80:0.5) ..  (Q.north east);
\draw[<->,color=Orange,line width=1.5] (S.south) to (N.north);
\draw[<->,color=Orange,line width=1.5] (T.south) to (M.north);
\draw[<->,color=Orange,line width=1.5] (U.south) to (L.north);
}
\end{table}

Our result is equally applicable in the case of \Nequals1 supersymmetric QCD, for which
Intriligator and Seiberg found in \cite{Intriligator:1995id} a duality exchanging $\so(N_c)$ and $\so(N_f-N_c+4)$,
which in our notation sends $n_c$ to $n_c'=n_f-n_c+2$.
In \cite{Aharony:2013hda}, this duality was refined to account for the global form of the gauge group and the discrete theta angle, and it was concluded that $\Spin$ is exchanged with $\SO_-$ while $\SO_+$ maps to itself.
This mapping was checked using supersymmetric localization on $S^3/\bZ_n \times S^1$ in \cite{Razamat:2013opa}.
Our analysis allows us to check this duality by comparing how the 1-form symmetry and the 0-form symmetry are combined in the dual pairs.
We superimposed the action of the duality on our main Table~\ref{table:main}.
It is satisfying to see that the duality action correctly preserves the labels `none', `anomaly' and `extension'.

The rest of the paper is organized as follows ...

\section{2-group structure}
\label{sec:2-group}
Let us first study whether the $\bZ_2$ 1-form symmetry and the $\SU(2n_f)/\bZ_2$ flavor symmetry form a nontrivial 2-group or not. 
This can be found rather physically by studying the line operators. 

\subsection{$\Spin$}
We start by discussing the $\Spin(2n_c)$ gauge theories. 
We first recall that the center of $\Spin(2n_c)$ is $\bZ_2\times \bZ_2$ when $n_c$ is even and $\bZ_4$ when $n_c$ is odd.
This corresponds to the fact that the tensor square of a spinor representation contains the identity representation when $n_c$ is even while it contains the vector representation when $n_c$ is odd.

We now consider the Wilson line $W$ in the spinor representation in the $\Spin(2n_c)$ gauge theory with $2n_f$ fermions in the vector representation.
When $n_c$ is even, $W^2$ contains the identity representation, and therefore we simply have a $\bZ_2$ 1-form symmetry independent of the $\su$ flavor symmetry, and there is nothing to see here.

When $n_c$ is odd, $W^2$ contains the vector representation.
This can be screened by the dynamical fermion, which however carries the fundamental representation of $\su(2n_f)$ flavor symmetry, 
and in particular transforms nontrivially under $-1\in \SU(2n_f)$.
In other words, the flavor Wilson line in the vector representation of $\SU(2n_f)$ can now be considered as the square  of the gauge Wilson line in the spinor representation of $\Spin(2n_c)$.
This means that we have the following extension of groups \begin{equation}
0\to \underbrace{\bZ_2}_{\substack{\text{subgroup of}\\
\text{rep.~of center of $\SU(2n_f)$}}}
\to \bZ_4 
\to \underbrace{\bZ_2}_{\substack{\text{group of gauge Wilson lines }\\
\text{up to screening}}} \to 0.
\label{charge-extension}
\end{equation}

As the groups of charges of $\su(2n_f)$ 0-form symmetry and $\bZ_2$ 1-form symmetry are combined nontrivially, 
the 0-form symmetry group and the 1-form symmetry group are also combined nontrivially.
This can be seen most clearly by considering background fields for the symmetry groups.

The fermion fields are in the vector of $\SO(2n_c)$ and in the fundamental of $\SU(2n_f)$, and therefore is a representation of $G=[\SO(2n_c)\times \SU(2n_f)]/\bZ_2$.
Given a $G$-bundle on a manifold $X$,
there is an $\SO(2n_c)/\bZ_2$ bundle and an $\SU(2n_f)/\bZ_2$ bundle associated to it.
Let us denote by $a_2,v_2\in H^2(X,\bZ_2)$ the obstruction classes controlling whether they lift to $\SO(2n_c)$ and $\SU(2n_f)$ respectively. 
Then we have $a_2=v_2$ for a $G$-bundle.
The flavor Wilson line in the fundamental representation is charged under $-1\in \SU(2n_f)$ in the center,
and $v_2$ can be considered as the background field for this $\bZ_2$ 1-form center symmetry.

Now, without the flavor background,  the background $E\in H^2(X,\bZ_2)$ for the electric $\bZ_2$ one-form symmetry of the $\Spin(2n_c)$ theory sets the Stiefel-Whitney class $w_2\in H^2(X,\bZ_2)$ of the $\SO(2n_c)$ gauge bundle to be $E=w_2$, which controls whether it lifts to a $\Spin(2n_c)$ bundle.
When the flavor background $v_2$ is nontrivial,
the obstruction class $a_2$ controlling the lift from $\SO(2n_c)/\bZ_2$ to $\SO(2n_c)$ is nontrivial.
In this situation when $n_c$ is odd, $w_2$ can no longer be defined as a closed cochain; rather it satisfies $\delta w_2 = \beta a_2$, where $\beta$ is the Bockstein operation,
since together they specify the obstruction class $x_2\in H^2(X,\bZ_4)$ 
controlling the lift from $\SO(2n_c)/\bZ_2=\Spin(2n_c)/\bZ_4$ to $\Spin(2n_c)$.
As $E=w_2$ and $a_2=v_2$, we conclude that the background fields satisfy \begin{equation}
\delta E=\beta v_2.
\label{ba}
\end{equation}
This means that the $\bZ_2$ 1-form symmetry and the $\SU(2n_f)/\bZ_2$ 0-form flavor symmetry form the 2-group $H$ fitting in the sequence \begin{equation}
0\to \bZ_2[1]\to H \to \SU(2n_f)/\bZ_2 \to 0
\label{group-extension}
\end{equation} whose Postnikov class is $\beta v_2 \in H^3(B\SU(2n_f)/\bZ_2,\bZ_2)$.

Before proceeding, we note that having the extension of groups of charges of line operators as in \eqref{charge-extension}
is equivalent to having a nontrivial 2-group extension \eqref{group-extension} 
whose background fields satisfy \eqref{ba}.
Therefore, to find a nontrivial 2-group extension, we can simply study the group of charges of line operators,
which we will carry out for $\SO_\pm$ gauge theories next.

\subsection{$\SO_\pm$}

We would like to study how the magnetic $\bZ_2$ 1-form symmetry of the $\SO(2n_c)_+$ theory is combined with the $\su(n_f)$ flavor symmetry.
According to the discussions in the previous subsection, 
we consider what happens if we take two copies of the 't Hooft line operator $H$ and fuse them.
At the very naive level, $H^2$ can be screened by a dynamical monopole,
but dynamical monopoles can receive flavor and gauge center charges from the fermion zero modes.
To study these issues, it is useful to deform the theories to make them simpler. 

For this purpose, we perform the following steps:
\begin{itemize}
\item We reduce the flavor symmetry from $\su(2n_f)$ to $\usp(2n_f)$. 
The fundamental representation still transforms nontrivially under $-1\in \USp(2n_f)$, which is enough for our purposes.
\item We add an adjoint scalar $\Phi_{[ab]}$ and the interaction $\psi^{ai}_\alpha \psi^{bj}_\beta J_{ij} \Phi_{ab}\epsilon^{\alpha\beta} +cc$, where $J_{[ij]}$ is the constant matrix for the $\usp(2n_f)$ part.
\item We give a vev to $\Phi_{ab}$ to break $\SO(2n_c)$ to $\SO(2)^{n_c}$.
\end{itemize}

The 't Hooft lines in the resulting $\SO(2)^{n_c}$ theory can be labeled by their magnetic charges $(m_1,\ldots,m_{n_c})\in \bZ^{n_c})$.
The dynamical monopoles have the charges in the `adjoint class', which are
in the root lattice $\Lambda$ of $\SO(2n_c)$. 

The group of the magnetic charges of 't Hooft lines up to screening by the dynamical monopoles is then \begin{equation}
\bZ^{n_c}/\Lambda = \bZ_2,
\end{equation}
which agrees with the 1-form symmetry before the deformation.
We now would like to study how this $\bZ_2$ is combined with the flavor/gauge center charge $\bZ_2$.

For this purpose we need to know slightly more details of the dynamical monopoles.
The dynamical monopoles associated to the breaking of $G$ to its Cartan were analyzed in many places, 
e.g.~in \cite{Weinberg:1979zt}.
There, the following was shown.
Let $\phi$ be the scalar vev in the real Cartan subalgebra, $\phi\in \mathfrak{h}\subset \mathfrak{g}$.
This determines the simple roots $\alpha$.
Then you can embed the standard spherically-symmetric 't Hooft-Polyakov monopole and have a monopole solution without additional bosonic moduli. 

Let us say we chose the standard $\phi$ such that the simple roots are $(1,-1,\ldots,0)$,
$(0,1,-1,\ldots,0)$, \ldots, $(0,\ldots,1,-1)$ and $(0,\ldots,1,+1)$,
which we call simple dynamical monopoles.

We now consider the group $\bZ^{n_c}\times \bZ_2$ which combine the magnetic charges $\bZ^{n_c}$ and the flavor/gauge center charge $\bZ_2$.
What we are after is the quotient of $\bZ^{n_c}\times \bZ_2$ 
by the subgroup generated by the charges of simple dynamical monopoles,
which we denote by
$(1,-1,\ldots, 0; q_1)$, $(0,1,-1,\ldots, 0; q_2)$, \ldots,
$(0,\ldots, 1,-1; q_{n_c-1})$ and $(0,\ldots,1,+1; q_{n_c})$, respectively.

Let us determine this quotient.
We do not have to determine $q_1$ to $q_{n_c-1}$. 
We simply use them to rewrite any charge vector $
(m_1,\ldots, m_{n_c-2}, m_{n_c-1},m_{n_c}; q)
$
into the form $
(0,\ldots,0, m, m' ; q').
$
We then have to determine $q_{n-1}$ and $q_{n_c}$.

This reduces the study to the case of $\so(4)\simeq \su(2)_1 \times \su(2)_2$.
The monopoles associated to the simple roots are just 't Hooft-Polyakov monopoles associated to the two factors of $\su(2)$'s.
The vev of the adjoint scalar in this basis can be written as $(a_1,a_2)$, which we assume $a_1>a_2>0$.
The fermion is in the vector representation of $so(4)$.
Under the monopole in $\su(2)_1$, it is a doublet coupled to an adjoint vev of size $a_1$ with bare mass $a_2$,
and similarly for the monopole in $\su(2)_2$.

Now, the explicit analysis in \cite[Sec.~IV]{Callias:1977kg} concerning the number of zero modes in the 't Hooft-Polyakov monopole says that 
a doublet coupled to an adjoint vev of size $a$ with bare mass $\mu$ has
 a zero mode if $|a|>|m|$ and 
 it has no zero mode if $|a|<|m|$.
 
With our assumption $a_1>a_2>0$, this means that the monopole in $\su(2)_1$ has a zero mode while the one in $\su(2)_2$ does not have any.
In our original basis, this means that the monopole with $(0,\ldots,1,-1;q_{n_c-1})$ does not have any zero modes and $q_{n_c-1}=0$,
while the one with $(0,\ldots,1,+1;q_{n_c})$  (which is in $\su(2)_1$) has two zero modes per flavor.

The one-form symmetry group is obtained by dividing $\bZ^2\times \bZ_2$ by the subgroup generated by $(1,-1;0)$ and $(1,+1;q_{n_c})$.
This is $\bZ_2\times \bZ_2$ or $\bZ_4$ depending on whether $q_{n_c}$ is $0$ or $1$.

Let us determine $q_{n_c}$.
We saw two zero modes per flavor; 
this means that there are fermionic zero modes transforming in \begin{equation}
V_{2} \otimes R_{2n_f} 
\end{equation} where $V_2$ is the doublet of $\su(2)_2$ (which is actually broken to $\u(1)$ but keeping $\su(2)_2$ representation is useful in organizing the answer), 
$R_{2n_f}$ is the fundamental of $\usp(2n_f)$,
and we need to impose the reality condition using the pseudoreality of both factors,
so that there are $4n_f$ Majorana fermion in total.

To determine the flavor/gauge center charge of the monopole,
it suffices to consider the case $n_f=1$;
the general case is given by simply multiplying it by $n_f$.
When $n_f=1$, there are $4$ Majorana fermions.
Quantizing them, we find the monopoles in \begin{equation}
V_2 \otimes \mathbf{1} \oplus \mathbf{1}\otimes R_2.
\end{equation}
It has the `vector' charge under $\usp(2)$ flavor symmetry or is a doublet under $\su(2)_2$,
which corresponds to the `vector' charge under $\so(4)$ gauge symmetry.
In either case, they have the flavor/gauge center charge $1 \in \{0,1\}=\bZ_2$.
Therefore we conclude the flavor/gauge center charge $q_{n_c}$ is simply given by $n_f$ mod 2.

Combining the intermediate steps above, we conclude the following: 
for the $\SO(2n_c)_+$ theory,
the group $\bZ_2$ of magnetic charges of 't Hooft lines is extended by the flavor/gauge center symmetry $\bZ_2$ to become $\bZ_4$ when $n_f$ is odd,
while they remain separate when $n_f$ is even.

The analysis of the $\SO(2n_c)_-$ theory is largely the same;
the only difference is that the discrete theta angle gives an additional gauge center charge to the simple dynamical monopole with the magnetic charge $(0,0,\ldots,1,1)$, so that $q_{n_c}=n_f+n_c$ mod 2.
Therefore, we conclude the following:
for the $\SO(2n_c)_-$ theory,
the group $\bZ_2$ of magnetic charges of 't Hooft lines is extended by the flavor/gauge center symmetry $\bZ_2$ to become $\bZ_4$ when $n_f+n_c$ is odd,
while they remain separate when $n_f+n_c$ is even.

The result of the analysis is summarized in Table~\ref{table:2group}.
There, `product' means that the $\bZ_2$ 1-form symmetry and $\SU(2n_f)/\bZ_2$ flavor symmetry are kept separate and form a direct product,
while `extended' means that they form a nontrivial 2-group.
We remark that the nontrivial 2-group is always given by the extension \eqref{group-extension}  whose background fields satisfy \eqref{ba}.

\begin{table}
\centering
\begin{tabular}{r@{\vphantom{$\Bigm|$}\,}l|ccc}
$(n_c,$&$n_f)$ & $\Spin$ & $\SO_+$ & $\SO_-$\\
\hline
(even,&even) & product & product & product \\
(odd,&even) & extended & product & extended \\
(even,&odd) & product & extended & extended \\
(odd,&odd) & extended & extended & product
\end{tabular}
\caption{How the $\bZ_2$ 1-form symmetry and the flavor symmetry $\SU(N_f)/\bZ_2$ are combined
in $\so(2n_c)$ QCD.
`product' means that they form a direct product,
and `extended' means that they form a nontrivial 2-group.
\label{table:2group}}
\end{table}

\section{$\SL(2,\bZ_2)$ action and the anomalies}

In the last section we determined the 2-group structure of the $\so(2n_c)$ gauge theories with $2n_f$ flavors, by studying the group of the charges of line operators. 
In a quantum field theory, any symmetry comes with its (possibly trivial) 't Hooft anomaly.
Here we determine the anomalies of these theories using the $\SL(2,\bZ_2)$ action on the set of quantum field theories (QFTs) with $\bZ_2$ 1-form symmetry.

\subsection{$\SL(2,\bZ_2)$ action}
Let us say that  a four-dimensional spin QFT $Q$ with $\bZ_2$ 1-form symmetry is given.
We denote its partition function on a manifold $X$ by $Z_Q[E]$, 
where we suppress the dependence on $X$ in the notation and $E\in H^2(X,\bZ_2)$ is the background field for the 1-form symmetry.
We then define $SQ$ and $TQ$ to be given by the formula \begin{equation}
Z_{SQ}[B] \propto \sum_{E} (-1)^{\int_X B\cup E} Z_Q{B},\qquad
Z_{TQ}[E]= (-1)^{\int_X \tfrac12 \fP(E)} Z_Q[E].
\end{equation}
We can show that $S^2=T^2=1$ and $(ST)^3=1$, meaning that they generate $\SL(2,\bZ_2)$.
This operation was considered in  \cite{Gaiotto:2014kfa} as an analogue of the $\SL(2,\bZ)$ action on 3d theories with $\U(1)$ symmetry of  \cite{Witten:2003ya} and then further studied in \cite{Bhardwaj:2020ymp}.

\subsection{$\SL(2,\bZ_2)$ action and $\so$ gauge theories}

Importantly, $\Spin(2n_c)$ and $\SO(2n_c)_\pm$  gauge theories with $2n_f$ flavors with the same $n_c$ and $n_f$ form a single orbit under this $\SL(2,\bZ)$ action.
More precisely, we need to make the distinction between $\Spin(2n_c)$ and $T(\Spin(2n_c))$ and  similarly between $\SO(2n_c)_\pm$ and $T\SO(2n_c)_\pm$, respectively.
Here the theories with $T$ prepended are different from the original ones only by its coupling to the background.
Then we have the following chain of actions: \begin{equation}
T\Spin \stackrel{T}{\longleftrightarrow} 
\Spin \stackrel{S}{\longleftrightarrow} 
\SO_+ \stackrel{T}{\longleftrightarrow} 
T\SO_+ \stackrel{S}{\longleftrightarrow} 
T\SO_- \stackrel{T}{\longleftrightarrow} 
\SO_- \stackrel{S}{\longleftrightarrow} .
\end{equation}

\subsection{$\SL(2,\bZ_2)$ actions with extra background}

Let us now study what happens if we perform this $\SL(2,\bZ_2)$ action when the $\bZ_2$ 1-form symmetry in question is part of a larger symmetry group.
So far we have been considering the effect of $\SU(2n_f)/\bZ_2$ flavor symmetry,
but the discussions in the last section shows that at a formal level only the background field $v_2 \in H^2(X,\bZ_2)$ matters, which controls the lift from $\SU(2n_f)/\bZ_2$ to $\SU(2n_f)$.
Let us regard $v_2$ as the background field for a flavor $\bZ_2$ 1-form symmetry.

The combined 1-form symmetry is either $\bZ_2\times \bZ_2$ or $\bZ_4$,
and we perform the $\SL(2,\bZ_2)$ action  by picking a $\bZ_2$ subgroup.
The symmetry and the anomaly of the resulting theory can be determined by a formal argument once the symmetry and the anomaly of the original theory 
and the action of the anomaly-free subgroup to be gauged is given \cite{Tachikawa:2017gyf}.

Let us work at the level of anomalies described at the level of cohomology, since we do not need to deal with anomalies more general than that.
We consider a $d$-dimensional QFT with a symmetry group $G$ with an anomaly specified by a cochain $\alpha\in C^{d+1}(G,\U(1))$. 
We pick a subgroup $H\subset G$ such that $\alpha$ trivializes in it, so that one can find $\mu \in C^d(H,\U(1))$ such that $\delta \mu = \alpha|_H$. 
We then gauge $H$, using $\mu$ as the action.
What determines the symmetry and the anomaly of the gauged theory is the data $(\mu,\alpha)$
Clearly, given $\beta\in C^d(G,\U(1))$, the pair $(\mu,\alpha)$ and the pair $(\mu-\beta,\alpha-\delta\beta)$ should give the same result, since we merely added the countertem $\beta$ as the action.
This allows us to always choose the pair of the form $(0,\alpha')$ equivalent to a given $(\mu,\alpha)$, 
by using $\beta$ to be an arbitrary extension of $\mu$ from $H$ to $G$.
This is convenient in discussing the $\SL(2,\bZ_2)$ action,
since our $S$ operation is defined in the convention that $\mu=0$.

Now, what are the possible choices of $(\mu,\alpha)$ or $(0,\alpha')$ we need to discuss?
Let us first consider $\bZ_2\times \bZ_2$ 1-form symmetry.
As detailed in the Appendix~\ref{sec:bordism}, the only possible anomaly for 4d spin QFTs with this symmetry is \begin{equation}
\alpha=2\pi i \tfrac12  B\beta E  \label{anom}
\end{equation} where $Y$ is the 5d spin manifold and $B,E\in H^2(Y,\bZ_2)$ are the background fields.
Its restriction to $\bZ_2$ 1-form symmetry subgroup is trivially zero,
and then the possible choice of $\mu$ is simply the discrete theta angle \begin{equation}
\mu = 2\pi i \tfrac12  \tfrac12\fP(E).\label{P}
\end{equation}
This $\mu$ can be extended from the $\bZ_2$ subgroup to the entire $\bZ_2\times \bZ_2$ subgroup as a closed cochain.
Therefore, we only have to consider pairs $(0,0)$ and $(0,\alpha)$.

Next, we consider $\bZ_4$ 1-form symmetry.
Again in the Appendix~\ref{sec:bordism}, we show that there is no anomaly for $\bZ_4$ 1-form symmetry.
Therefore we can pick $\alpha=0$. Then the possible choice of $\mu$ for the $\bZ_2$ 1-form subgroup is again the discrete theta angle \eqref{P}.
One difference here is that the discrete theta angle \eqref{P} cannot be extended as a closed cochain to the entire $\bZ_4$ 1-form subgroup.
As discussed in the Appendix~\ref{sec:nonclosedP}, with $\delta E=\beta a_2$,
where $a_2\in H^2(X,\bZ_4/\bZ_2)$,
one finds \begin{equation}
\alpha':=\delta \mu = 2\pi i \tfrac12 a_2 \beta_2 a_2.
\end{equation}
Therefore, the pairs we need to consider are $(0,0)$ and $(\mu,0)\sim (0,\alpha')$. 

\def\Textended{extended$_T$}
Summarizing,  we need to consider the following four choices, namely:
\begin{itemize}
\item For $\bZ_2\times \bZ_2$, the pairs $(0,0)$ and $(0,\alpha)$, which we call `none' and `anomaly'
\item For $\bZ_4$, the pairs $(0,0)$ and $(\mu,0)\sim (0,\alpha')$, which we call `extended' and `\Textended'.
\end{itemize}


Let us now determine how the $\SL(2,\bZ_2)$ action affects these data.
The case `none' is very easy.
The additional $\bZ_2$ factor plays no role, and we find the chain of actions given by \begin{equation}
\text{none} \stackrel{T}{\longleftrightarrow} 
\text{none} \stackrel{S}{\longleftrightarrow} 
\text{none} \stackrel{T}{\longleftrightarrow} 
\text{none} \stackrel{S}{\longleftrightarrow} 
\text{none} \stackrel{T}{\longleftrightarrow} 
\text{none} \stackrel{S}{\longleftrightarrow} 
\label{trivial-chain}.
\end{equation} 

Let us now consider $\bZ_4$ 1-form symmetry and gauge its $\bZ_2$ subgroup.
The case  `extended' was analyzed in \cite{Tachikawa:2017gyf},
the resulting theory has $\bZ_2\times \bZ_2$ 1-form symmetry with the anomaly \eqref{anom}, 
which we decided to call `anomaly'.

The case `\Textended' was analyzed in \cite{Hsin:2020nts},
where it was shown that the gauged theory has again `\Textended'.
Let us quickly recall why this is the case. 
The gauging process involves the term
 \begin{equation}
\exp\left(2\pi i\int_X ( \frac14 \fP(E) +\frac12 B E)\right)
\label{boo}
\end{equation}
where $E$ is the variable to be gauged and $B$ is the newly introduced background field.
When $\bZ_2$ to be gauged is the $\bZ_2$ subgroup of a $\bZ_4$ symmetry,
$E$ is not necessarily closed, but satisfies rather the relation \begin{equation}
\delta E = \beta v_2
\end{equation}
where $v_2$ is the background field for the quotient $\bZ_4/\bZ_2$ 1-form symmetry.
Then the second term in \eqref{boo} is not closed, 
and to even talk about the first term in \eqref{boo}, one first needs to extend the definition of the Pontryagin square $\fP$ to non-closed cochains.

To make  the coupling \eqref{boo}  well-defined, we consider adding a background term to \eqref{boo} so that we have \begin{equation}
\exp\left(2\pi i\int_X ( \frac14 \fP(E) +\frac12 B E + \frac14 \fP(B) )\right)
= 
\exp \left(2\pi i \frac14 \int_X \fP(E+B)\right) .
\end{equation}
This is perfectly well-defined if the newly-introduced background field $B$  also satisfies \begin{equation}
\delta B=\beta v_2.
\end{equation}
This means that, starting from `\Textended', performing $S$ and then $T$, we find a theory with the data `extended'.
Therefore, simply performing $S$ for the theory of the type `\Textended',  one  finds `\Textended'.

Combined, the preceding arguments allows us to  see the chain of actions \begin{equation}
\text{\Textended} \stackrel{T}{\longleftrightarrow} 
\text{extended} \stackrel{S}{\longleftrightarrow} 
\text{anomaly} \stackrel{T}{\longleftrightarrow} 
\text{anomaly} \stackrel{S}{\longleftrightarrow} 
\text{extended}\stackrel{T}{\longleftrightarrow} 
\text{\Textended}\stackrel{S}{\longleftrightarrow}.
\label{nontrivial-chain}
\end{equation}


\subsection{Anomalies from $\SL(2,\bZ_2)$ action} 

\begin{table}
\centering
\[
\begin{array}{r@{\,}l|cccccc}
(n_c,&n_f)  & T\Spin & \Spin & \SO_+ & T\SO_+& T\SO_- & \SO_- \\
\hline
(\text{even},&\text{even}) & \text{none}& \color{Purple}\text{none}& \text{none}& \text{none}& \text{none}& \text{none}\\
(\text{odd},&\text{even}) & \text{\Textended} &\color{Purple}\text{extended} & \text{anomaly} & \text{anomaly} & \text{extended } & \text{\Textended} \\
(\text{even},&\text{odd}) & \text{anomaly} & \color{Purple}\text{anomaly} & \text{extended } & \text{\Textended}  & \text{\Textended} & \text{extended} \\
(\text{odd},&\text{odd}) & \text{extended } & \color{Purple}\text{\Textended} & \text{\Textended} & \text{extended}
 &  \text{anomaly} & \text{anomaly}   \end{array} 
\]
\caption{The symmetry structure of $\so(2n_c)$ QCD with $2n_f$ flavors,
as deduced from the 2-group structures found in Sec.~\ref{sec:2-group}
and from the $\SL(2,\bZ_2)$ action discussed in this section.
The symmetry structure of the $\Spin$ case will be checked independently in the next section.
\label{table:refined}}
\end{table}


Comparing the chains of actions \eqref{trivial-chain} and \eqref{nontrivial-chain} we determined above 
and Table~\ref{table:2group},
we see that the anomaly is automatically determined and 
we obtain the result already presented in Table~\ref{table:main}.
More precisely, we should better distinguish $\Spin(2n_c)$ and $T\Spin$, etc.,
and we find the assignment given in Table~\ref{table:refined}.
As the way we determined the symmetry structures were somewhat indirect,
we confirm the symmetry structures of the $\Spin$ case in the next section
in a different means.

\section{Fermion contribution to anomalies}


\appendix
\section{Bordism group computations}
\label{sec:bordism}
\subsection{$\bZ_4$ 1-form symmetry}
According to \cite[Appendix C.3]{Clement2002} and \cite[Eq.\,(6.3)]{Wan:2018bns}, it seems that we have
\begin{equation}
	\begin{array}{ccc}
		E^2_{p,q}=H_p\big(K(\bZ_4,2);\Omega_q^{\text{spin}}\big) && \widetilde\Omega_{p+q}^{\text{spin}}(K(\bZ_4,2))\vspace{4mm}\\
		\begin{array}{c|c:cccccccccccc}
			6  &&&&&& \\
			5  & \cellcolor{lightyellow} & \hphantom{\bZ_2} & \hphantom{\bZ_2} & \hphantom{\bZ_2} & \hphantom{\bZ_2} & \hphantom{\bZ_2} \\
			4  & \bZ & \cellcolor{lightyellow} & \ast && \ast & \ast & \ast\\
			3  &  && \cellcolor{lightyellow} &&&\\
			2  & \bZ_2 &  & \red{\fbox{\black{$\bZ_2$}}} & \Blue{\fbox{\black{$\bZ_2$}}}\cellcolor{lightyellow} & \ast & \ast & \ast\\
			1  & \bZ_2 && \red{\dbox{\black{$\bZ_2$}}} & \Blue{\dbox{\black{$\bZ_2$}}} & \red{\fbox{\black{$\bZ_2$}}}\cellcolor{lightyellow} & \Blue{\fbox{\black{$\bZ_2$}}} & \ast\\
			0 & \bZ &  & \bZ_4 &  & \red{\dbox{\black{$\bZ_8$}}} & \Blue{\dbox{\black{$\bZ_2$}}} \cellcolor{lightyellow} & \ast\\
			\hline
			& 0 & 1 & 2 & 3 & 4 & 5 & 6 \\
		\end{array}
		& \quad\longrightarrow & 
		\begin{array}{c|c}
			6  & \ast\\
			5  & \cellcolor{lightyellow}\\
			4  & \bZ_4\\
			3  & \\
			2  & \bZ_4\\
			1  & \\
			0 & \\
			\hline\\
		\end{array}
	\end{array}
\end{equation}

The corresponding invariant in 4d is simply \begin{equation}
\exp(2\pi i \frac{p}{4} \int \frac12\fP(a) )
\end{equation}
where $\fP:H^2(-,\bZ_4)\to H^4(-,\bZ_8)$ is the Pontryagin square,
which is even mod 8 on a spin manifold.

\subsection{$\bZ_2\times \bZ_2$ 1-form symmetry}
Exploiting the fact that $K(\bZ_2\times \bZ_2, 2) = K(\bZ_2, 2) \times K(\bZ_2, 2)$,
it seems that we have
\begin{equation}
	\begin{array}{ccc}
		E^2_{p,q}=H_p\big(K(\bZ_2\times \bZ_2,2);\Omega_q^{\text{spin}}\big)
		&& \widetilde\Omega_{p+q}^{\text{spin}}(K(\bZ_2\times \bZ_2,2))\vspace{4mm}\\
		\begin{array}{c|c:cccccccccccc}
			6  &&&&&& \\
			5  & \cellcolor{lightyellow} & \hphantom{\bZ_2} & \hphantom{\bZ_2} & \hphantom{\bZ_2} & \hphantom{\bZ_2} & \hphantom{\bZ_2} \\
			4  & \bZ & \cellcolor{lightyellow} & \ast && \ast & \ast & \ast\\
			3  &  && \cellcolor{lightyellow} &&&\\
			2  & \bZ_2 &  & \red{\fbox{\black{$\bZ_2^{\oplus 2}$}}} & \Blue{\fbox{\black{$\bZ_2^{\oplus 2}$}}}\cellcolor{lightyellow} & \ast & \ast & \ast\\
			1  & \bZ_2 && \red{\dbox{\black{$\bZ_2^{\oplus 2}$}}} & \Blue{\dbox{\black{$\bZ_2^{\oplus 2}$}}} & \green{\fbox{\red{\fbox{\black{$\bZ_2^{\oplus 3}$}}}}}\cellcolor{lightyellow} & \Blue{\fbox{\black{$\bZ_2^{\oplus 6}$}}} & \ast\\
			0 & \bZ &  & \bZ_2^{\oplus 2} &  & \red{\dbox{\black{$\bZ_4^{\oplus 2}\oplus \bZ_2$}}} & \Blue{\dbox{\black{$\bZ_2^{\oplus 3}$}}} \cellcolor{lightyellow} & \green{\fbox{\black{$\ast$}}}\\
			\hline
			& 0 & 1 & 2 & 3 & 4 & 5 & 6 \\
		\end{array}
		& \quad\longrightarrow & 
		\begin{array}{c|c}
			6  & \ast\\
			5  & \bZ_2\cellcolor{lightyellow}\\
			4  & \bZ_2^{\oplus 3}\\
			3  & \\
			2  & \bZ_2^{\oplus 2}\\
			1  & \\
			0 & \\
			\hline\\
		\end{array}
	\end{array}
\end{equation}
The $\bZ$ homology of $K(\bZ_2, 2)$ is again read off from \cite{Clement2002},
while the $\bZ_2$ (co)homology is known \cite{Serre1953} to be
\begin{equation*}
	H^\ast(K(\bZ_2,2);\bZ_2)
	=
	\bZ_2[x_2, Sq^1 x_2, Sq^2Sq^1 x_2, \cdots].
\end{equation*}

The corresponding bordism invariants in 4d are $\fP(a)/2$, $ab$, $\fP(b)/2$, 
and the one in 5d is $a\beta b$.


\section{Pontrjagin square for non-closed cochains}
\label{sec:nonclosedP}
By definition, the variation of the Pontrjagin square $\fP : H^\bullet(-;\bZ_{2^m}) \to H^{2\bullet}(-;\bZ_{2^{m+1}})$ term is
\begin{equation}
	\label{deltaPontrjagin}
	\begin{array}{ccl}
		\delta \left(\dfrac{1}{2^{m+1}}\fP(x)\right)
		& = & \dfrac{1}{2^{m+1}}\cdot \delta \Big(\widetilde x \cup \widetilde x - \widetilde x \cup_1 \delta \widetilde x\Big)\\
		& = & \dfrac{1}{2^{m+1}}\cdot\left[\Big(\delta \widetilde x \cup \widetilde x + \widetilde x \cup \delta \widetilde x\Big)
				- \Big(\widetilde x \cup \delta \widetilde x - \delta \widetilde x \cup \widetilde x + \delta \widetilde x \cup_1 \delta \widetilde x\Big)\right]\vspace{2mm}\\
		& = & \dfrac{1}{2^{m+1}}\cdot\Big[2\cdot \delta \widetilde x \cup \widetilde x - \delta \widetilde x \cup_1 \delta \widetilde x\Big],\\
	\end{array}
\end{equation}
and thus if $x$ were a $\bZ_{2^m}$-cocycle, 
its integral lift $\widetilde x \in C^\bullet(-;\bZ)$ would be a cocycle mod $2^m$ \textit{i.e.}~$\delta \widetilde x = 0$ (mod $2^m$),
and the right hand side would be $0$ mod 1,
which then means that $\fP(x)$ would be a $\bZ_{2^{m+1}}$-cocycle as desired.
However, when $x$ is not a cocycle but merely a cochain,
$\fP(x)$ is also not a cocycle and it is not clear whether this term is well-defined in the first place.

This problem arises when we consider $SO(2n_c)$ QCD,
but it turns out that this can be saved somewhat miraculously as follows.
First, a short exact sequence
\begin{equation*}
	0
	\to
	\bZ_{2^f}
	\overset{\times 2^{f}}{\longrightarrow}
	\bZ_{2^{2f}}
	\overset{p}{\longrightarrow}
	\bZ_{2^f}
	\to
	0
\end{equation*}
buys us cohomology operations called the higher Bockstein $\beta_f: H^{\bullet}(-;\bZ_{2^f}) \to H^{\bullet+1}(-;\bZ_{2^f})$,
and for the element $y \in C^{\bullet}(-;\bZ_{2^f})$ one has
\begin{equation*}
	\delta (p^\ast y)
	=
	2^f \beta_f (y) \in C^\bullet(-;\bZ_{2^{2f}}).
\end{equation*}
Now, let us consider the case of odd $n_c$.
Here, the cochain $w_2(c) \in C^2(SO(4n'_c+2)\times SU(2n_f); \bZ_2)$
can be thought of as a mod-2 reduction of $\widetilde w_2(c)\in C^2(SO(4n'_c+2)\times SU(2n_f); \bZ_4)$.
Dividing the $SO\times SU$ by $\bZ_2$, these cochains become non-closed
\begin{equation*}
	\delta \widetilde w_2(c) = 2 \beta v_2(c)
\end{equation*}
where $v_2(c)=a_2\in C^2\left(\tfrac{SO(4n'_c+2)\times SU(2n_f)}{\bZ_2}; \bZ_2\right)$.
%\begin{equation*}
%	\begin{array}{ccccccccc}
%		&& w_2 && \widetilde w_2 & \longmapsto & v_2\\
%		&& \rotatebox[]{270}{$\in$} && \rotatebox[]{270}{$\in$} && \rotatebox[]{270}{$\in$}\\
%		0
%		& \longrightarrow &
%		C^2(-;\bZ_2)
%		& \overset{i^\ast}{\longrightarrow} &
%		C^2(-;\bZ_4)
%		& \overset{p^\ast}{\longrightarrow} &
%		C^2(-;\bZ_2)
%		& \longrightarrow &
%		0\vspace{1mm}\\
%		&& \downarrow\,\delta && \downarrow\,\delta && \downarrow\vspace{1mm}\\
%		0
%		& \longrightarrow  &
%		C^3(-;\bZ_2)
%		& \longrightarrow &
%		C^3(-;\bZ_4)
%		& \longrightarrow &
%		C^3(-;\bZ_2)
%		& \longrightarrow &
%		0\\
%		&& \rotatebox[]{90}{$\in$} && \rotatebox[]{90}{$\in$}\\
%		&& \delta w_2 && \delta \widetilde w_2
%	\end{array}
%\end{equation*}
Since this implies $\delta \widetilde w_2(c) = 0$ or $2$ mod 4
and furthermore $\delta (2\widetilde w_2(c)) = 0$ mod 4,
one can safely define the Pontrjagin square $\tfrac{1}{8}\fP\big(2\widetilde w_2(c)\big)$.\footnote{
	Be careful that this factor $2$ here is not the usual map sending $\{0,1\}=\bZ_2$ to $\{0,2\}\subset \bZ_4$.
	This time we are \textit{really} multiplying by $2$.
}
Note that this can naively be regarded as $2\cdot \tfrac{1}{4}\fP\big(w_2(c)\big)$ at the integral cochain level.
While the second term in the last line of \eqref{deltaPontrjagin} together with the overall factor takes value in
$\tfrac{(2^m)^2}{2^{m+1}}\bZ = 2^{m-1}\bZ = 2\bZ$ and dividing by two does not cause any trouble,
the first term does as it takes value in $\tfrac{2\cdot 2^m}{2^{m+1}}\bZ = \bZ$.

Let us take a closer look at the latter.
The long exact sequence of cochain groups implies that $2\widetilde w_2(c)$ can be replaced by
$\widetilde v_2(c) = \widetilde a_2$, which is a cocycle mod 4.
Then the term of interest is
\begin{equation*}
	\dfrac{1}{8}\cdot 2\cdot 4\beta_2\widetilde a_2 \cup \widetilde a_2.
\end{equation*}
Therefore, the anomalous variation of the Pontrjagin square term seems to result in
\begin{equation*}
	\delta\left(
		\dfrac{1}{4}\fP(w_2(c))
	\right)
	=
	\dfrac{1}{2}a_2 \beta_2 a_2.
\end{equation*}


\def\arxivfont{\rm}
\bibliographystyle{ytamsalpha}

\baselineskip=.98\baselineskip
\let\originalthebibliography\thebibliography
\renewcommand\thebibliography[1]{
  \originalthebibliography{#1}
%  \setlength{\parskip}{0pt}
  \setlength{\itemsep}{0pt plus 0.3ex}
}

\bibliography{ref}

\end{document}