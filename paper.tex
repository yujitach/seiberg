\pdfoutput=1
\RequirePackage[T1]{fontenc}
\documentclass[12pt]{article}

\usepackage[height=8.85in,width=6.45in]{geometry}
%\usepackage{showkeys}
\renewcommand{\baselinestretch}{1.05}

\usepackage[utf8]{inputenc}
\usepackage{amsmath}
\usepackage{amssymb}
\usepackage{mathtools}
\numberwithin{equation}{section}
\usepackage{slashed}
\usepackage{braket}
\usepackage[svgnames,psnames]{xcolor}
\usepackage[colorlinks,citecolor=DarkGreen,linkcolor=FireBrick,urlcolor=FireBrick,linktocpage,unicode]{hyperref}
\urlstyle{rm}
%\usepackage[colorlinks,citecolor=black,linkcolor=black]{hyperref}
\usepackage{cite}
\usepackage{graphicx}
\usepackage{tikz}
%\usepackage{tikz-cd}
\newcommand{\tikzmark}[1]{\tikz[remember picture,overlay]\node (#1){};}
\tikzset{>=stealth}

\usepackage{times}
\usepackage{courier}
\usepackage{bm}
\usepackage{subfig}

\usepackage{xcolor}
\usepackage{mdframed}
\newenvironment{claim}{  \begin{mdframed}[linecolor=black!0,backgroundcolor=black!10]\noindent\itshape\ignorespaces}{\end{mdframed}}

\let\originalfigure=\figure
\let\endoriginalfigure=\endfigure

\renewenvironment{figure}[1][]{
  \begin{originalfigure}[#1]
    \begin{mdframed}[linecolor=black!0,backgroundcolor=black!1]
}{
    \end{mdframed}
  \end{originalfigure}
}
%% Comment
\newcommand{\comment}[1]{\textcolor{red}{[#1]}}

%% Yuji's macros
%%list SeibergDual



%% Color
\usepackage{colortbl}
\definecolor{lightyellow}{rgb}{1.0, 0.95, 0.7}
\definecolor{lightblue}{rgb}{0.7, 0.9, 1.0}
\definecolor{lightpink}{rgb}{1.0, 0.85, 0.95}
\definecolor{lightgreen}{rgb}{0.7, 1.0, 0.4}

\def\Nequals#1{$\mathcal{N}{=}#1$}
\def\bZ{\mathbb{Z}}
\def\MF{\mathrm{MF}}
\def\TMF{\mathrm{TMF}}
\def\Tmf{\mathrm{Tmf}}
\def\tmf{\mathrm{tmf}}
\def\tr{\mathop{\mathrm{tr}}}

\def\SU{\mathrm{SU}}
\def\SO{\mathrm{SO}}
\def\su{\mathfrak{su}}
\def\so{\mathfrak{so}}
\def\Spin{\mathrm{Spin}}
\def\boo{0.0}
\def\xlattice#1#2#3{
\begin{tikzpicture}[scale=.5]
\filldraw[color=black!5!white](-.5,-.5) rectangle (1.5,1.5);
\draw[->] (-1,0) -- (2,0);
\draw[->] (0,-1) -- (0,2);
\foreach \x in {0,1} {
	\foreach \y in {0,1}{
		\pgfmathsetmacro\a{mod(#1 * \x - #2 * \y,2)}
		\ifx\a\boo
			\filldraw[color=#3] (\x,\y) circle (.5em);
		\else
			\filldraw[fill=white,draw=gray] (\x,\y) circle (.5em);
		\fi
	}
}
\end{tikzpicture}
}

\begin{document}



\begin{titlepage}

\begin{flushright}
% IPMU-21-XXXX
\end{flushright}

\vskip 3cm

\begin{center}

{\Large \bfseries Higher symmetries and anomalies \\[1em]
in $\so$ QCD  and \Nequals1 duality}

\vskip 1cm
Yasunori Lee$^1$, Kantaro Ohmori$^2$, and Yuji Tachikawa$^1$
\vskip 1cm

\begin{tabular}{ll}
$^1$ & Kavli Institute for the Physics and Mathematics of the Universe (WPI), \\
& University of Tokyo,  Kashiwa, Chiba 277-8583, Japan\\
$^2$ & Department of Physics, Faculty of Science, \\
& University of Tokyo, Bunkyo, Tokyo 113-0033, Japan
\end{tabular}

\vskip 1cm

\end{center}


\noindent
We study higher symmetries and anomalies of 4d  $\so(2n_c)$ gauge theory with $N_f=2n_f$ flavors.
We find that they depend on the parity of $n_c$ and $n_f$,
on the global form of the gauge group, and the discrete theta angle.
The contribution from the fermions plays a central role in our analysis.
Furthermore, our conclusion applies to \Nequals1 supersymmetric cases as well, and
we see that higher symmetries and anomalies match across the duality 
$\so(2n_c)\leftrightarrow\so(2n_f-n_c+4)$ of Intriligator and Seiberg.


\end{titlepage}

\setcounter{tocdepth}{3}
\tableofcontents

\section{Introduction and summary}
\label{sec:introduction}
Our understanding of the concept of symmetries in quantum field theories has been greatly improved in the last several years.
We now have the concept of $p$-form symmetries acting on $p$-dimensional objects \cite{Gaiotto:2014kfa}.
This concept  gives a unifying point of view to both
ordinary symmetries acting on point operators for $p=0$
and center symmetries of gauge theories acting on Wilson line operators for $p=1$.
In addition, the 't Hooft magnetic flux \cite{tHooft:1979rtg} can now be thought of as a background gauge field for the 1-form center symmetry.
It is also realized more recently that 0-form symmetries and 1-form symmetries can not only coexist in a direct product but also mix in a more intricate manner.
They can have mixed anomalies between them.
They can also combine to form a symmetry structure called 2-groups \cite{Cordova:2018cvg,Benini:2018reh}.

In this paper we study these issues in the case of 4d $\so(N_c)$ gauge theories with 
$N_f$ flavors of fermion fields in vector representation.
Let us quickly recall the 0-form and 1-form symmetries these theories have.

As for the 1-form symmetry, we first need to recall that 
such theories come in three versions, $\Spin$, $\SO_+$ and $\SO_-$,
distinguished by the global form of the gauge group ($\Spin$ vs.~$\SO$)
and by the choice of a discrete theta angle ($\SO_+$ vs.~$\SO_-$) \cite{Aharony:2013hda}\footnote{%
This is when the theories are considered on spin manifolds.
On more general manifolds a further distinction needs to be made \cite{Ang:2019txy}.
For simplicity we only consider spin manifolds in this paper.
}.
They also differ by the nontrivial line operator they possess: 
the $\Spin$ theory has the Wilson line $W$ in the spinor representation,
the $\SO_+$ theory has the 't Hooft line $H$ which is mutually non-local with respect to $W$,
and the $\SO_-$ theory has the dyonic line $D=WH$. 
Furthermore, these line operators are charged under corresponding $\bZ_2$ 1-form symmetries,
which we can all electric, magnetic and dyonic 1-form symmetries.

As for the 0-form symmetry, 
we focus our attention on the $\su(N_f)$ symmetry acting on $N_f$ flavors of matter fields 
in the vector representation.
There can be and definitely are other discrete symmetries, but we will not consider them in this paper for brevity.

The main question is then how the $\bZ_2$ 1-form symmetry and the $\su(N_f)$ 0-form symmetry are related.\footnote{%
A partial answer was given in \cite{Hsin:2020nts}, but the contribution from fermions was not taken into account in that reference.
Our conclusion is therefore somewhat different from theirs.
}
We concentrate on the case when $N_c$ and $N_f$ are both even, $N_c=2n_c$ and $N_f=2n_f$.

Take for example the $Spin(2n_c)$ gauge theory with $2n_f$ flavors,
when $n_c$ is odd.
Take two copies of  the Wilson line $W$ in the spinor representation. 
They form a Wilson line in the vector representation.
This can be screened by a dynamical fermion, which was why $W^2=1$ 
as far as the 1-form symmetry charge was concerned.
Now let us recall that this dynamical fermion transforms nontrivially under $-1\in \SU(2n_f)$.
Therefore, when we take the flavor symmetry into account, $W^2$ is still nontrivial.
As we will recall below,  formally this means that the $\bZ_2$ 1-form symmetry extends the $\SU(2n_f)/\bZ_2$ 0-form symmetry in a nontrivial manner, forming a nontrivial 2-group  $H$\begin{equation}
0\to \bZ_2[1] \to H\to \SU(2n_f)/\bZ_2\to 0,\label{2-group}
\end{equation}
whose Postnikov class is specified by \begin{equation}
\beta v_2 \in H^3(\SU(2n_f)/\bZ_2,\bZ_2),\label{postnikov}
\end{equation}
where $a_2$ is the generator of $H^2(\SU(2n_f)/\bZ_2,\bZ_2)=\bZ_2$ and $\beta$ is the Bockstein.

The $\SO_+$ gauge theory is then obtained by gauging the $\bZ_2$ 1-form symmetry \cite{Kapustin:2014gua}.
The presence of the fermions significantly complicates the analysis.
For the moment let us suppose that we have $N_f$ scalars instead of fermions in the vector representation.
Then, the argument of \cite{Tachikawa:2017gyf} immediately applies, and 
we see that the $\bZ_2$ 1-form symmetry of the $\SO(2n_f)$ theory and the $\SU(2n_f)/\bZ_2$ 0-form flavor symmetry remains a direct product but with a mixed anomaly given by \begin{equation}
2\pi i \frac12 \int B\beta v_2.\label{mixed}
\end{equation}


In the rest of the paper, we will carefully analyze how the $\bZ_2$ 1-form symmetry and the $\SU(2n_f)/\bZ_2$ 0-form symmetry are combined.
The derivation will be detailed in the following, and here we simply summarize the result  in Table~\ref{table:main}.
There, `none' specifies that they remain a direct product without mixed anomaly;
`anomaly' implies that they remain a direct product but with mixed anomaly of the form \eqref{mixed};
and `extension' means that they combine into a 2-group given by \eqref{2-group} with the Postnikov class \eqref{postnikov}.


\begin{table}
\centering
\begin{tabular}{r@{\vphantom{$\Bigm|$}\,}l|ccc}
$(n_c,$&$n_f)$ & $\Spin$ & $\SO_+$ & $\SO_-$\\
\hline
(even,&even) & no\tikzmark{A}ne & no\tikzmark{B}ne & no\tikzmark{C}ne \\
(odd,&even) & exte\tikzmark{P}nsion & ano\tikzmark{Q}maly & exte\tikzmark{R}nsion \\
(even,&odd) & ano\tikzmark{S}maly & exte\tikzmark{T}nsion & exte\tikzmark{U}nsion \\
(odd,&odd) & exte\tikzmark{L}nsion & exte\tikzmark{M}nsion & ano\tikzmark{N}maly 
\end{tabular}
\caption{How the $\bZ_2$ 1-form symmetry and the $\SU(2n_f)/\bZ_2$ 0-form symmetry are combined
in $\so(2n_c)$ QCD.
`none' implies that they remain a direct product without mixed anomaly;
`anomaly' means that they remain a direct product but with mixed anomaly;
and `extension' is when they combine into a nontrivial 2-group. 
The orange lines show how the duality of Intriligator and Seiberg acts on this set of theories.
\label{table:main}}
\end{table}
\tikz[overlay,remember picture]{%
\draw[<->,bend left,color=Orange,line width=1.5] (A.north) to (C.north);
\draw[<->,color=Orange,line width=1.5] (B.north west) .. controls +(80:0.5) ..  (B.north east);
\draw[<->,bend left,color=Orange,line width=1.5] (P.north) to (R.north);
\draw[<->,color=Orange,line width=1.5] (Q.north west) .. controls +(80:0.5) ..  (Q.north east);
\draw[<->,color=Orange,line width=1.5] (S.south) to (N.north);
\draw[<->,color=Orange,line width=1.5] (T.south) to (M.north);
\draw[<->,color=Orange,line width=1.5] (U.south) to (L.north);
}

Our result is equally applicable in the case of \Nequals1 supersymmetric QCD, for which
Intriligator and Seiberg found in \cite{Intriligator:1995id} a duality exchanging $\so(N_c)$ and $\so(N_f-N_c+4)$,
which in our notation sends $n_c$ to $n_c'=n_f-n_c+2$.
In \cite{Aharony:2013hda}, this duality was refined to account for the global form of the gauge group and the discrete theta angle, and it was concluded that $\Spin$ is exchanged with $\SO_-$ while $\SO_+$ maps to itself.
This mapping was checked using supersymmetric localization on $S^3/\bZ_n \times S^1$ in \cite{Razamat:2013opa}.
Our analysis allows us to check this duality by comparing how the 1-form symmetry and the 0-form symmetry are combined in the dual pairs.
We superimposed the action of the duality on our main Table~\ref{table:main}.
It is satisfying to see that the duality action correctly preserves the labels `none', `anomaly' and `extension'.

The rest of the paper is organized as follows ...

\section{2-group structure}
Let us first study whether the $\bZ_2$ 1-form symmetry and the $\SU(2n_f)/\bZ_2$ flavor symmetry form a nontrivial 2-group or not. 
This can be found rather physically by studying the line operators. 

\subsection{$\Spin$}
We start by discussing the $\Spin(2n_c)$ gauge theories. 
We first recall that the center of $\Spin(2n_c)$ is $\bZ_2\times \bZ_2$ when $n_c$ is even and $\bZ_4$ when $n_c$ is odd.
This corresponds to the fact that the tensor square of a spinor representation contains the identity representation when $n_c$ is even while it contains the vector representation when $n_c$ is odd.

We now consider the Wilson line $W$ in the spinor representation in the $\Spin(2n_c)$ gauge theory with $2n_f$ fermions in the vector representation.
When $n_c$ is even, $W^2$ contains the identity representation, and therefore we simply have a $\bZ_2$ 1-form symmetry independent of the $\su$ flavor symmetry, and there is nothing to see here.

When $n_c$ is odd, $W^2$ contains the vector representation.
This can be screened by the dynamical fermion, which however carries the fundamental representation of $\su(2n_f)$ flavor symmetry, 
and in particular transforms nontrivially under $-1\in \SU(2n_f)$.
In other words, the flavor Wilson line in the vector representation of $\SU(2n_f)$ can now be considered as the square  of the gauge Wilson line in the spinor representation of $\Spin(2n_c)$.
This means that we have the following extension of groups \begin{equation}
0\to \underbrace{\bZ_2}_{\substack{\text{subgroup of}\\
\text{rep.~of center of $\SU(2n_f)$}}}
\to \bZ_4 
\to \underbrace{\bZ_2}_{\substack{\text{group of gauge Wilson lines }\\
\text{up to screening}}} \to 0.
\end{equation}

As the groups of charges of $\su(2n_f)$ 0-form symmetry and $\bZ_2$ 1-form symmetry are combined nontrivially, 
the 0-form symmetry group and the 1-form symmetry group are also combined nontrivially.
This can be seen most clearly by considering background fields for the symmetry groups.

The fermion fields are in the vector of $\SO(2n_c)$ and in the fundamental of $\SU(2n_f)$, and therefore is a representation of $G=[\SO(2n_c)\times \SU(2n_f)]/\bZ_2$.
Given a $G$-bundle on a manifold $X$,
there is an $\SO(2n_c)/\bZ_2$ bundle and an $\SU(2n_f)/\bZ_2$ bundle associated to it.
Let us denote by $a_2,v_2\in H^2(X,\bZ_2)$ the obstruction classes controlling whether they lift to $\SO(2n_c)$ and $\SU(2n_f)$ respectively. 
Then we have $a_2=v_2$ for a $G$-bundle.
The flavor Wilson line in the fundamental representation is charged under $-1\in \SU(2n_f)$ in the center,
and $v_2$ can be considered as the background field for this $\bZ_2$ 1-form center symmetry.

Now, without the flavor background,  the background $E\in H^2(X,\bZ_2)$ for the electric $\bZ_2$ one-form symmetry of the $\Spin(2n_c)$ theory sets the Stiefel-Whitney class $w_2\in H^2(X,\bZ_2)$ of the $\SO(2n_c)$ gauge bundle to be $E=w_2$, which controls whether it lifts to a $\Spin(2n_c)$ bundle.
When the flavor background $v_2$ is nontrivial,
the obstruction class $a_2$ controlling the lift from $\SO(2n_c)/\bZ_2$ to $\SO(2n_c)$ is nontrivial.
In this situation when $n_c$ is odd, $w_2$ can no longer be defined as a closed cochain; rather it satisfies $\delta w_2 = \beta a_2$, where $\beta$ is the Bockstein operation.
As $E=w_2$ and $a_2=v_2$, we conclude that the background fields satisfy \begin{equation}
\delta E=\beta v_2.
\end{equation}
This means that the $\bZ_2$ 1-form symmetry and the $\SU(2n_f)/\bZ_2$ 0-form flavor symmetry form the 2-group $H$ fitting in the sequence \begin{equation}
0\to \bZ_2[1]\to H \to \SU(2n_f)/\bZ_2 \to 0
\end{equation} whose Postnikov class is $\beta v_2 \in H^3(B\SU(2n_f)/\bZ_2,\bZ_2)$.



\def\arxivfont{\rm}
\bibliographystyle{ytamsalpha}
\if0
\baselineskip=.93\baselineskip
\let\originalthebibliography\thebibliography
\renewcommand\thebibliography[1]{
  \originalthebibliography{#1}
%  \setlength{\parskip}{0pt}
  \setlength{\itemsep}{0pt plus 0.3ex}
}
\fi
\bibliography{ref}

\end{document}