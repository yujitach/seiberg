\pdfoutput=1
\RequirePackage[T1]{fontenc}
\documentclass[12pt]{article}

\usepackage[height=8.85in,width=6.45in]{geometry}
%\usepackage{showkeys}
\renewcommand{\baselinestretch}{1.05}

\usepackage[utf8]{inputenc}
\usepackage{amsmath}
\usepackage{amssymb}
\usepackage{mathtools}
\numberwithin{equation}{section}
\usepackage{slashed}
\usepackage{braket}
\usepackage[svgnames,psnames]{xcolor}
\usepackage[colorlinks,citecolor=DarkGreen,linkcolor=FireBrick,urlcolor=FireBrick,linktocpage,unicode]{hyperref}
\urlstyle{rm}
%\usepackage[colorlinks,citecolor=black,linkcolor=black]{hyperref}
\usepackage{cite}
\usepackage{graphicx}
\usepackage{tikz}
%\usepackage{tikz-cd}
\tikzset{>=stealth}

\usepackage{times}
\usepackage{courier}
\usepackage{bm}
\usepackage{subfig}

\usepackage{xcolor}
\usepackage{mdframed}
\newenvironment{claim}{  \begin{mdframed}[linecolor=black!0,backgroundcolor=black!10]\noindent\itshape\ignorespaces}{\end{mdframed}}

\let\originalfigure=\figure
\let\endoriginalfigure=\endfigure

\renewenvironment{figure}[1][]{
  \begin{originalfigure}[#1]
    \begin{mdframed}[linecolor=black!0,backgroundcolor=black!1]
}{
    \end{mdframed}
  \end{originalfigure}
}
%% Comment
\newcommand{\comment}[1]{\textcolor{red}{[#1]}}

%% Yuji's macros
%%list SeibergDual



%% Color
\usepackage{colortbl}
\definecolor{lightyellow}{rgb}{1.0, 0.95, 0.7}
\definecolor{lightblue}{rgb}{0.7, 0.9, 1.0}
\definecolor{lightpink}{rgb}{1.0, 0.85, 0.95}
\definecolor{lightgreen}{rgb}{0.7, 1.0, 0.4}

\def\Nequals#1{$\mathcal{N}{=}#1$}
\def\bZ{\mathbb{Z}}
\def\MF{\mathrm{MF}}
\def\TMF{\mathrm{TMF}}
\def\Tmf{\mathrm{Tmf}}
\def\tmf{\mathrm{tmf}}
\def\tr{\mathop{\mathrm{tr}}}

\def\SU{\mathrm{SU}}
\def\SO{\mathrm{SO}}
\def\so{\mathfrak{so}}
\def\Spin{\mathrm{Spin}}
\def\boo{0.0}
\def\xlattice#1#2#3{
\begin{tikzpicture}[scale=.5]
\filldraw[color=black!5!white](-.5,-.5) rectangle (1.5,1.5);
\draw[->] (-1,0) -- (2,0);
\draw[->] (0,-1) -- (0,2);
\foreach \x in {0,1} {
	\foreach \y in {0,1}{
		\pgfmathsetmacro\a{mod(#1 * \x - #2 * \y,2)}
		\ifx\a\boo
			\filldraw[color=#3] (\x,\y) circle (.5em);
		\else
			\filldraw[fill=white,draw=gray] (\x,\y) circle (.5em);
		\fi
	}
}
\end{tikzpicture}
}

\begin{document}



\begin{titlepage}

\begin{flushright}
% IPMU-21-XXXX
\end{flushright}

\vskip 3cm

\begin{center}

{\Large \bfseries Higher symmetries and anomalies \\[1em]
in $\mathfrak{so}$ QCD  and \Nequals1 duality}

\vskip 1cm
Yasunori Lee$^1$, Kantaro Ohmori$^2$, and Yuji Tachikawa$^1$
\vskip 1cm

\begin{tabular}{ll}
$^1$ & Kavli Institute for the Physics and Mathematics of the Universe (WPI), \\
& University of Tokyo,  Kashiwa, Chiba 277-8583, Japan\\
$^2$ & Department of Physics, Faculty of Science, \\
& University of Tokyo, Bunkyo, Tokyo 113-0033, Japan
\end{tabular}

\vskip 1cm

\end{center}


\noindent
We study higher symmetries and anomalies of 4d  $\mathfrak{so}(2n_c)$ gauge theory with $N_f=2n_f$ flavors.
We find that they depend on the parity of $n_c$ and $n_f$,
on the global form of the gauge group, and the discrete theta angle.
The contribution from the fermions plays a central role in our analysis.
Furthermore, our conclusion applies to \Nequals1 supersymmetric cases as well, and
we see that higher symmetries and anomalies match across the duality 
$\mathfrak{so}(2n_c)\leftrightarrow\mathfrak{so}(2n_f-n_c+4)$ of Intriligator and Seiberg.


\end{titlepage}

\setcounter{tocdepth}{3}
\tableofcontents

\section{Introduction and summary}
\label{sec:introduction}
Our understanding of the concept of symmetries in quantum field theories has been greatly improved in the last several years.
We now have the concept of $p$-form symmetries acting on $p$-dimensional objects \cite{Gaiotto:2014kfa}.
This concept  gives a unifying point of view to both
ordinary symmetries acting on point operators for $p=0$
and center symmetries of gauge theories acting on Wilson line operators for $p=1$.
In addition, the 't Hooft magnetic flux \cite{tHooft:1979rtg} can now be thought of as a background gauge field for the 1-form center symmetry.
It is also realized more recently that 0-form symmetries and 1-form symmetries can not only coexist in a direct product but also mix in a more intricate manner.
They can have mixed anomalies between them.
They can also combine to form a symmetry structure called 2-groups \cite{Cordova:2018cvg,Benini:2018reh}.

In this paper we study these issues in the case of 4d $\so(N_c)$ gauge theories with 
$N_f$ flavors of fermion fields in vector representation.
Let us quickly recall the 0-form and 1-form symmetries these theories have.

As for the 1-form symmetry, we first need to recall that 
such theories come in three versions, $\Spin$, $\SO_+$ and $\SO_-$,
distinguished by the global form of the gauge group ($\Spin$ vs.~$\SO$)
and by the choice of a discrete theta angle ($\SO_+$ vs.~$\SO_-$) \cite{Aharony:2013hda}\footnote{%
This is when the theories are considered on spin manifolds.
On more general manifolds a further distinction needs to be made \cite{Ang:2019txy}.
For simplicity we only consider spin manifolds in this paper.
}.
They also differ by the nontrivial line operator they possess: 
the $\Spin$ theory has the Wilson line $W$ in the spinor representation,
the $\SO_+$ theory has the 't Hooft line $H$ which is mutually non-local with respect to $W$,
and the $\SO_-$ theory has the dyonic line $D=WH$. 
Furthermore, these line operators are charged under corresponding $\bZ_2$ 1-form symmetries,
which we can all electric, magnetic and dyonic 1-form symmetries.

As for the 0-form symmetry, 
we focus our attention on the $\SU(N_f)$ symmetry acting on $N_f$ flavors of matter fields 
in the vector representation.
There can be and definitely are other discrete symmetries, but we will not consider them in this paper for brevity.

Our main question is then how the $\bZ_2$ 1-form symmetry and the $\SU(N_f)$ 0-form symmetry are related.\footnote{%
A partial answer was given in \cite{Hsin:2020nts}, but the contribution from fermions was not taken into account in that reference.
}
We concentrate on the case when $N_c$ and $N_f$ are both even, $N_c=2n_c$ and $N_f=2n_f$.

\section{Charges of line operators in $\so$ QCD}



\def\arxivfont{\rm}
\bibliographystyle{ytamsalpha}
\if0
\baselineskip=.93\baselineskip
\let\originalthebibliography\thebibliography
\renewcommand\thebibliography[1]{
  \originalthebibliography{#1}
%  \setlength{\parskip}{0pt}
  \setlength{\itemsep}{0pt plus 0.3ex}
}
\fi
\bibliography{ref}

\end{document}