\RequirePackage[T1]{fontenc}
\documentclass[12pt]{article}

\usepackage[height=8.85in,width=6.45in]{geometry}
%\usepackage{showkeys}
\renewcommand{\baselinestretch}{1.08}

\usepackage[utf8]{inputenc}
\usepackage{amsmath}
\usepackage{amssymb}
\usepackage{mathtools}
\numberwithin{equation}{section}
\usepackage{slashed}
\usepackage{braket}
\usepackage[svgnames]{xcolor}
\usepackage[colorlinks,citecolor=DarkGreen,linkcolor=FireBrick,urlcolor=FireBrick,linktocpage,unicode]{hyperref}
\urlstyle{rm}
\usepackage{cite}
\usepackage{graphicx}
\usepackage{tikz}
\usepackage{tikz-cd}
\usepackage{times}
\usepackage{courier}
\usepackage{bm}
\usepackage{subfig}

\usepackage{xcolor}
\usepackage{mdframed}
\newenvironment{claim}{  \begin{mdframed}[linecolor=black!0,backgroundcolor=black!10]\noindent\itshape\ignorespaces}{\end{mdframed}}

\let\originalfigure=\figure
\let\endoriginalfigure=\endfigure

\renewenvironment{figure}[1][]{
  \begin{originalfigure}[#1]
    \begin{mdframed}[linecolor=black!0,backgroundcolor=black!1]
}{
    \end{mdframed}
  \end{originalfigure}
}
%% Comment
\newcommand{\comment}[1]{\textcolor{red}{[#1]}}

%% Yuji's macros
%%list SeibergDual

\def\bZ{\mathbb{Z}}
\def\diag{\mathop{\mathrm{diag}}}

\begin{document}

\if0
\begin{titlepage}

\begin{flushright}
IPMU-18-????
\end{flushright}

\vskip 3cm

\begin{center}

{\Large \bfseries Title}


\vskip 1cm
Yuji Tachikawa and friends
\vskip 1cm

\begin{tabular}{ll}
 & Kavli Institute for the Physics and Mathematics of the Universe, \\
& University of Tokyo,  Kashiwa, Chiba 277-8583, Japan
\end{tabular}


\vskip 1cm

\end{center}


\noindent
Abstract comes here.

\end{titlepage}

\setcounter{tocdepth}{2}
\tableofcontents

\newpage

\fi

This file is to collect various notes on our project on the higher symmetry and Seiberg duality.

\section{Explicit configurations detecting anomalies}

Here we describe geometries detecting $\int_{M_5} B\beta E$, $\int_{M_4} B\beta w_2$, etc.
All cohomologies in this section is $\bZ_2$-valued.



\paragraph{Klein bottle:}

We start from the Klein bottle $K$ as a nontrivial $S^1$ bundle over $S^1$.
Let us denote by $a$ the Poincar\'e dual to the fiber $S^1$,
and $t$ the Poincar\'e dual to the base $S^1$.

We have $\beta a=ta$, since $\int_K \beta a = \int_K w_1 a$.

\paragraph{$T^4$ bundle over $S^1$:}

We now consider a $T^4$ bundle over $S^1$. 
We denote four directions of $T^4$ as $1$, $2$, $3$ and $4$,
and we let the directions $1$ and $3$ to flip the orientation when we go around $S^1$.
We let $a_{1,2,3,4}\in H^1(T^4)$ be the dual basis to the $S^1$ along four directions.

We now take $B=a_1a_2$ and $E=a_3 a_4$.
Then $\beta B=tB$ and $\beta E=tE$, and $\int B\beta E=1$.

\paragraph{Realizing as $SO(3)$ bundles}

We now look for $SO(3)$ bundles realizing these $B$ and $E$ as $w_2$ in this $T^4$ bundle over $S^1$.

We note that an $SO(3)$ bundle over $T^2$ with two commuting holonomies around two directions \[
R_x = \diag(+1,-1,-1),\qquad
R_y = \diag(+1,-1,-1)
\] has a nontrivial $w_2$, since their lift to $SU(2)$ is given by $i\sigma_x$ and $i\sigma_y$ which anticommute.

Luckily, these $R_x$ and $R_y$ are of order two, so we can put it over our $T^4$ bundle. Done.

\section{Anomaly of trifundamental}

Lee-kun's computation says that \[
(D\Omega^\text{spin})^6 (B[SU(2)^3/\bZ_2^2]) = \bZ_2 
\] generated by \[
\int w_2 \beta w_2'.
\]

We would like to know if a trifundamental fermion has this anomaly.

To see this, we need to compute the eta invariant under an explicit configuration where $\int w_2 \beta w_2'=1$. 
Such a configuration is constructed above. 
Let us first find an explicit configuration of $SU(2)^{(1)}\times SU(2)^{(2)} \times SU(2)^{(3)}$ commuting up to $\bZ_2\times \bZ_2$, which is generated by $(-1,-1,+1)$ and $(-1,+1,-1)$.
So we just have to choose, say, \begin{align}
\text{holonomy around direction 1} &= (i\sigma_x, i\sigma_x, 1), \\
\text{holonomy around direction 2} &= (i\sigma_y, i\sigma_y, 1), \\
\text{holonomy around direction 3} &= (i\sigma_x, 1,i\sigma_x), \\
\text{holonomy around direction 4} &= (i\sigma_y, 1,i\sigma_y).
\end{align}

Furthermore, the spinor on our $T^4$ bundle over $S^1$ is glued around $S^1$ via the action of $\Gamma_1\Gamma_3$, 
since we flip the directions $1$ and $3$.

These are enough data to construct the fermion bundle over our $T^4$ bundle over $S^1$. 
Since the Euclidean spinor in 5d is pseudoreal, and our trifundamental is also pseudoreal,
the tensor product is strictly real.
The Dirac operator is therefore a real antisymmetric matrix, and the eigenvalues come in pairs $\pm \lambda$
except the zero modes.
Therefore in our case the eta invariant reduces to the mod-2 index,
and we just have to count the zero modes.



\bibliographystyle{ytphys}
\baselineskip=.95\baselineskip
\bibliography{ref}

\end{document}